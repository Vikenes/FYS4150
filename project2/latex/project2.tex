

\documentclass[english,notitlepage,nofootinbib]{revtex4-1}  % defines the basic parameters of the document
%For preview: skriv i terminal: latexmk -pdf -pvc filnavn



% if you want a single-column, remove reprint

% allows special characters (including æøå)
\usepackage[utf8]{inputenc}
%\usepackage[english]{babel}

%% note that you may need to download some of these packages manually, it depends on your setup.
%% I recommend downloading TeXMaker, because it includes a large library of the most common packages.

\usepackage{physics,amssymb}  % mathematical symbols (physics imports amsmath)
\include{amsmath}
\usepackage{graphicx}         % include graphics such as plots
\usepackage{xcolor}           % set colors
\usepackage{hyperref}         % automagic cross-referencing (this is GODLIKE)
\usepackage{listings}         % display code
\usepackage{subfigure}        % imports a lot of cool and useful figure commands
\usepackage{float}
%\usepackage[section]{placeins}
\usepackage{algorithm}
\usepackage[noend]{algpseudocode}
\usepackage{subfigure}
\usepackage{tikz}
\usetikzlibrary{quantikz}
% defines the color of hyperref objects
% Blending two colors:  blue!80!black  =  80% blue and 20% black
\hypersetup{ % this is just my personal choice, feel free to change things
    colorlinks,
    linkcolor={red!50!black},
    citecolor={blue!50!black},
    urlcolor={blue!80!black}}

%C++
\newcommand{\CC}{C\nolinebreak\hspace{-.05em}\raisebox{.4ex}{\tiny\bf +}\nolinebreak\hspace{-.10em}\raisebox{.4ex}{\tiny\bf +}}
%   NEW COMMANDS
\renewcommand{\vec}{\mathbf}
\newcommand{\transpose}{^\intercal} % transpose

\graphicspath{{../output/plots/}} % path for figures

\begin{document}

\title{Project 2 FYS4150}      % self-explanatory
\author{Vetle Vikenes, Johan Mylius Kroken \& Nanna Bryne}          % self-explanatory
\date{\today}                             % self-explanatory
\noaffiliation                            % ignore this, but keep it.


\maketitle 
    
The code is available on GitHUb at \url{https://github.com/Vikenes/FYS4150}.
    

\section*{Introduction}

To describe a one-dimensional buckling beam, we have the second order differential equation 

\begin{align}\label{eq:unscaled_eq}
    \gamma \dv{u}{x} = -F u(x), \quad x\in [0,L]
\end{align}

with $u(0)=u(L)=0$. \dots

\section*{Problem 1}

We define $\hat{x}\equiv x/L$. Now $\dv*[2]{\hat{x}}{x}=L^{-2}$ and we can rewrite eq. \eqref{eq:unscaled_eq}.

\begin{align*}
    \gamma \dv[2]{u}{\hat{x}} \dv{\hat{x}}{x} & = -Fu(x) \\
    \frac{\gamma}{L^2} \dv[2]{u}{\hat{x}} &= -Fu(\hat{x}) \\
    \dv[2]{u}{\hat{x}} &= - \frac{FL^2}{\gamma}
\end{align*}

Letting $\lambda \equiv FL^2/\gamma$ yields

\begin{align}\label{p1_scaled_eq}
    \dv[2]{u}{\hat{x}} = -\lambda u(\hat{x}).
\end{align}

\section*{Problem 2}

In order to make sure that we can set up the tridiagonal $N\cross N$ matrix $A$ correctly, we write a short program in \CC\, that (1) defines $A$ for $N=6$, (2) solves $A\vec{v} = \lambda \vec{v}$ using the Armadillo library and (3) compares the solution to the analyical result.


\section*{Problem 3}

\subsection*{a)}

In \CC, we write a function that identifies the largest off-diagonal element in absolute value in a symmetric Armadillo matrix and notes the matrix indices (in the upper tiangle) of this element.

\subsection*{b)}

We test the function on the matrix

\begin{align*}
    \begin{pmatrix}
        1 & 0 & 0 & 0.5 \\
        0 & 1 & -0.7 & 0 \\
        0 & -0.7 & 1 & 0 \\
        0.5 & 0 & 0 & 1
    \end{pmatrix}.
\end{align*}

In particular, we make sure that the \CC-function yields the maximum value 0.7 at row $k=1$ and column $l=2$.

\section*{Problem 4}

\subsection*{a)}
In order to implement the Jacobi rotation algorithm we write two separate algorithms: Jacobi rotate, and Jacobi eigensolver:
\begin{algorithm}[H]
    \caption{Jacobi rotation}\label{algo:p4_jacobi_rotation}
    \begin{algorithmic}
        \If{$A_{kl}^m = 0$}
            \State $c=1$
            \State $s=0$
            \State $t=0$
        \Else 
            \State $\tau = (A_{ll}^m-A_{kk}^m) / (2A_{kl}^m)$
            \If{$\tau > 0$}
                \State $t= 1/(\tau + \sqrt{1+\tau^2})$
            \Else
                \State $t = -1/(-\tau + \sqrt{1+\tau^2})$
            \EndIf
            \State $c=1/(\sqrt{1+t^2})$
            \State $s=ct$
        \EndIf
        \State $A_{kk}^{m+1} = c^2A_{kk}^m - 2csA_{kl}^m + s^2A_{ll}^m$
        \State $A_{ll}^{m+1} = c^2A_{ll}^m + 2csA_{kl}^m + s^2A_{kk}^m$
        \State $A_{kl}^{m+1} = 0$
        \State $A_{lk}^{m+1} = 0$
        \For{$i=0,1,2,\dots,N-1$}
            \If{$i\neq k \wedge i \neq l$}
                \State $A_{ik}^{m+1} = cA_{ik}^m -sA_{il} $
                \State $A_{ki}^{m+1} = A_{ik}^{m+1}$
                \State $A_{il}^{m+1} = cA_{il}^m + sA_{ik}^m$
                \State $A_{li}^{m+1} = A_{il}^{m+1}$
            \EndIf
            \State $R_{ik}^{m+1} = cR_{ik} - sR_{il}$
            \State $R_{il}^{m+1} = cR_{il} + sR_{ik}$
        \EndFor
    \end{algorithmic}
\end{algorithm}
\subsection*{b)}



\section*{Problem 5}

\subsection*{a)}

Consider our symmetric, tridiagonal matrix $A \in \mathbb{R}^{N\cross N}$ with signature $(a,d,a)$. Let $M$ denote the number of transformations needed for a Jacobi rotation algorithm to converge. That is, $M$ represents the number of iterations in the Jacobi rotation algorithm needed for the transformed matrix $A'$ to be similar enough\footnote{Choosing the number $\varepsilon=10^{-8}$ to be \textit{close enough} to zero.} to a diagonal matrix.

For $N=2,3,\dots,100$, we run the Jacobi eigensolver and save the corresponding integer $M$. The result is plotted in Figure \ref{fig:p5_transformations_per_N}.


\begin{figure}[h!]
    \centering
    \includegraphics[width=0.9\textwidth]{jacobi_comparison.pdf}
    \caption{The number of iterations $M$ as function of the matrix size $N$. NEED UPDATE IN PLOT.}\label{fig:p5_transformations_per_N}
\end{figure}


\subsection*{b)}
\section*{Problem 6}

\subsection*{a)}

For $n=10$ steps, i.e. $n+1=11$ points $\hat{x}_i$ and $A\in\mathbb{R}^{(n-1)\cross(n-1)} =\mathbb{R}^{9\cross 9}$, we solve the eigenvalue problem $A\vec{v} = \lambda \vec{v}$ using the Jacobi eigensolver. In addition, we solve the same problem with the analytic expressions for $\lambda^{(i)}$ and $\vec{v}^{(i)}$. This yields two versions of the normalised vectors $\vec{v}^{(i)}$, and for some $i$'s these are counter-oriented. When we encounter this situation, the issue is solved by forcing the result from the Jacobi algorithm $\vec{v}^{(i)} \rightarrow - \vec{v}^{(i)}$.

The three resulting eigenvectors $\vec{v}^{(1)}$, $\vec{v}^{(2)}$ and $\vec{v}^{(3)}$ corresponding to eigenvalues $\lambda^{(1)}$, $\lambda^{(2)}$ and $\lambda^{(3)}$ s.t. $\lambda^{(i)}<\lambda^{(j)}$ for $i<j$, are plotted in Figure \ref{fig:p6_solution_10steps}. The vectors are extrented with the boundary points, i.e. $v_0^{(i)}=v_0 = 0$ and $v_n^{(i)}=v_n=0$ for all $i$.

% figure or Figure?

\begin{figure}[h!]
    \centering
    \includegraphics[width=0.9\textwidth]{solution_10steps.pdf}
    \caption{The first three eigenvectors $\vec{v}^{(i)}$ respectively corresponding to the three lowest eigenvalues $\lambda^{(i)}$ computed with the Jacobi eigensolver using $n=10$ discretisation steps. The white overplotted graphs are the predictions from the analytic expression.}\label{fig:p6_solution_10steps}
\end{figure}

\subsection*{b)}

We do exactly the same as in \textbf{a)}, only now we use $n=100$ discretisation steps. The resulting plot is presented in Figure \ref{fig:p6_solution100steps}.

\begin{figure}[h!]
    \centering
    \includegraphics[width=0.9\textwidth]{solution_100steps.pdf}
    \caption{The first three eigenvectors $\vec{v}^{(i)}$ respectively corresponding to the three lowest eigenvalues $\lambda^{(i)}$ computed with the Jacobi eigensolver using $n=100$ discretisation steps. The white overplotted graphs are the predictions from the analytic expression.}\label{fig:p6_solution100steps}
\end{figure}

   
\end{document}

























\section*{Problem 2}
We write equations using the LaTeX \texttt{equation} (or \texttt{align}) environments. Here is an equation with numbering
\begin{equation}\label{eq:newton}
    \vb{F} = \dv{\vb{p}}{t},
\end{equation}
and here is one without numbering:
\begin{equation*}
\oint_C \vb{F}\cdot \dd \vb{r} = 0.
\end{equation*}
Sometimes it is useful to refer back to a previous equation, like we're demonstrating here for equation \ref{eq:newton}.

We can include figures using the \texttt{figure} environment. Whenever we include a figure or table, we \textit{must} make sure to actually refer to it in the main text, e.g.\ something like this: ``In figure \ref{fig:rel_err} we show \ldots''. 
\begin{figure}%[h!]
    \centering %Centers the figure
    \includegraphics[scale=0.55]{imgs/rel_err.pdf} %Imports the figure.
    \caption{Write a descriptive caption here that explains the content of the figure. Note the font size for the axis labels and ticks --- the size should approximately match the document font size.}
    \label{fig:rel_err}
\end{figure}
Also, note the LaTeX code we used to get correct quotation marks in the previous sentence. (Simply using the \texttt{"} key on your keyboard will give the wrong result.) Figures should preferably be vector graphics (e.g.\ a \texttt{.pdf} file) rather than raster graphics (e.g.\ a \texttt{.png} file).

By the way, don't worry too much about where LaTeX decides to place your figures and tables --- LaTeX knows more than we do about proper document layout. As long as you label all your figures and tables and refer to them in the text, it's all good. Of course, in some cases it can be worth trying to force a specific placement, to avoid the figure/table appearing many pages away from the main text discussing it, but this isn't something you should spend time on until the very end of the writing process.


Next up is a table, created using the \texttt{table} and \texttt{tabular} environments. We refer to it by table \ref{tab:output_table}.
\begin{table}%[h!]
    \centering
    \caption{Write a descriptive caption here, explaining the content of your table.}
    \begin{tabular}{c@{\hspace{1cm}} c}
        \hline
        Number of points & Output \\
        \hline
        10 &  0.3086\\
        100 &  0.2550\\
        \hline
    \end{tabular}\label{tab:output_table}
\end{table}

Finally, we can list algorithms by using the \texttt{algorithm} environment, as demonstrated here for algorithm \ref{algo:midpoint_rule}.
\begin{algorithm}[H]
    \caption{Some algorithm}\label{algo:midpoint_rule}
    \begin{algorithmic}
        \State Some maths, e.g $f(x) = x^2$.  \Comment{Here's a comment}
        \For{$i = 0, 1, ..., n-1$}
        \State Do something here 
        \EndFor
        \While{Some condition}
        \State Do something more here 
        \EndWhile
        \State Maybe even some more math here, e.g $\int_0^1 f(x) \dd x$
    \end{algorithmic}
\end{algorithm}