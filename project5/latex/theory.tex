
\section{Theory}\label{sec:theory}
    For a single, non-relativistic particle with mass $m_\mathrm{P}$ in a two-dimensional potential $\mathcal{V}(t, \vec{x})$, the Schrödinger equation reads\footnote{In position space, that is.}
    \begin{equation}\label{eq:2d_schrodinger}
        \im \hbar \pdv{}{t}\Psi(t, \vec{x}) = -\frac{\hbar^2}{2m_\mathrm{P}}\nabla^{\!2} \Psi(t, \vec{x})  + \mathcal{V}(t, \vec{x}) \Psi(t, \vec{x}).
    \end{equation}
    For a set of initial and boundary conditions, the partial differential equation (PDE) describes the temporal and spatial evolution of the complex-valued (wave) function $\Psi(t, \vec{x})$ related to the quantum state of the aforementioned particle. In such a case, at a time $t$, the probability density for an experimentalist to locate the particle at $\vec{x}$ is
    \begin{equation}\label{eq:2d_bornrule}
        P(\vec{x}; \, t) = \abs{\Psi(t, \vec{x})}^2 = \Psi^{\!*}(t, \vec{x}) \Psi(t, \vec{x}),
    \end{equation} 
    originating from the Born rule.

    In this report we will consider a dimensionless time-independent potential, i.e. we let $\mathcal{V}(t, \vec{x}) \to v(\vec{x})$. The specifics of the scaling do not concern us here, and we simply rewrite equation \eqref{eq:2d_schrodinger} to the dimensionless equation
    \begin{equation}\label{eq:2d_schrodinger_dimless}
        \im \pdv{}{t}u(t, \vec{x}) = - \nabla^{\!2}u(t, \vec{x})+ v(\vec{x})u(t, \vec{x}),
    \end{equation}
    where we substituted $\Psi(t, \vec{x}) \to u(t, \vec{x})$. In equation \eqref{eq:2d_schrodinger_dimless} all variables are dimensionless. When demanding the proper normalisation on $u(t,\vec{x})$, it follows that the Born rule now takes the form of
    \begin{equation}\label{eq:2d_bornrule_dimless}
        p(\vec{x}; \, t) = \abs{u(t, \vec{x})}^2 = u^{\!*} (t, \vec{x})u(t, \vec{x}).
    \end{equation}

    In this report we will refer to $u(t, \vec{x})$ as the \textit{wave function}, which is the solution of the dimensionless Schrödinger equation from equation~\eqref{eq:2d_schrodinger_dimless}. The aim is to solve for $u(t,\vec{x})$ in a confined space, with and without slits, and simulate the time evolution of Gaussian wave packets in this space. We will focus on solving this equation as a regular two-dimensional partial differential equation, and not delve much into details of quantum mechanics. As for the probabilities we will use both the terms probability,  probability distribution and probability density to describe the quantity represented by equation~\eqref{eq:2d_bornrule_dimless}\footnote{``Probability density'' will technically be wrong since we solve for a two-dimensional lattice where each lattice point represent a spatial area of $h^2$, and the probability measured is the probability that the particle is confined within this are. However, we will use the terms interchangeably, and you are now warned.}.