
\section{Methods}\label{sec:methods}

We will use the Schrödinger equation to simulate a particle inside a two-dimensional quadratic box of size $L^2$. The use of periodic boundary conditions gives rise to the confinement of the particle \Nanna{or something like that}. We will place two or more walls inside the box to create one or more slits. The specifics of such configurations is elaborated in section \ref{sec:method:slit_config}. A position inside if the box is called $\vec{x} =(x, y)$ and a time $t\in$, all unit less. Now $x, y \in [0,\,L]$ and $t\in[0,\,T]$ if we let $T$ be the time we simulate for. In this paper, we will use $L=1$.

\subsection{Numerical scheme}

    \Nanna{How to solve etc.} 
    \subsubsection{Discretisation}
        We discretise the position $\vec{x} \to \vec{x}_{i,j} = (x_i, y_j) = h(i,j)$, where $h$ is the spatial separation between two points on the grid. We consider $M$ points in each direction, giving $i, j \in [0,\, M)$. The lattice is then made up of $(M-2) \cross (M-2)$ internal points and boundary points. The time points become $t\to t_n=n\Delta t$, $n\in[0, N_t)$, where $\Delta t$ is the time step size and $N_t = T/\Delta t$ is the number of time points. Further, we find the wave function as
        \begin{equation}
            u(t, \vec{x}) \to u(t_n, \vec{x}_{i,j}) \equiv u_{i,j}^{(n)},
        \end{equation}
        and subsequently the probability density,
        \begin{equation}
            p(\vec{x};\, t) \to p(\vec{x}_{i,j};\, t_n) \equiv p_{i,j}^{(n)}.
        \end{equation}
        The potential is discretised as $v(\vec{x}) \to v(\vec{x}_{i,j})\equiv v_{i,j}$. 

    

\subsection{Periodic boundaries}\label{sec:method:periodic_boundaries}
    \Nanna{Dirichlet}

\subsection{Initial wave packet}\label{sec:method:initial_wave}


    \Nanna{choose one of these:}
    \begin{equation}\label{eq:p4_initial_condition}
    \begin{split}
        u(t\!=\!0, \vec{x}) = & \exp{-(\vec{x}-\vec{x}_\mathrm{c})^\mathrm{T}\Sigma^{-1}(\vec{x}-\vec{x}_\mathrm{c}) + \im \vec{p}^\mathrm{T}(\vec{x}-\vec{x}_\mathrm{c})} \,;  \\ 
        &\quad\Sigma = \mathrm{diag}(\svec{\sigma}^2) 
    \end{split}
    \end{equation}

    \begin{equation}\label{eq:p4_initial_condition_scalarform}
        u(t\!=\!0, \vec{x}) = \exp{-\frac{(x-x_\mathrm{c})^2}{2\sigma_x^2} -\frac{(y-y_\mathrm{c})^2}{2\sigma_y^2} + \im \vec{p}\cdot (\vec{x}-\vec{x}_\mathrm{c})}
    \end{equation}


\subsection{Slit configuration}\label{sec:method:slit_config}
    To create a single slit, we need \textit{two} objects with the property that the particle cannot propagate through it. Said objects are to have a separation between them, a \textit{slit}, perpendicular to the dominating motion of the particle \Nanna{Pls send help, cannot write proper sentences any more:(}. A natural choice for the shape is the rectangular with the longer side parallel to the slit, \Nanna{(Some theory about interference??)} hence the name "wall". Three such walls give rise to two slits, etcetera. Therefore, 
    \begin{equation}
        \#\mathrm{walls} = \begin{cases}
            \#\mathrm{slits} + 1 \quad &\mathrm{if}\,\, \#\mathrm{slits} \geq 1 \\
            0 \quad                     &\mathrm{if}\,\, \#\mathrm{slits} =0 
        \end{cases}.
    \end{equation}
    % OK, I admit, this was unnecessary...

    To give the Schrödinger equation information about these walls, we simply create a very large potential in the walls. These potential barriers will resemble concrete \Nanna{(mener egt ikke betong, men konkrete... hva er ordet??)} walls in the sense that the particle will have to navigate around it in order to get through to the other side.

    In particular, we will set up a number of such walls in the middle of our box, the separation between which is the aperture, aligned perpendicular to the initial momentum $\vec{p}$ in equation \eqref{eq:p4_initial_condition}. We will make sure that the slit setup is symmetric around the box centre in both directions.


\subsection{Artificial detector screen}\label{sec:method:screen}

    To reduce the dimensionality of the problem, we can assume to measure a particle with a detector screen at some horisontal position $x=x_\mathrm{sc}$ and time $t=t_\mathrm{sc}$, so that the screen spans of the vertical axis $y$. With a smart choice of pairing ($t_\mathrm{sc}$ \& $x_\mathrm{sc}$), we can safely assume that the particle is located somewhere along this line. Thus, it makes sense to consider a normalised one-dimensional probability function $p_{x=x_\mathrm{sc}}(y;\, t_\mathrm{sc})$ (read: probability of $y$ given $x=x_\mathrm{sc}$, at $t=t_\mathrm{sc}$) such that
    \begin{equation}\label{eq:tot_probability_along_y}
        p^\mathrm{tot}_{x=x_\mathrm{sc}}(t\!=\!t_\mathrm{sc}) =\sum_{j\in [0, M)} p_{x=x_\mathrm{sc}}(y_j;\, t_\mathrm{sc}) = 1.
    \end{equation}
    \Nanna{Do we have to explain how we find this?}



\subsection{Simulation parameters}\label{sec:method:simulation_parameters}

    \begin{table}[ht!]
        \centering
        \begin{tabular}{l|c}
            \hline
            \multicolumn{2}{l}{\textbf{Simulation parameters} \Nanna{??}} \\
            \hline
            Spatial step size ($h$)         & $5.0 \cdot 10^{-3}$   \\
            Temporal step size ($\Delta t$) & $2.5 \cdot 10^{-5}$   \\
            Total time ($T$)                & $*$                   \\
            \hline
            \multicolumn{2}{l}{\textbf{Slit configuration} (following section \ref{sec:method:slit_config})} \\
            \hline
            Number of slits ($\#\mathrm{slits}$)    & $*$                           \\
            Number of identical walls               & $\#\mathrm{slits} + 1 \geq 2$ \\
            Wall dimensions; width$\cross$height    & $0.02\cross 0.05$             \\
            Wall centre position                    & $(0.5,\, \sim\!0.5)$\footnote{Slit setup symmetric around $y=0.5$.}   \\
            Slit aperture (separation between walls)& $0.05$                        \\
            Potential inside barrier ($v_0$)        & $1.0\cdot 10^{10}$            \\
            \hline
            \multicolumn{2}{l}{\textbf{Initial wave packet} (the Gaussian in equation \eqref{eq:p4_initial_condition})} \\
            \hline
            Centre position ($\vec{x}_\mathrm{c}$)  & $(0.25,\,0.50)$           \\
            Spatial extent ($\svec{\sigma}$)        & $(0.05, \,*)$             \\
            Momentum ($\vec{p}$)                    & $(200,\, 0)$              \\
            \hline
            \multicolumn{2}{l}{\textbf{Vertical screen} (following section \ref{sec:method:screen})} \\
            \hline
            Horisontal position ($x_\mathrm{sc}$)   & $0.8$     \\
            Time point ($t_\mathrm{sc}$)            & $0.002$   \\
            \hline
        \end{tabular}
        \caption{Numerical values for static simulation parameters. Simulation-specific values are denoted "$*$".}
        \label{tab:simulation_parameters}
    \end{table}



    \begin{table}[ht!]
        \centering
        \begin{tabular}{clr|ccc}
            \hline
            & Simulation name & (label\footnote{Prefix to look for in \Nanna{link to animations etc.}}) & $\#\mathrm{slits}$ & $T$ & $\svec{\sigma}\cdot \hat{\vec{e}}_y$ \\
            \hline
            1. & No slits        & (\textit{NS})   & $0$   & $0.008$   & $0.05$ \\
            2. & Double-slit (1) & (\textit{DS1})  & $2$   & $0.008$   & $0.10$ \\
            3. & Double-slit (2) & (\textit{DS2})  & $2$   & $0.002$   & $0.20$ \\
            4. & \dots             & (\textit{SS})   & $1$   & $0.002$   & $0.20$ \\
            5. & \dots             & (\textit{TS})   & $3$   & $0.002$   & $0.20$ \\
            \hline
        \end{tabular}
        \caption{Supplement to table \ref{tab:simulation_parameters}. Additional information about the simulations considered in this paper.}
        \label{tab:simulations_info}
    \end{table}


% ===========================================
