
\section{Methods}\label{sec:methods}

\sout{We will use the Schrödinger equation to simulate a particle inside a two-dimensional quadratic box of size $L^2$. The use of  boundary conditions gives rise to the confinement of the particle \Nanna{or something like that}. We will place two or more walls inside the box to create one or more slits. The specifics of such configurations is elaborated in section \ref{sec:methods:simulation:slit_config}. A position inside if the box is called $\vec{x} =(x, y)$ and a time $t\in$, all unit less. Now $x, y \in [0,\,L]$ and $t\in[0,\,T]$ if we let $T$ be the time we simulate for. In this paper, we will use $L=1$.}

We will solve the dimensionless wave function from equation~\eqref{eq:2d_schrodinger_dimless} on a two-dimensional grid. Through the Born rule, equation~\eqref{eq:2d_bornrule_dimless}, we may interpret this solution to be proportional to the probability of finding a particle somewhere on this grid. We confine the particle by imposing Dirichlet boundary conditions. In order to investigate the behaviour of the wave function we place two or more walls inside the box to create one or more slits. The specifics of such configurations is elaborated in section \ref{sec:methods:simulation:slit_config}. A position \Vetle{(position perhaps not the best erm)}inside if the box is called $\vec{x} =(x, y)$ and a time $t\in$, all unit less. Now $x, y \in [0,\,L]$ and $t\in[0,\,T]$ if we let $T$ be the time we simulate for. In this paper, we will use $L=1$. \Vetle{(Last sentence confusing)}

\subsection{Numerical scheme}\label{sec:methods:numerical_scheme}

    \subsubsection{Discretisation}\label{sec:methods:numerical_scheme:discretisation}
        We discretise the position $\vec{x} \to \vec{x}_{i,j} = (x_i, y_j) = h(i,j)$, where $h$ is the spatial separation between two points on the grid and $i$ and $j$ are integers. We consider $M$ points in each direction, giving $i, j \in [0,\, M)$. The lattice is then made up of $(M-2) \cross (M-2)$ internal points and boundary points. The time points become $t\to t_n=n\Delta t$, where $\Delta t$ is the time step size and $N_t = T/\Delta t$ is the number of time points, so that the integer time steps are $n\in[0, N_t)$. In discretised form, we now write the wave function, probability density \Vetle{(density? Just double-checking)} and potential as 
        \begin{align}
            u(t, \vec{x}) \to u(t_n, \vec{x}_{i,j}) &\equiv u_{i,j}^{(n)}, \\ 
            p(\vec{x};\, t) \to p(\vec{x}_{i,j};\, t_n) &\equiv p_{i,j}^{(n)}, \\
            v(\vec{x}) \to v(\vec{x}_{i,j})&\equiv v_{i,j}, 
        \end{align}
        where the superscript $(n)$ refers to the time index. 

    \subsubsection{Implementation}\label{sec:methods:numerical_scheme:implementation}
        In order to solve equation \eqref{eq:2d_schrodinger_dimless} numerically on a discretised grid, as explained above, we use the Crank-Nicolson method. A detailed derivation of the method we implement is given in appendix \ref{app:discretisation}. It allows us to express the time evolution of the system as:
        \begin{align}\label{eq:equation_to_solve_non-matrix}
            u_{i,j}^{(n+1)} - \mathcal{F}_{i,j}^{(n+1)} = u_{i, j}^{(n)} + \mathcal{F}_{i,j}^{(n)},
        \end{align}
        where
        \begin{equation}
            \begin{split}
            \mathcal{F}_{i,j}^{(n)} = \quad&r\closed{ u_{i\!+\!1, j}-2u_{i,j} + u_{i\!-\!1, j}}^{(n)}  \\
            +\,&  r\closed{u_{i,j\!+\!1}-2u_{i,j} + u_{i,j\!-\!1}}^{(n)}\\ 
            -\,& \frac{\im \Delta t}{2} v_{i,j}\,u_{i,j}^{(n)},
            \end{split}
        \end{equation}
        and we have defined $r\equiv \frac{\im\Delta t}{2h^2}$. This time evolution is valid for any time step with the time range $n\in[0,N_t-2]$. Its spatial validity is restricted to the internal points of the grid; $i,j\in[1,M-2]$. For the boundary points we impose the Dirichlet boundary conditions (stated in the nomenclature). 

        If we define 
        \begin{equation}
            \begin{split}
                \vec{u}^{(n)} \equiv [&u_{1,1}^{(n)}, \dots, u_{M-2,1}^{(n)}, \dots, \\ 
                &u_{1,M-2}^{(n)}, \dots u_{M-2, M-2}^{(n)}],
            \end{split}
        \end{equation}
        to be the column vector with dimension $((M-2)^2 \cross 1)$ that contains all $u_{i,j}^{(n)}$ values for the internal points on the grid\footnote{$u_k^{(n)}$ is the $k$-th element of $\vec{u}^{(n)}$ where $k=(j-1)\cdot(M-2) + (i-1)$ is valid for the internal points.} \Vetle{(How is this footnote related to the remainder of the methods?)} This, combined with the boundary conditions allows us to rewrite equation \eqref{eq:equation_to_solve_non-matrix} into a matrix equation of the form:
        \begin{equation}\label{eq:equation_to_solve_matrix_form}
            A\vec{u}^{(n+1)} = B\vec{u}^{(n)}.
        \end{equation}
        $A$ and $B$ are matrices, whose specific forms are governed by the particular problem we consider in this report. We refer to appendix \ref{app:A_and_B} for a detailed description of these two matrices.
        We are now able to evolve the system in time by solving equation \eqref{eq:equation_to_solve_matrix_form} in a time loop, which we divide into two steps:
        \begin{equation}\label{eq:matrix_equation_to_solve}
            \begin{split}
                &\text{1. Perform matrix multiplication: } \vec{b} = B\vec{u}^{(n)}. \\
                &\text{2. Solve the matrix equation: } A\vec{u}^{(n+1)} = \vec{b},
            \end{split}
        \end{equation}
        where $u^{(n+1)}$ is the unknown we solve for. When solving equation \eqref{eq:matrix_equation_to_solve} we can use the structure of $A$ to our advantage. $A$ arise from a boundary value problem. We could use direct methods, such as Gaussian elimination or LU-decomposition. The latter may be used to compute the inverse, $A^{-1}$. This could be beneficial, since the constant $A^{-1}$ would apply at all time steps. We could also use iterative methods such as the Jacobi method or Gauss-Seidel. $A$ is diagonally dominant\footnote{See appendix \ref{app:A_and_B}}, and we are thus guaranteed convergence with the Jacobi method\citep{lecture_notes}. However, noticing that the vast majority of the entries in $A$ are zeros, we represent it numerically as a \textit{sparse matrix}, only saving the indices and values of non-zero entries. This allows us to use the ``spsolve()'' method of the Armadillo to solve equation \eqref{eq:matrix_equation_to_solve}. The method solves the system of sparse linear equations by a special algorithm for performing LU decomposition, called a Supernoidal LU decomposition (SuperLU) \cite{SuperLU}. We refer to their article for a more detailed description of the algorithm.  
        
\subsection{Initialisation and simulation}\label{sec:methods:simulation}
    When simulating a system that evolves in time, we have to know how the system is before the simulation, at time $t=0$.
    
    \subsubsection{Initial wave packet}\label{sec:methods:simulation:initial_wave}
        The initialisation is done by imposing the following initial condition. 
        \begin{equation}\label{eq:p4_initial_condition}
        \begin{split}
            u(t\!=\!0, \vec{x}) = & \exp{-(\vec{x}-\vec{x}_\mathrm{c})^\mathrm{T}\Sigma^{-1}(\vec{x}-\vec{x}_\mathrm{c}) + \im \vec{p}^\mathrm{T}(\vec{x}-\vec{x}_\mathrm{c})}  
        \end{split}
        \end{equation}
        where $\Sigma = \mathrm{diag}(\svec{\sigma}^2)$, $\vec{x}_c$ and $\vec{p}$ are the spatial variance, spatial centre location and momentum of the initial wave packet. This is a two dimensional complex Gaussian function. The parameters for different initialisations can be found in tables \ref{tab:simulation_parameters} and \ref{tab:simulations_info}. We normalise this initial state so that:
        \begin{equation}
            \sum_{i,j}u_{i,j}^{(0)*}u_{i,j}^{(0)} = 1, 
        \end{equation}
        which enables us to interpret $p_{i,j}^{n} = u_{i,j}^{(n)*}u_{i,j}^{(n)}$ as the probability that the particle is located in a small grid cell of size $h^2$, centred at $(x_i, y_i)$, which is in accordance with the Born rule from equation \eqref{eq:2d_bornrule_dimless}.
    
        % \begin{equation}\label{eq:p4_initial_condition_scalarform}
        %     u(t\!=\!0, \vec{x}) = \exp{-\frac{(x-x_\mathrm{c})^2}{2\sigma_x^2} -\frac{(y-y_\mathrm{c})^2}{2\sigma_y^2} + \im \vec{p}\cdot (\vec{x}-\vec{x}_\mathrm{c})}
        % \end{equation}

    \subsubsection{Slit configuration}\label{sec:methods:simulation:slit_config}

        To create a \textit{single} slit, we need \textit{two} objects with the property that the particle cannot propagate through it. Said objects are to have a separation between them, a \textit{slit}, perpendicular to the dominating motion of the wave packet. A natural choice for the shape of these objects are rectangles. The separation of these rectangles, or ``walls'', then give rise to the slits. Three such walls give rise to two slits, and so on. However, to achieve proper $n$-slit experiments, we should make sure that the two outermost walls of this setup reach the edges of the grid, so that the only way of bypassing the wall is through the slits.

        A significantly large potential will cause problems for the propagating wave. Hence, these walls may be simulated easily by creating large potential at the wall positions. This effectively set up a potential barrier where the walls are supposed to be, and the slits are represented as zero-potential gateways through the otherwise large potential.

        We will make sure that the slit setup is symmetric around the box centre in both directions. Thus, say we want $n$ slits, we would need $n-1$ identical wall pieces and two identical walls that closes the gap between the outer slits and the box edges.
    
    \subsubsection{Artificial detector screen}\label{sec:methods:simulation:screen}

        To reduce the dimensionality of the problem, we can assume to measure a particle with a detector screen at some horisontal position $x=x_\mathrm{sc}$ and time $t=t_\mathrm{sc}$, so that the screen spans of the vertical axis $y$. With a smart choice of pairing ($t_\mathrm{sc}$ \& $x_\mathrm{sc}$), we can safely assume that the particle is located somewhere along this line. Thus, it makes sense to consider a normalised one-dimensional probability function $p_{x=x_\mathrm{sc}}(y;\, t_\mathrm{sc})$ (read: probability of $y$ given $x=x_\mathrm{sc}$, at $t=t_\mathrm{sc}$) such that
        \begin{equation}\label{eq:tot_probability_along_y}
            p^\mathrm{tot}_{x=x_\mathrm{sc}}(t\!=\!t_\mathrm{sc}) =\sum_{j\in [0, M)} p_{x=x_\mathrm{sc}}(y_j;\, t_\mathrm{sc}) = 1.
        \end{equation}
        To study the expected diffraction pattern, it will be appropriate to choose a time and place right after the particle has passed through the slit(s). In the case with two slits, the setup imitates the famous double-slit experiment.

    \subsubsection{Simulations}\label{sec:methods:simulation:simulations}

        We aim to run a few experiments and their common setup is presented in table~\ref{tab:simulation_parameters}. The parameters specific to the simulations are presented in table~\ref{tab:simulations_info}. Our initial experiment box will have no slits or wall.

        \begin{table}[ht!]
            \centering
            \begin{tabular}{l|c}
                \hline
                \multicolumn{2}{l}{\textbf{Simulation parameters}} \\
                \hline
                Spatial step size ($h$)         & $5.0 \cdot 10^{-3}$   \\
                Temporal step size ($\Delta t$) & $2.5 \cdot 10^{-5}$   \\
                Total time ($T$)                & $*$                   \\
                \hline
                \multicolumn{2}{l}{\textbf{Slit configuration} (following section~\ref{sec:methods:simulation:slit_config})} \\
                \hline
                Number of slits ($N_\mathrm{slits}$)    & $*$                           \\
                Number of walls; inner + outer          & $(N_\mathrm{slits}-1) \, + \, 2$\\
                Wall dimensions; width$\cross$height    & $0.02\cross 0.05$\footnote{The height of the outer walls will be different.} \\
                Wall centre position                    & $(0.5,\, \sim\!0.5)$\footnote{Slit setup symmetric around $y=0.5$.}   \\
                Slit aperture; separation between walls & $0.05$                        \\
                Potential inside barrier ($v_0$)        & $1.0\cdot 10^{10}$            \\
                \hline
                \multicolumn{2}{l}{\textbf{Initial wave packet} (the Gaussian in equation~\eqref{eq:p4_initial_condition})} \\
                \hline
                Centre position ($\vec{x}_\mathrm{c}$)  & $(0.25,\,0.50)$           \\
                Spatial extent ($\svec{\sigma}$)        & $(0.05, \,*)$             \\
                Momentum ($\vec{p}$)                    & $(200,\, 0)$              \\
                \hline
                \multicolumn{2}{l}{\textbf{Vertical screen} (following section~\ref{sec:methods:simulation:screen})} \\
                \hline
                Horisontal position ($x_\mathrm{sc}$)   & $0.8$     \\
                Time point ($t_\mathrm{sc}$)            & $0.002$   \\
                \hline
            \end{tabular}
            \caption{Numerical values for static simulation parameters. Simulation-specific values are denoted ``$*$''.}
            \label{tab:simulation_parameters}
        \end{table}



        \begin{table}[ht!]
            \centering
            \begin{tabular}{clr|ccc}
                \hline
                & Simulation name & (label\footnote{Prefix to look for in the \href{\animlink}{animations}}) & $N_\mathrm{slits}$ & $T$ & $\svec{\sigma}\cdot \hat{\vec{e}}_y$ \\
                \hline
                1. & No slits        & (\textit{NS})   & $0$   & $0.008$   & $0.05$ \\
                2. & Double-slit (1) & (\textit{DS1})  & $2$   & $0.008$   & $0.10$ \\
                3. & Double-slit (2) & (\textit{DS2})  & $2$   & $0.002$   & $0.20$ \\
                4. & Single-slit     & (\textit{SS})   & $1$   & $0.004$   & $0.20$ \\
                5. & Triple-slit     & (\textit{TS})   & $3$   & $0.004$   & $0.20$ \\
                \hline
            \end{tabular}
            \caption{Supplement to table \ref{tab:simulation_parameters}. Additional information about the simulations considered in this paper.}
            \label{tab:simulations_info}
        \end{table}


% ===========================================
