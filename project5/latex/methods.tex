
\section{Methods}\label{sec:methods}

We will use the Schrödinger equation to simulate a particle inside a two-dimensional quadratic box of size $L^2$. The use of periodic boundary conditions gives rise to the confinement of the particle \Nanna{or something like that}. We will place two or more walls inside the box to create one or more slits. The specifics of such configurations is elaborated in section \ref{sec:methods:slit_config}. A position inside if the box is called $\vec{x} =(x, y)$ and a time $t\in$, all unit less. Now $x, y \in [0,\,L]$ and $t\in[0,\,T]$ if we let $T$ be the time we simulate for. In this paper, we will use $L=1$.

\subsection{Numerical scheme}


\Nanna{How to solve etc.} 
    \subsubsection{Discretisation}
        We discretise the position $\vec{x} \to \vec{x}_{i,j} = (x_i, y_j) = h(i,j)$, where $h$ is the spatial separation between two points on the grid. We consider $M$ points in each direction, giving $i, j \in [0,\, M)$. The lattice is then made up of $(M-2) \cross (M-2)$ internal points and boundary points. The time points become $t\to t_n=n\Delta t$, $n\in[0, N_t)$, where $\Delta t$ is the time step size and $N_t = T/\Delta t$ is the number of time points. Further, we find the wave function as
        \begin{equation}
            u(t, \vec{x}) \to u(t_n, \vec{x}_{i,j}) \equiv u_{i,j}^{(n)},
        \end{equation}
        and subsequently the probability density,
        \begin{equation}
            p(\vec{x};\, t) \to p(\vec{x}_{i,j};\, t_n) \equiv p_{i,j}^{(n)}.
        \end{equation}
        The potential is discretised as $v(\vec{x}) \to v(\vec{x}_{i,j})\equiv v_{i,j}$. 

    \subsubsection{Implementation}
        In order to solve equation \eqref{eq:2d_schrodinger_dimless} numerically on a discretised grid, as explained above, we use the Crank-Nicolson approximation. A derivation of this is given in appendix \ref{app:discretisation}. It allows us to express the time evolution of the system as:
        \begin{align}\label{eq:equation_to_solve_non-matrix}
            u_{i,j}^{(n+1)} - \mathcal{F}_{i,j}^{(n+1)} = u_{i, j}^{(n)} + \mathcal{F}_{i,j}^{(n)},
        \end{align}
        where 
        \begin{equation}
            \begin{split}
            \mathcal{F}_{i,j}^{(n)} = \quad&r\closed{ u_{i\!+\!1, j}-2u_{i,j} + u_{i\!-\!1, j}}^{(n)}  \\
            +\,&  r\closed{u_{i,j\!+\!1}-2u_{i,j} + u_{i,j\!-\!1}}^{(n)}\\ 
            -\,& \frac{\im \Delta t}{2} v_{i,j}\,u_{i,j}^{(n)},
            \end{split}
        \end{equation}
        and we have defined $r\equiv \frac{\im\Delta t}{2h^2}$. This time evolution is valid for any time step with the time range $n\in[0,N_t-2]$. Its spatial validity is restricted to the internal points of the grid; $i,j\in[1,M-2]$. For the boundary points we impose the Dirichlet boundary conditions (stated in the nomenclature). 

        If we define 
        \begin{equation}
            \vec{u}^{(n)} \equiv [u_{1,1}^{(n)}, \dots, u_{M-2,1}^{(n)}, \dots u_{1,M-2}^{(n)}, \dots u_{M-2, M-2}^{(n)}],
        \end{equation}
        to be the column vector of shape $((M-2)^2 \cross 1)$ that contains all $u_{i,j}^{(n)}$ values for the internal points on the grid.
        
        \Johan{is this necessary?}
        $u_k^{(n)}$ is the $k$-th element of $\vec{u}^{(n)}$ where $k=(j-1)\cdot(M-2) + (i-1)$ is valid for the internal points.
        
        This, combined with the boundary conditions allows us to rewrite equation \eqref{eq:equation_to_solve_non-matrix} into a matrix equation of the form:
        \begin{equation}\label{eq:equation_to_solve_matrix_form}
            A\vec{u}^{(n+1)} = B\vec{u}^{(n)},
        \end{equation}
        where $A$ and $B$ are special matrices, explained in appendix \Johan{some appendix reference goes here}. 
        We are now able to evolve the system in time by solving equation \eqref{eq:equation_to_solve_matrix_form} in a time loop by dividing it into two steps:
        \begin{equation}\label{eq:matrix_equation_to_solve}
            \begin{split}
                \vec{b} &= B\vec{u}^{(n)} \\
                A\vec{u}^{(n+1)} &= \vec{b},
            \end{split}
        \end{equation}
        where we solve for $u^{(n+1)}$. When solving equation \eqref{eq:matrix_equation_to_solve} we can use the structure of $A$ to our advantage. $A$ arise from a boundary value problem. We could use direct methods, such as Gaussian elimination or LU-decomposition to find $A^{-1}$. It could be beneficial to find this inverse since we solve for many time instances without changing $A$. We could also use iterative methods such as the Jacobi method or Gauss-Seidel. $A$ is by construction diagonally dominant\footnote{See appendix \Johan{ref to appendix about A and B}}, we are thus guaranteed convergence with the Jacobi method \Johan{More here?}. However, noticing that the vast majority of the entries in $A$ are zeros, we represent it numerically as a \textit{sparse matrix}, only saving the indices and values of non-zero entries. We thus use a build in solver in Armadillo to solve equation \eqref{eq:matrix_equation_to_solve} \Johan{this citation should perhaps not be here}\citep{lecture_notes}.
        
\subsection{Simulation}
    When simulating a system, we have to evolve it in time, and thus initialise it with an initial state.
    \subsubsection{Initial wave packet}\label{sec:methods:initial_wave}
    The initialisation is done by imposing the following initial condition. 
    \begin{equation}\label{eq:p4_initial_condition}
    \begin{split}
        u(t\!=\!0, \vec{x}) = & \exp{-(\vec{x}-\vec{x}_\mathrm{c})^\mathrm{T}\Sigma^{-1}(\vec{x}-\vec{x}_\mathrm{c}) + \im \vec{p}^\mathrm{T}(\vec{x}-\vec{x}_\mathrm{c})} \,;  \\ 
        &\quad\Sigma = \mathrm{diag}(\svec{\sigma}^2)  
    \end{split}
    \end{equation}
    \Johan{check how thourough you need to be when defining the variables in the above equation}
    This initial state is normalised such that $\langle \vec{u}^{(0)} | \vec{u}^{(0)}\rangle = 1$, so that the initial probability starts out normalised to 1. \Johan{rewrite thsi, make it clearer}.
 
    % \begin{equation}\label{eq:p4_initial_condition_scalarform}
    %     u(t\!=\!0, \vec{x}) = \exp{-\frac{(x-x_\mathrm{c})^2}{2\sigma_x^2} -\frac{(y-y_\mathrm{c})^2}{2\sigma_y^2} + \im \vec{p}\cdot (\vec{x}-\vec{x}_\mathrm{c})}
    % \end{equation}


    \subsubsection{Slit configuration}\label{sec:methods:slit_config}

        To create a single slit, we need \textit{two} objects with the property that the particle cannot propagate through it. Said objects are to have a separation between them, a \textit{slit}, perpendicular to the dominating motion of the particle \Nanna{Pls send help, cannot write proper sentences any more:}. A natural choice for the shape is the rectangular with the longer side parallel to the slit, \Nanna{(Some theory about interference??)} hence the name ``wall''. Three such walls give rise to two slits, etcetera. However, to achieve proper $n$-slit experiments, we should make sure that the two outermost walls of this setup are as long as they need to be to reach the edges of the box.

        To give the Schrödinger equation information about these walls, we simply create a very large potential in the walls. These potential barriers will resemble concrete \Nanna{(mener egt ikke betong, men konkrete... hva er ordet??)} walls in the sense that the particle will have to navigate between them in order to get through to the other side.
    
        In particular, we will set up a number of such walls in the middle of our box, the separation between which is the aperture, aligned perpendicular to the initial momentum $\vec{p}$ in equation~\eqref{eq:p4_initial_condition}. We will make sure that the slit setup is symmetric around the box centre in both directions. Thus, say we want $n$ slits, we would need $n-1$ identical wall pieces and two identical walls that closes the gap between the outer slits and the box edges. \Nanna{Does this make sense?}
    


    \subsubsection{Artificial detector screen}\label{sec:methods:screen}

        To reduce the dimensionality of the problem, we can assume to measure a particle with a detector screen at some horisontal position $x=x_\mathrm{sc}$ and time $t=t_\mathrm{sc}$, so that the screen spans of the vertical axis $y$. With a smart choice of pairing ($t_\mathrm{sc}$ \& $x_\mathrm{sc}$), we can safely assume that the particle is located somewhere along this line. Thus, it makes sense to consider a normalised one-dimensional probability function $p_{x=x_\mathrm{sc}}(y;\, t_\mathrm{sc})$ (read: probability of $y$ given $x=x_\mathrm{sc}$, at $t=t_\mathrm{sc}$) such that
        \begin{equation}\label{eq:tot_probability_along_y}
            p^\mathrm{tot}_{x=x_\mathrm{sc}}(t\!=\!t_\mathrm{sc}) =\sum_{j\in [0, M)} p_{x=x_\mathrm{sc}}(y_j;\, t_\mathrm{sc}) = 1.
        \end{equation}
        To study the infamous diffraction pattern, it will be \Nanna{appropriate} to choose a time and place right after the particle has passed through the slit(s). In the case with two slits, this \Nanna{analysis} imitates the double-slit experiment.



    \subsubsection{Simulations}\label{sec:methods:simulations}

        We aim to run a few experiments and their common setup is presented in table~\ref{tab:simulation_parameters}. The parameters specific to the simulations are presented in table~\ref{tab:simulations_info}. Our initial experiment box will have no slits or wall.

        \begin{table}[ht!]
            \centering
            \begin{tabular}{l|c}
                \hline
                \multicolumn{2}{l}{\textbf{Simulation parameters}} \\
                \hline
                Spatial step size ($h$)         & $5.0 \cdot 10^{-3}$   \\
                Temporal step size ($\Delta t$) & $2.5 \cdot 10^{-5}$   \\
                Total time ($T$)                & $*$                   \\
                \hline
                \multicolumn{2}{l}{\textbf{Slit configuration} (following section~\ref{sec:methods:slit_config})} \\
                \hline
                Number of slits ($N_\mathrm{slits}$)    & $*$                           \\
                Number of walls; inner + outer          & $(N_\mathrm{slits}-1) \, + \, 2$\\
                Wall dimensions; width$\cross$height    & $0.02\cross 0.05$\footnote{The height of the outer walls will be different.} \\
                Wall centre position                    & $(0.5,\, \sim\!0.5)$\footnote{Slit setup symmetric around $y=0.5$.}   \\
                Slit aperture; separation between walls & $0.05$                        \\
                Potential inside barrier ($v_0$)        & $1.0\cdot 10^{10}$            \\
                \hline
                \multicolumn{2}{l}{\textbf{Initial wave packet} (the Gaussian in equation~\eqref{eq:p4_initial_condition})} \\
                \hline
                Centre position ($\vec{x}_\mathrm{c}$)  & $(0.25,\,0.50)$           \\
                Spatial extent ($\svec{\sigma}$)        & $(0.05, \,*)$             \\
                Momentum ($\vec{p}$)                    & $(200,\, 0)$              \\
                \hline
                \multicolumn{2}{l}{\textbf{Vertical screen} (following section~\ref{sec:methods:screen})} \\
                \hline
                Horisontal position ($x_\mathrm{sc}$)   & $0.8$     \\
                Time point ($t_\mathrm{sc}$)            & $0.002$   \\
                \hline
            \end{tabular}
            \caption{Numerical values for static simulation parameters. Simulation-specific values are denoted ``$*$''.}
            \label{tab:simulation_parameters}
        \end{table}



        \begin{table}[ht!]
            \centering
            \begin{tabular}{clr|ccc}
                \hline
                & Simulation name & (label\footnote{Prefix to look for in \Nanna{link to animations etc.}}) & $N_\mathrm{slits}$ & $T$ & $\svec{\sigma}\cdot \hat{\vec{e}}_y$ \\
                \hline
                1. & No slits        & (\textit{NS})   & $0$   & $0.008$   & $0.05$ \\
                2. & Double-slit (1) & (\textit{DS1})  & $2$   & $0.008$   & $0.10$ \\
                3. & Double-slit (2) & (\textit{DS2})  & $2$   & $0.002$   & $0.20$ \\
                4. & Single-slit     & (\textit{SS})   & $1$   & $0.004$   & $0.20$ \\
                5. & Triple-slit     & (\textit{TS})   & $3$   & $0.004$   & $0.20$ \\
                \hline
            \end{tabular}
            \caption{Supplement to table \ref{tab:simulation_parameters}. Additional information about the simulations considered in this paper.}
            \label{tab:simulations_info}
        \end{table}


% ===========================================
