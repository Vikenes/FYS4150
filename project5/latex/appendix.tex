
\newpage
\appendix

\section{Discretisation}\label{app:discretisation}

Suppose you have the (1+1)-dimensional PDE $\pdv*{u}{t}=F$ where $u=u(t,x)$ and $F = F(t, x, u, \pdv*{u}{x}, \pdv*[2]{u}{x})$. 
The Crank-Nicolson scheme reads \citep{Crank-Nicolson}: 
\begin{equation}\label{eq:general_CN}
    \frac{u_i^{(n+1)}- u_{i}^{(n)} }{\Delta t} = \frac{1}{2} \closed{F_i^{(n+1)}+F_i^{(n)}},
\end{equation}
where $u_i^{(n)} = u(n\Delta t, i\Delta x)$ and $F_i^{(n)}$ is $F$ evaluated for $i, n$ and $u_i^{(n)}$. In our (2+1)-dimensional case where $u= u(t, \vec{x})$ we have
\begin{align}\label{eq:2d_schrodinger_CN}
    \pdv{u}{t} = F(t, \vec{x}, u, \nabla^{\! 2}u)= \im \closed{\nabla^{\! 2} u - v(\vec{x}) u},
\end{align}
and this approach translates to 
\begin{equation}\label{eq:2d_CN}
    \frac{u_{i,j}^{(n+1)}-u_{i,j}^{(n)}}{\Delta t} = \frac{1}{2} \closed{F_{i,j}^{(n+1)} + F_{i,j}^{(n)}} 
\end{equation}
where $u_{i,j}^{(n)} = u(n\Delta t, \vec{x}_{i,j})$, $\vec{x}_{i,j}=h(i,j)$, and $F_{i,j}^{(n)}$ is the right-hand side of equation \eqref{eq:2d_schrodinger_CN}, explicitly:
\begin{equation}
    F_{i, j}^{(n)} = \im \closed{  \bracket{\pdv[2]{u}{x}}_{i,j}^{(n)} + \bracket{\pdv[2]{u}{y}}_{i,j}^{(n)} - v_{i,j}u_{i,j}^{(n)}} \, ;
\end{equation} 
We can approximate the two spatial double derivatives \Nanna{(correct way to say?)} according to \Nanna{(Don't know what this approximation is called)}:
\begin{subequations}
    \begin{align}
        \bracket{\pdv[2]{u}{x}}_{i,j}^{(n)} \approx \frac{1}{h^2} \closed{u_{i\!+\!1, j}-2u_{i,j} + u_{i\!-\!1, j}}^{(n)}\, ;\\
        \bracket{\pdv[2]{u}{y}}_{i,j}^{(n)} \approx \frac{1}{h^2} \closed{u_{i, j\!+\!1}-2u_{i,j} + u_{i, j\!-\!1}}^{(n)}\, ;
    \end{align}
\end{subequations}
Define $r\equiv \frac{\im \Delta t}{2h^2}$. Further, let
\begin{equation}
    \begin{split}
    \mathcal{F}_{i,j}^{(n)} = \quad&r\closed{ u_{i\!+\!1, j}-2u_{i,j} + u_{i\!-\!1, j}}^{(n)}  \\
    +\,&  r\closed{u_{i,j\!+\!1}-2u_{i,j} + u_{i,j\!-\!1}}^{(n)}\\ 
    -\,& \frac{\im \Delta t}{2} v_{i,j}\,u_{i,j}^{(n)}.
    \end{split}
\end{equation}
Equation \eqref{eq:2d_CN} becomes:
\begin{equation}\label{eq:final_CN_scheme}
    u_{i,j}^{(n+1)} - \mathcal{F}_{i,j}^{(n+1)} = u_{i, j}^{(n)} + \mathcal{F}_{i,j}^{(n)} \, ;
\end{equation}
The final discretisation \eqref{eq:final_CN_scheme} is valid for any step in time within the time range ($n\in[0, N_t-2]$) and all internal points on the grid ($i,j \in [1, M-2]$).


\newpage

\section{$A$ and $B$ matrices}\label{app:A_and_B}
    In order to generalise the Crank-Nicolson approximation to the Schrödinger equation as a matrix equation $A\vec{u}^{(n+1)} = B\vec{u}^{(n)}$, that satisfy the Dirichlet boundary conditions on a discretised grid, we require $A$ and $B$ to take specific forms. We need both of them to be square matrices with dimensions $((M-2)^2\cross(M-2)^2)$, with vectors $\vec{a}$ and $\vec{b}$ as diagonal. These are given by:
    \begin{equation}
        \begin{split}
            a_k &= 1+4r +\frac{\im\Delta t}{2}v_{i,j} \\
            b_k &= 1-4r -\frac{\im\Delta t}{2}v_{i,j},
        \end{split}
    \end{equation}
    where $r$ is still $r\equiv \frac{\im\Delta t}{2h^2}$. We fill the matrices $A$ and $B$ with $\pm r$-values along the first and third super- and subdiagonal; $A$ with $-r$ and $B$ with $+r$. The $(M-2)$-th element along the first super- and subdiagonal for each matrix must be zero. For demonstrative purposes, if $(M-2)=3$, $A$ and $B$ would look like the following:

    \begin{equation}
        A = 
        \begin{bmatrix} 
            a_0 & -r  & 0 & -r & 0 & 0 & 0 & 0 & 0 \\
            -r  & a_1 & -r & 0 & -r & 0 & 0 & 0 & 0 \\
             0  & -r  & a_2 & 0 & 0 & -r & 0 & 0 & 0 \\
            -r  & 0   & 0 & a_3 & -r & 0 & -r & 0 & 0 \\
             0  & -r  & 0 & -r & a_4 & -r & 0 & -r & 0 \\ 
             0  & 0   & -r & 0 & -r & a_5 & 0 & 0 & -r \\
             0  & 0   & 0 & -r & 0 & 0 & a_6 & -r & 0 \\
             0  & 0   & 0 & 0 & -r & 0 & -r & a_7 & -r \\
             0  & 0   & 0 & 0 & 0 & -r & 0 & -r & a_8  
        \end{bmatrix}
    \end{equation}

    \begin{equation}
        B = 
        \begin{bmatrix} 
            b_0 & r & 0 & r & 0 & 0 & 0 & 0 & 0 \\
            r & b_1 & r & 0 & r & 0 & 0 & 0 & 0 \\
            0 & r & b_2 & 0 & 0 & r & 0 & 0 & 0 \\
            r & 0 & 0 & b_3 & r & 0 & r & 0 & 0 \\
            0 & r & 0 & r & b_4 & r & 0 & r & 0 \\
            0 & 0 & r & 0 & r & b_5 & 0 & 0 & r \\
            0 & 0 & 0 & r & 0 & 0 & b_6 & r & 0 \\
            0 & 0 & 0 & 0 & r & 0 & r & b_7 & r \\
            0 & 0 & 0 & 0 & 0 & r & 0 & r & b_8
        \end{bmatrix}
    \end{equation}
By construction (and inspection), we notice that $B$ is the complex conjugate of $A$, and that $A$ is a diagonally dominant matrix: $\abs{a_{i,i}} > \sum_{i\neq j}\abs{a_{i,j}}$ for each row in $A$.