\section{Discussion}\label{sec:discussion}


% The first two simulations:

The sanity check represented by the graphs figure~\ref{fig:ptot_deviation_NS_DS1} indicates that our solution is stable for $t\leq 0.008$, which is satisfying. 

\Nanna{FILL ME}

The animations (found \href{\animlink}{here}) show very different behaviours for \textit{NS} and \textit{DS1}. From the former to the ladder, we can clearly see the effect of introducing slits and how the probability distribution navigates around the barriers. 

The first experiment, for which the initial wave packet is circular, shows the expected behaviour as this is analytically solvable \Nanna{(Should check this solution)}. As the particle propagates in the $x$-direction, we can see the dispersive behaviour of the probability density in the $y$-direction. In other words, the initially narrow spatial probability distribution gets broader over time.
% Something about the particle?

%We see that the spatial distribution evolves from being circular and small to being elliptic with increasing eccentricity, having its semi-major axis in the $y$-direction. Even though the distribution is broader after a time, there are still large areas where one does not expect to find the particle at all.

The second experiment starts out with a somewhat larger wave packet, being more spread out in the $y$-direction. It propagates to the double-slit where it is split in several ``pieces''. After hitting the walls and being bounced back, the slits make sure to divide the blobs into even more blobs. \Nanna{More here}

\Nanna{Comment on boundary conditions}

\Nanna{FILL ME}


% The third sim.:


The probability density 


% The screen-thingy:







% ===========================================