\section{Discussion}\label{sec:discussion}

The sanity check represented by the graphs in figure~\ref{fig:ptot_deviation_NS_DS1} indicates that our solution is stable for $t\leq 0.008$. However, this result was only obtained from two particular simulation configurations. One of the main features in all our simulations were that we considered vertically symmetric wave packets propagating towards higher values of $x$. A more proper verification of our implementations would require tests with different initializations. Some natural choices for further investigation would be to consider wave packets initialized with a non-zero momentum in the vertical direction, as well as different choices for their vertical centre. Nonetheless, the initial testing provided desirable results within the scope of our analysis. 

The animations show very different behaviours for \textit{NS} and \textit{DS1}. From the former to the latter, we can clearly see the effect of introducing slits and how the probability distribution navigates around the barriers. With no slits, we can clearly see the dispersive behaviour of the probability density in the $y$-direction. This effect is less noticeable when slits are present, as it's difficult to distinguish this feature from effects caused by interactions with the slits. 

When it comes to our choice of using the SuperLU method to solve the linear equations, this was an efficient way to deal with a complex aspect of the numerical implementations. One drawback with this choice is that we are left with little insight to the workings of our algorithm as a whole. For instance, figure~\ref{fig:ptot_deviation_NS_DS1} shows that the total probability deviates less when we had double slits, compared to the case when we had no slits. This seems to be a bit counterintuitive, and there may be several reasonable explanations for why we get this result. For one thing, our measurements of the deviations is not very informative, in the sense that we only measure the sum of absolute deviations. However, even with further improvements of the analysis, the SuperLU method prevents us from drawing assertive conclusions. This method utilizes concepts like partial pivoting, where the accuracy may depend on specifics of the configurations. Concluding about varying performance for different simulation configurations would thus require more extensive studies, where other numerical methods are considered.        






% ===========================================