\section{Results}\label{sec:results}




\subsection{\Nanna{dunno what to call this section}}

    The first simulation\footnote{Animation available here \Nanna{link to anim}.}, where there are no slits, is run with parameters according to tables \ref{tab:simulation_parameters} and \ref{tab:simulations_info} (NS). By rights, at any time $t$, the probabilities $p(\vec{x}_{i,j}; \, t)$ should sum up to one, as the particle is present inside the box. As a sanity check, we plot 
    \begin{equation}\label{eq:total_probability_deviation}
        \abs{1- p^\mathrm{tot}(t)}\,; \quad p^\mathrm{tot}(t) = \sum_{i,j\in [0, M)} p(\vec{x}_{i,j};\, t)
    \end{equation}
    as a function of time in figure \ref{fig:ptot_deviation_NS_DS1}, where we see that the deviation ($\sim 10^{-14}$) is close to machine precision.

    The next simulation\footnote{Animation available here \Nanna{link to anim}.} we consider has a double-slit barrier and a somewhat broader initial wave packet in the vertical direction (DS1 in \ref{tab:simulations_info}). We perform the same sanity check as above and once again get deviations of order $ 10^{-14}$, also shown in figure \ref{fig:ptot_deviation_NS_DS1}.

    \begin{figure}[ht!]
        \centering
        \includegraphics[width=\linewidth]{temp/NS_DS1_Ptot_deviation.png}
        \caption{Some stupid caption. Fix ticks on this plot}
        \label{fig:ptot_deviation_NS_DS1}
    \end{figure}

    % should this be in appendix?
    
    We change the settings of our simulations so that they are only run for a fourth of the duration and with an even larger initial wave packet. Still two slits considered, our third simulation\footnote{Animation available here \Nanna{link to anim}.} is run with parameters in accordance with \ref{tab:simulation_parameters} and \ref{tab:simulations_info} (DS2). Snapshots of the probability distribution $p(\vec{x};\, t)$ at $t=0$, $T/2$ and $T$, $T=0.002$, are presented in figure \ref{fig:snapshots_p_DS2}. \Nanna{COMMENT ON NORMALISATION OF COLOURMAP!} In addition, the real and imaginary part of $u(t, \vec{x})$ at the same time points are presented in figures \ref{fig:snapshots_ReU_DS2} and \ref{fig:snapshots_ImU_DS2}, respectively.


    \begin{figure*}
        \centering
        \begin{subfigure}{0.92\textwidth}
            \includegraphics[width=\linewidth]{temp/DS2_snapshots_P.png}
            \caption{probability}
            \label{fig:snapshots_p_DS2}
        \end{subfigure}
        \hfill
        \begin{subfigure}{0.92\textwidth}
            \includegraphics[width=\linewidth]{temp/DS2_snapshots_ReU.png}
            \caption{real part}
            \label{fig:snapshots_ReU_DS2}
        \end{subfigure}
        \hfill
        \begin{subfigure}{0.92\textwidth}
            \includegraphics[width=\linewidth]{temp/DS2_snapshots_ImU.png}
            \caption{imaginary part}
            \label{fig:snapshots_ImU_DS2}
        \end{subfigure}
        \caption{Colour maps showing \Nanna{blah blah}}
        \label{fig:snapshots_DS2}
    \end{figure*}




\subsection{\Nanna{Screen? idk}}

    Take note of the dotted vertical line in the right panel of figure \ref{fig:snapshots_p_DS2}. Following the method described in section \ref{sec:method:screen}, we find the probability density along this screen for detecting the particle. \Nanna{Is this clear?} The distribution is plotted as a function of $y$ in figure \ref{fig:p_along_y_DS2}.



    \begin{figure}[ht!]
        \centering
        \includegraphics[width=\linewidth]{temp/DS2_P_along_screen.png}
        \caption{Some stupid caption}
        \label{fig:p_along_y_DS2}
    \end{figure}

    \begin{figure}[ht!]
        \centering
        \includegraphics[width=\linewidth]{temp/SS_P_along_screen.png}
        \caption{Some stupid caption}
        \label{fig:p_along_y_SS}
    \end{figure}

    \begin{figure}[ht!]
        \centering
        \includegraphics[width=\linewidth]{temp/TS_P_along_screen.png}
        \caption{Some stupid caption}
        \label{fig:p_along_y_TS}
    \end{figure}



% ===========================================