\section{Results}\label{sec:results}

For the interested reader with a keen interest in visualisation, we present animations of the probability density resulting from each simulation listed in table~\ref{tab:simulations_info} in our \href{\animlink}{GitHub}\footnote{\url{https://github.com/Vikenes/FYS4150/tree/main/project5/output/videos}} repository. 

The different results presented in the following subsections are related to parameters that are given in table \ref{tab:simulation_parameters} and \ref{tab:simulations_info}.   

\subsection{Probability conservation}\label{sec:results:sanity}

    The first simulation, where there are no slits, gives deviations in $p^\mathrm{tot}(t)$ with and order of $\sim 10^{-14}$, as seen in figure~\ref{fig:ptot_deviation_NS_DS1}. In the case of the double slit ((\textit{DS1})), we see from the figure that we get deviations that are somewhat smaller in magnitude. Both simulations yield deviations that are close to machine precision. These results imply that our numerical implementations are reasonable in a physical sense, and that the strength of the potential barriers is appropriate for the total probability to be conserved.   

    \begin{figure}[ht!]
        \centering
        \includegraphics[width=\linewidth]{temp/NS_DS1_Ptot_deviation.png}
        \caption{\textit{NS} \& \textit{DS1}: The total probability's deviation from the theoretical value, following equation~\eqref{eq:total_probability_deviation}, as function of time, for the two initial experiments. The vertical axis is logarithmic.}
        \label{fig:ptot_deviation_NS_DS1}
    \end{figure}


\subsection{Two-dimensional problem}
    
    We now simulate the box with a double slit (\textit{DS2}), and present snapshots of the probability distribution $p(\vec{x};\, t)$ at $t\in\{0,T/2,T\}$, shown in figure~\ref{fig:snapshots_p_DS2}, with $T=0.002$. In addition, the real and imaginary part of $u(t, \vec{x})$ at the same times are presented in figures~\ref{fig:snapshots_ReU_DS2} and~\ref{fig:snapshots_ImU_DS2}, respectively.

    \begin{figure*}
        \centering
        \begin{subfigure}{0.92\textwidth}
            \includegraphics[width=\linewidth]{temp/DS2_snapshots_P.png}
            \caption{\textit{DS2}: Snapshots representing the evolution of the probability density. The dotted vertical line represents the screen we consider in figure~\ref{fig:p_along_y}.}
            \label{fig:snapshots_p_DS2}
        \end{subfigure}
        \hfill
        \begin{subfigure}{0.92\textwidth}
            \includegraphics[width=\linewidth]{temp/DS2_snapshots_ReU.png}
            \caption{\textit{DS2}: Snapshots representing the evolution of the real part of the wave function in position space.}
            \label{fig:snapshots_ReU_DS2}
        \end{subfigure}
        \hfill
        \begin{subfigure}{0.92\textwidth}
            \includegraphics[width=\linewidth]{temp/DS2_snapshots_ImU.png}
            \caption{\textit{DS2}: Snapshots representing the evolution of the imaginary part of the wave function in position space.}
            \label{fig:snapshots_ImU_DS2}
        \end{subfigure}
        \caption{Colour maps showing the solution of the Schrödinger equation for the \textit{DS2} setup at times $t=0.000,\, 0.001, \,0.002$. Pay attention to the varying colour bar scale from frame to frame.}
        \label{fig:snapshots_DS2}
    \end{figure*}

    We stress that the scales on the colour bars in figure~\ref{fig:snapshots_DS2} change between simulations. In figure~\ref{fig:snapshots_p_DS2}, the dark areas correspond to a low likelihood of detecting the particle, whereas the areas coloured yellow represent the places where one is most likely to find the particle. In the wave function components in figures~\ref{fig:snapshots_ReU_DS2} and~\ref{fig:snapshots_ImU_DS2}, negative values are coloured green, values close to zero are coloured navy and positive values have a light blue shade, roughly speaking. Note also that for all three quantities, the snapshot at $t=0.001$ have a different scaling on the colour bar compared to the snapshots at the other two instances in time. The variations of the scales of the colour bars are chosen to better resolve the spatial distribution in the individual snapshots. 

\subsection{One-dimensional problem}

    Using the same parameters, we run two additional simulations (\textit{SS} and \textit{TS}), except that we vary the number of slits (see tables~\ref{tab:simulation_parameters} and~\ref{tab:simulations_info}). In the following, we study the system further by removing one dimension.

    Take note of the dotted vertical line in the right panel of figure~\ref{fig:snapshots_p_DS2}. Following the method described in section~\ref{sec:methods:simulation:screen}, we find the probability density along this screen for detecting the particle. \Nanna{Is this clear?} The distribution for this particular experiment is plotted as a function of $y$ in figure~\ref{fig:p_along_y_DS2}. The same screen is used for the single- and triple-slit experiments, resulting in the graphs in figures~\ref{fig:p_along_y_SS} and~\ref{fig:p_along_y_TS}, respectively. %The resulting one-dimensional probability distributions are presented in figure~\ref{fig:p_along_y}.

    \begin{figure}
        \centering
        \begin{subfigure}{0.46\textwidth}
            \includegraphics[width=\linewidth]{temp/SS_P_along_screen.png}
            \caption{\textit{SS}: The probability along the $y$-axis for the single-slit experiment.}
            \label{fig:p_along_y_SS}
        \end{subfigure}
        \hfill
        \begin{subfigure}{0.46\textwidth}
            \includegraphics[width=\linewidth]{temp/DS2_P_along_screen.png}
            \caption{\textit{DS2}: The probability along the $y$-axis for the double-slit (2) experiment.}
            \label{fig:p_along_y_DS2}
        \end{subfigure}
        \hfill
        \begin{subfigure}{0.46\textwidth}
            \includegraphics[width=\linewidth]{temp/TS_P_along_screen.png}
            \caption{\textit{TS}: The probability along the $y$-axis for the triple-slit experiment.}
            \label{fig:p_along_y_TS}
        \end{subfigure}
        \caption{The amplitude for detecting the particle at vertical positions $y$ at time $t=0.002$ given that it is located at horisontal position $x=0.8$, that is the quantity $p_{x=0.8}(y;\, 0.002)$ obeying equation~\eqref{eq:tot_probability_along_y}, for simulations with one, two and three slits, but otherwise equal parameters.}
        \label{fig:p_along_y}
    \end{figure}






% ===========================================