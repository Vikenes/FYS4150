\section{Conclusion}\label{sec:conclusion}

Through the Crank-Nicolson scheme, we have succeeded in building an algorithm that solves the time evolution of the quantum state initialised by a Gaussian wave packet existing in a two-dimensional box with a zero- to three-slitted (potentially more) barrier.

The deviation in the total probability is insignificant for the scope of durations we consider, and does not seemingly tend to increase. This is an indication of a robust algorithm and a stable solution.

The scattering pattern on the detector screen is consistent with results from well-known excrements.

Future work might explore other sorts of wave packets. Our approach should be straight-forwardly generalised to work in three dimensions, with the caveat of high computational expenses. Experimenting with both the temporal and spatial solution could also be useful.



% ===========================================