\section{Introduction}\label{sec:introduction}


Blah blah 





For a single, non-relativistic particle with mass $m_\mathrm{P}$ in a two-dimensional potential $\mathcal{V}(t, \vec{x})$, the Schrödinger equation reads\footnote{In position space, that is. \Nanna{Should we comment on this?}}
\begin{equation}\label{eq:2d_schrodinger}
    \im \hbar \pdv{}{t}\Psi(t, \vec{x}) = -\frac{\hbar^2}{2m_\mathrm{P}}\nabla^{\!2} \Psi(t, \vec{x})  + \mathcal{V}(t, \vec{x}) \Psi(t, \vec{x}).
\end{equation}
For a set of initial and boundary conditions, the partial differential equation (PDE) describes the temporal and spatial evolution of the complex-valued function $\Psi(t, \vec{x})$ related to the quantum state of the aforementioned particle. In such a case, at a time $t$, the probability density for an experimentalist to locate the particle at $\vec{x}$ \Nanna{("for detecting ..." is better, but I don't want to copy Anders)} is
\begin{equation}\label{eq:2d_bornrule}
    P(\vec{x}; \, t) = \abs{\Psi(t, \vec{x})}^2 = \Psi^{\!*}(t, \vec{x}) \Psi(t, \vec{x}),
\end{equation} 
originating from the Born rule; \Nanna{fill me}

In this paper we will consider a dimensionless time-independent potential, i.e. we let $\mathcal{V}(t, \vec{x}) \to v(\vec{x})$. The specifics of the scaling do not concern us in this paper, and we simply rewrite equation \eqref{eq:2d_schrodinger} to the dimensionless equation
\begin{equation}\label{eq:2d_schrodinger_dimless}
    \im \pdv{}{t}u(t, \vec{x}) = - \nabla^{\!2}u(t, \vec{x})+ v(\vec{x})u(t, \vec{x}),
\end{equation}
where we substituted $\Psi(t, \vec{x}) \to u(t, \vec{x})$. In equation \eqref{eq:2d_schrodinger_dimless} all variables are dimensionless. When demanding the proper normalisation on $u(t,\vec{x})$, it follows that the Born rule now takes the form of
\begin{equation}\label{eq:2d_bornrule_dimless}
    p(\vec{x}; \, t) = \abs{u(t, \vec{x})}^2 = u^{\!*} (t, \vec{x})u(t, \vec{x}).
\end{equation}
\Nanna{Should maybe rephrase this paragraph.}


The outline of this report will be as follows: Section~\ref{sec:methods} will be all about the methods we use, and how we discretise and solve equation~\eqref{eq:2d_schrodinger_dimless} on a two-dimensional grid. It is divided into two subsections, section~\ref{sec:methods:numerical_scheme} which is about discretisation (section~\ref{sec:methods:numerical_scheme:discretisation}) and implementation (section~\ref{sec:methods:numerical_scheme:implementation}) of the numerical scheme. Section~\ref{sec:methods:simulation} is about the initialisation of the system (section~\ref{sec:methods:simulation:initial_wave}), the slit configuration (section~\ref{sec:methods:simulation:slit_config}), the detector screen (section~\ref{sec:methods:simulation:screen}) and the simulation itself (section~\ref{sec:methods:simulation:simulations}). We present the results and sanity checks in section~\ref{sec:results}. In section~\ref{sec:discussion} we provide a brief discussion, and a conclusion in section~\ref{sec:conclusion}. There are also two appendices explaining the Crank-Nicolson discretisation (appendix~\ref{app:discretisation}), and the explanation of the matrices $A$ and $B$ (appendix~\ref{app:A_and_B}) used to solve our system. 



\textbf{HUSK}:
\begin{enumerate}
    \item[*] Noe om diffraction
    \item[*] Noe om hva vi forventer å se med og uten slits
    \item[*] Disclaimer: Forsvare bruket av begrepet wave packet / wave function
\end{enumerate}







% ===========================================


