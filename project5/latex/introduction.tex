\section{Introduction}\label{sec:introduction}

As with all good stories, this too starts in Ancient Greece, where Democritus laughingly\footnote{This is hypothetical, but plausible as he was known as the ``laughing philosopher''.} postulated the world to be made up of tiny, indestructible elements; The atom. \citep{berryman_2016}. Along with this followed a description of the universe where everything was made up of these atoms. With the years passing, the elementary particles and eventually quarks was discovered. With the discovery of smaller and smaller building blocks came the desire to truly understand the workings of nature on a microlevel. This was formalised with the theory of quantum mechanics in the first half of the 20th century \citep{QMH}, challenging established concept of physics like causality and deterministic behavioural descriptions of particles. 

This was manifested through the otherwise well known double slit experiment, where you propagate particles through two thin slits, and record whatever pattern they make onto a screen at the other side of the slits. For particles of substantial size, each slit gave rise to a band on the screen, as we would intuitively expect. However, for smaller particles (first discovered with electrons), the pattern recorded turned out to be a diffraction pattern, similar to the patterns created by liquids (and light), indicating wave-like behaviour of the electrons. 

This wave-like behaviour is encompassed into the so-called \textit{wave function}, whose spatial and temporal dynamics are described with the Schrödinger equation. The aim of this report is to solve this equation in a confined spatial environment, effectively evolving the wave function in time. We will propagate a Gaussian distributed wave packet through different slit configuration in order to replicate the interference pattern caused by the dispersive and diffractive behaviour of the particles, similar to those exhibited by liquids and light.

The outline of this report will be as follows: Section~\ref{sec:methods} will be all about the methods we use, and how we discretise and solve the Schrödinger equation on a two-dimensional grid. It is divided into two subsections, section~\ref{sec:methods:numerical_scheme} which is about discretisation (section~\ref{sec:methods:numerical_scheme:discretisation}) and implementation (section~\ref{sec:methods:numerical_scheme:implementation}) of the numerical scheme. Section~\ref{sec:methods:simulation} is about the initialisation of the system (section~\ref{sec:methods:simulation:initial_wave}), the slit configuration (section~\ref{sec:methods:simulation:slit_config}), the detector screen (section~\ref{sec:methods:simulation:screen}) and the simulation itself (section~\ref{sec:methods:simulation:simulations}). We present the results and sanity checks in section~\ref{sec:results}. In section~\ref{sec:discussion} we provide a brief discussion, and a conclusion in section~\ref{sec:conclusion}. There are also two appendices explaining the Crank-Nicolson discretisation (appendix~\ref{app:discretisation}), and the explanation of the matrices $A$ and $B$ (appendix~\ref{app:A_and_B}) used to solve our system. 

% ===========================================


