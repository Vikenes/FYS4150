\section{Introduction}\label{sec:introduction}


Blah blah 



For a single, non-relativistic particle with mass $m_\mathrm{P}$ in a two-dimensional potential $V(t, \vec{x})$, the Schrödinger equation reads\footnote{In position space, that is. \Nanna{Should we comment on this?}}
\begin{equation}\label{eq:2d_schrodinger}
    \im \hbar \pdv{}{t}\Psi(t, \vec{x}) = -\frac{\hbar^2}{2m_\mathrm{P}}\nabla^{\!2} \Psi(t, \vec{x})  + V(t, \vec{x}) \Psi(t, \vec{x}).
\end{equation}
For a set of initial and boundary conditions, the partial differential equation (PDE) describes the temporal and spatial evolution of the complex-valued function $\Psi(t, \vec{x})$ related to the quantum state of the aforementioned particle. In such a case, at a time $t$, the probability density for an experimentalist to locate the particle at $\vec{x}$ \Nanna{("for detecting ..." is better, but I don't want to copy Anders)} is \Nanna{large P or small p???}
\begin{equation}\label{2d_bornrule}
    P(\vec{x}; \, t) = \abs{\Psi(t, \vec{x})}^2 = \Psi^{\!*}(t, \vec{x}) \Psi(t, \vec{x}),
\end{equation} 
originating from the Born rule; \Nanna{fill me}

In this paper we will consider a dimensionless time-independent potential, i.e. we let $V(t, \vec{x}) \to v(\vec{x})$. The specifics of the scaling do not concern us in this paper, and we simply rewrite equation \eqref{eq:2d_schrodinger} to the dimensionless equation
\begin{equation}\label{eq:2d_schrodinger_dimless}
    \im \pdv{}{t}u(t, \vec{x}) = - \nabla^{\!2}u(t, \vec{x})+ v(\vec{x})u(t, \vec{x}),
\end{equation}
where we substituted $\Psi(t, \vec{x}) \to u(t, \vec{x})$. In equation \eqref{eq:2d_schrodinger_dimless} all variables are dimensionless. When demanding the proper normalisation on $u(t,\vec{x})$, it follows that the Born rule now takes the form of
\begin{equation}\label{eq:2d_bornrule_dimless}
    p(\vec{x}; \, t) = \abs{u(t, \vec{x})}^2 = u^{\!*} (t, \vec{x})u(t, \vec{x}).
\end{equation}
\Nanna{Should maybe rephrase this paragraph.}







% ===========================================

% Let $x^\mu = (x^0,x^j)=(t, \vec{x})$ be the (2+1)-dimensional Lorentz vector describing a position $\vec{x}\in \mathbb{R}^2$ at a time $t$. That is, we let Greek indices run form 0 to 2 and Latin indices run from 1 to 2. \Nanna{Rephrase}


% \begin{equation}\label{eq:2d_schrodinger}
%     \im \hbar \partial^0 \Psi(x) = -\frac{\hbar^2}{2m} \partial^j\partial^j \Psi(x)  + V(x) \Psi(x)
% \end{equation}


% \begin{equation}\label{2d_bornrule}
%     p(x) = \abs{\Psi(x)}^2 = \Psi^{*}(x) \Psi(x) 
% \end{equation}

% \begin{equation}\label{eq:2d_schrodinger_dimless}
%     \im \partial^0 u(x) = - \partial^j \partial^j u(x)+ v(\vec{x})u(x)
% \end{equation}


% \begin{equation}\label{eq:2d_bornrule_dimless}
%     p(x) = \abs{u(x)}^2 = u^{*} (x)u(x)
% \end{equation}


