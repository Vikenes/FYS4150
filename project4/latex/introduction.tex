\section{Introduction}\label{sec:introduction}

In his 1944 paper \cite{Onsager_Ising2D}, Lars Onsager found analytical solutions of the 2D Ising model. The aim of this investigation is to reproduce, numerically, one of his solution; that of critical temperature for an infinite lattice. Very briefly, the Ising model is a mathematical model designed to reproduce statistical properties of nature by modelling a configuration of magnetic moments; in our case on a two-dimensional discrete lattice. Changes to this configuration cause a change of both energy and magnetization of the model. Measuring this exactly is hard due to the size and properties of statistical ensembles (especially for large grids). However, they may be estimated using stochastic sampling, for example by means of Markov chain Monte Carlo sampling. Using this we are able to numerically investigate and compare our model to the analytical result found by Onsager. 

The investigation will be structured as follows: Section \ref{sec:theory} gives a brief theoretical introduction into the workings of the Ising model, some 

% ===========================================
