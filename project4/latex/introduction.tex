\section{Introduction}\label{sec:introduction}
In his 1944 paper \cite{Onsager_Ising2D}, Lars Onsager found an analytical solution to the 2D Ising model. Very briefly, the Ising model is a mathematical model designed to model ferromagnetism in statistical mechanics on a microscopic level. When the 1D model (Ising chain) was introduced, it was quickly discovered that there was no evidence for any phase transition occurring. In two dimensions however, approximate solutions of the Ising model showed indication of phase transitions. The analytical solution of Onsager came around 20 years after the analytical solution of the 1D model. To this day, an analytical solution of the Ising model in three dimensions does not exist. The aim of this investigation is to reproduce, numerically, one of Onsager's results; that of the critical temperature for the 2D Ising model of infinite size. 

Measuring this exactly is hard due to the size and properties of statistical ensembles (especially for large grids). However, they may be estimated using stochastic sampling, for example by means of Markov chain Monte Carlo sampling. We will implement this through the Metropolis algorithm and use theoretical properties of phase transition phenomena to numerically investigate and compare our model to the analytical result found by Onsager. 

The investigation will be structured as follows: Section \ref{sec:theory} gives a brief introduction to the relevant theoretical background of our methods. Section \ref{sec:implementations} outlines how we implement and adapt various aspects of the theory to simulate the Ising model numerically. In section  \ref{sec:methods} we present the different methods we will consider for our analysis. The results are then presented in section \ref{sec:results}, discussed in section \ref{sec:discussion}, and we make a final conclusion in section \ref{sec:conclusion}. Supporting figures, equations and tables that are not directly contributing towards the main analysis are presented as supplementary material in Appendix.

% ===========================================
