\section{Results}\label{sec:results}


\begin{figure}[!ht]
    \includegraphics[width=0.9\linewidth]{temp/equilibriation_time_energy.png}
    \caption{Equilibration time for the mean energy $\langle \epsilon \rangle$ as function of the number of Monte Carlo cycles, $\nmc$, for $T=1$ and $T=2.4$. The red line correspond to an ordered initial state in which all the spins point in the same direction. The blue line correspond to an unordered initial state with random initialization of the spins.}

    \label{fig:equil_energy}
\end{figure} 



\begin{figure}[!ht]
    \includegraphics[width=0.9\linewidth]{temp/equilibriation_time_magnetization.png}
    \caption{Equilibration time for the mean magnetization $\langle \vert m \vert \rangle$ as function of the number of Monte Carlo cycles, $\nmc$, for $T=1$ and $T=2.4$. The red line correspond to an ordered initial state in which all the spins point in the same direction. The blue line correspond to an unordered initial state with random initialization of the spins.}
    \label{fig:equil_magn}
\end{figure}


\begin{figure}[!ht]
    \includegraphics[width=0.9\linewidth]{temp/histogram.png}
    \caption{Estimates of the probability distribution $p_e(\epsilon; T)$ for $T=1$ and $T=2.4$}
    \label{fig:histogram}
\end{figure}


\begin{figure}[!ht]
    \includegraphics[width=0.9\linewidth]{temp/phase_transitionNT50.png} 
    \caption{Hvem vinner VM?}
    \label{fig:phase_transition}
\end{figure} 

\begin{figure}[!ht]
    \includegraphics[width=0.9\linewidth]{temp/phase_transitionNT101.png} 
    \caption{Spania?}
    \label{fig:phase_transition_zoomed}
\end{figure} 

\begin{figure}[!ht]
    \includegraphics[width=0.9\linewidth]{temp/critical_temperatures.png} 
    \caption{Spania?}
    \label{fig:critical_temperatures}
\end{figure} 

\begin{figure}[!ht]
    \includegraphics[width=0.9\linewidth]{temp/eficiency.png} 
    \caption{Spania?}
    \label{fig:efficiency}
\end{figure} 




\begin{table}[!ht]
    \begin{tabular}{lrrrr}
\toprule
$N$ & $\langle \epsilon \rangle$ & $\langle \abs{m} \rangle$ & $\langle C_V \rangle$ & $\langle \chi \rangle$ \\
\midrule
$10^{2}$ & -1.940 & 0.980 & 0.466 & 0.058 \\
$10^{3}$ & -1.984 & 0.994 & 0.127 & 0.023 \\
$10^{4}$ & -1.996 & 0.999 & 0.029 & 0.004 \\
$10^{5}$ & -1.996 & 0.999 & 0.029 & 0.003 \\
Analytical & -1.996 & 0.999 & 0.032 & 0.004 \\
\bottomrule 
\end{tabular}

    \caption{Comparison of analytical results. }
    \label{tab:compare_analytical}
\end{table}

\begin{table}[!ht]
    \begin{tabular}{lrrr}
\toprule
$L$ & $\quad T_c^\mathrm{C_V}\,[\tunit]$ & $\quad T_c^\mathrm{linreg}$ [$\tunit$] & $\quad \mathrm{MSE}(C_V/J k_B)$ \\
\midrule
40 & 2.2875 & 2.2873 & 0.0004 \\
60 & 2.2793 & 2.2799 & 0.0008 \\
80 & 2.2764 & 2.2762 & 0.0011 \\
100 & 2.2741 & 2.2740 & 0.0031 \\
$\infty$ &   & 2.2651 &  \\
\bottomrule
\end{tabular}

    \caption{Critical temperatures.}
    \label{tab:critical_temperatures}
\end{table}

% =========================================== 