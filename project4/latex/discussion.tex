\section{Discussion}\label{sec:discussion}
\subsection{Limitations of the Metropolis algorithm}\label{subsec_discussion:limitations}
One important limitations of our methods in general is that the Metropolis algorithm is not very efficient close to the critical temperature \cite{lecture_notes}. Cluster algorithms such as Wolff and Swendsen-Wang are much more efficient in simulating properties near the critical temperature. With this in mind, we limit the following discussions to improvements that can be done on our own implementations, with less emphasis on how to accurately determine the critical temperatures. 

\subsection{Sample sizes and equilibration}\label{subsec_discussion:sample_size_equi}
When we estimated the number of equilibration cycles, we did not account for other lattice sizes than $L=20$. For the lattice sizes we used to determine critical temperatures, a higher number of cycles could be required to properly equilibrate the system. In Fig. \ref{fig:phase_transition_zoomed} we see that there are large variations in $C_V$ at different temperatures, and from the last column in table \ref{tab:critical_temperatures} we see that the variance increases for larger lattice sizes. One major drawback of our method is that we have only sampled averages at each temperature from a single MC simulation. To obtain results with more statistical significance an obvious choice is to perform repeated MC simulations, add these individual averages to obtain some final estimates of average quantities. An alternative approach which requires less additional computations would be to implement dynamical values of $\nmc$ and $\nequi$, where we increase the number of cycles for larger lattices. Despite yielding less rigorous results in a statistical sense, it would reduce the uncertainties to some extent.  


\subsection{Estimating critical temperatures from curve fitting}\label{subsec_discussion:critical_temps_curvefit}
When we estimate the critical temperature from the heat capacities, fitting the $C_V$ data with cubic splines is arguable an approach with room for improvement. For $L=100$ this is particularly evident, as there are some temperature regimes where the $C_V$ data is almost consistently lower or higher than the fit. In Fig. \ref{fig:phase_transition_zoomed} we see examples of this for $T\sim2.31\tunit$, where the data points lie below the spline, and for $T\sim2.33\tunit$ where the data points lie above it. A more careful consideration of how we perform the fit should therefore have been taken into account. If samples from multiple simulations had been used, as mentioned in the previous subsection, a natural way of approaching this would be to weight the fit at different temperatures according to the variance of the individual simulations.  

This leads us to the linear regression, which was also performed without taking any statistical weights into account. It's evident from Fig. \ref{fig:phase_transition_zoomed} that there is less variance in the measured $C_V$ for $L=40$ compared to $L=100$. For the latter lattice size, we also see a large spread in the measured heat capacities near the critical temperature. This raises some concerns, as it implies that we have significant uncertainties in our estimates of the critical temperatures, particularly for larger lattice sizes. The linear regression should therefore have been adjusted so that it emphasizes less the critical temperature estimates obtained from the larger lattice sizes.


% ==========================================