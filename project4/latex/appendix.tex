\appendix 

\section{Analytical results} \label{app:analytical_expressions}
To derive the analytical expression for the $(2\cross2)$ Ising model, we consider the number of spins with $+1$, and compute the energy, magnetization and degeneracy of the system for each of these. The resulting values are given in table \ref{tab:2d_configurations}. For all positive or all negative spins, the energy of the system is $E=-8J$. For $N_\uparrow=2$ there the only configuration yielding a non-negative energy is when the spins on the diagonal have the same orientation. In that case, we have $E=+8J$. Otherwise, the remaining configurations all have $E=0$.    

\Johan{Check this, two times where 2 spins point up, and is $s$ supposed to be $\nu$?}
\begin{table}[h!]
    \begin{tabular}{crrc}
    $N_\uparrow$ & $E(s)$ & $M(s)$ & Degeneracy \\ \hline
    $4$ & $-8\,\mathrm{J}$ & $4$ & $1$ \\
    $3$ & $0$ & $2$ & $4$ \\
    $2$ & $8\,\mathrm{J}$ & $0$ & $2$ \\
    $2$ & $0$ & $0$ & $4$ \\
    $1$ & $0$ & $-2$ & $4$ \\
    $0$ & $-8\,\mathrm{J}$ & $-4$ & $1$
    \end{tabular}
    \caption{Analytical solutions to the $(2\cross2)$ Ising model, where features of the system is tabulated as a function of the number of spins pointing up (being $+1$): $N_\uparrow$.}
    \label{tab:2d_configurations}
\end{table}
The partition function can now be calculated with Eq. \eqref{eq:partition_function}, using the degeneracy associated with the three possible energy values from table \ref{tab:2d_configurations},
\begin{equation}\label{eq:analytical_partition_func}
    Z = 12 + 2 e^{-8\bj} + 2 e^{8\bj} = 12 + 4 \cosh(8\bj).
\end{equation}
Using Eq. \eqref{eq:observable_exp_val}, we get the following analytical expectation values for the lattice \Johan{$s$ or $\nu$?} 
\begin{align}
    \expval{\eps} &= \sum_s \eps(s) \frac{1}{Z} e^{-\beta E(s)} = -\frac{8J}{Z}\sinh(8\bj), \label{eq:analytical_avg_energy} \\ 
    \expval{\eps^2} &= \sum_s \eps(s)^2 \frac{1}{Z} e^{-\beta E(s)} = \frac{16 J^2}{Z}\cosh(8\bj),\label{eq:analytical_avg_energy_squared} \\ 
    \expval{\abs{m}} &= \sum_s \abs{m(s)} \frac{1}{Z}e^{-\beta E(s)} = \frac{2}{Z}\closed{2 + e^{8\bj}},\label{eq:analytical_avg_absolute_mag} \\ 
    \expval{m^2} &= \sum_s m(s)^2 \frac{1}{Z} e^{-\beta E(s)} = \frac{2}{Z}\closed{1 + e^{8\bj}}. \label{eq:analytical_avg_mag_squared}
\end{align}
The resulting expressions for $C_V$ and $\chi$ are thus 
\begin{align}
    C_V &= \frac{\beta}{N T}\closed{\expval{E^2} - \expval{E}^2} \nonumber \\ 
    &= \frac{64\beta J^2}{Z T} \bracket{\cosh(8\bj) - \frac{4}{Z}\sinh^2(8\bj)}, \label{eq:analytical_CV} \\ 
    \chi &= \frac{\beta}{N}\closed{\expval{M^2} - \expval{M}^2} \nonumber \\
    &= \frac{8\beta}{Z}\bracket{\closed{1+e^{8\bj}} - \frac{2}{Z}\closed{2+e^{8\bj}}^2}. \label{eq:analytical_magnetic_susceptibility}
\end{align}


\section{Parallel}

\begin{figure}[!ht]
    \includegraphics[width=\linewidth]{temp/eficiency.png} 
    \caption{Duration as function of lattice size for running $\nmc=10^6$ Monte Carlo cycles with $N_\mathrm{eq}=1.5\cdot 10^4$ equilibration cycles for 20 temperatures on the interval $T\in[2.0,2.5]$. The parallelization is done using 5 threads and is shown in blue in the plot, against the serial implementation in red.} 
    \label{fig:efficiency}
\end{figure} 