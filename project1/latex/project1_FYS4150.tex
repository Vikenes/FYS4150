\documentclass[english,notitlepage]{revtex4-1}  % defines the basic parameters of the document
%For preview: skriv i terminal: latexmk -pdf -pvc filnavn



% if you want a single-column, remove reprint

% allows special characters (including æøå)
\usepackage[utf8]{inputenc}
%\usepackage[english]{babel}

%% note that you may need to download some of these packages manually, it depends on your setup.
%% I recommend downloading TeXMaker, because it includes a large library of the most common packages.

\usepackage{physics,amssymb}  % mathematical symbols (physics imports amsmath)
\include{amsmath}
\usepackage{graphicx}         % include graphics such as plots
\usepackage{xcolor}           % set colors
\usepackage{hyperref}         % automagic cross-referencing (this is GODLIKE)
\usepackage{listings}         % display code
\usepackage{subfigure}        % imports a lot of cool and useful figure commands
\usepackage{float}
%\usepackage[section]{placeins}
\usepackage{algorithm}
\usepackage[noend]{algpseudocode}
\usepackage{subfigure}
\usepackage{tikz}
\usetikzlibrary{quantikz}
% defines the color of hyperref objects
% Blending two colors:  blue!80!black  =  80% blue and 20% black
\hypersetup{ % this is just my personal choice, feel free to change things
    colorlinks,
    linkcolor={red!50!black},
    citecolor={blue!50!black},
    urlcolor={blue!80!black}}

\begin{document}

\title{Project 1 FYS4150}      % self-explanatory
\author{Vetle Vikenes, Johan Mylius Kroken and Nanna Bryne}          % self-explanatory
\date{\today}                             % self-explanatory
\noaffiliation                            % ignore this, but keep it.


\maketitle 
    
\textit{List a link to your github repository here!}
    
\section*{Problem 1}

We have the equation 

\begin{align}\label{eq:poisson_eq}
    -\dv[2]{u}{x} =  100e^{-10x}, \quad \text{where} \quad x\in [0,1], \quad u(0)=u(1)=0.
\end{align}


We want to check that 

\begin{align}\label{eq:p1_analytical_sol}
    u(x) = 1 - (1-e^{-10})x - e^{-10x}
\end{align}

is the solution of eq. \ref{eq:poisson_eq}. We first control that eq. \ref{eq:p1_analytical_sol} satisfies the boundary conditions. 

\begin{align*}
    u(0) &= 1 - 0 - e^{0} = 0 \\
    u(1) &= 1 - (1-e^{-10}) - e^{-10} = 0
\end{align*}
 
We find the double derivative of $u(x)$,

\begin{align*}
    \dv{u}{x} &= -(1-e^{-10}) - (-10)e^{-10x} \quad \Rightarrow \quad \dv[2]{u}{x} = -100e^{-10x},
\end{align*}

and see that this satisfies the differential eq. \ref{eq:poisson_eq}.


\section*{Problem 2}

\section*{Problem 3}
    We have the discretisation of the second order derivative as follows:
    \begin{align*}
        \dv[2]{u}{x}\bigg\rvert_{x_i} = u_i'' = \frac{u_{i+1}-2u_i+u_{i-1}}{h^2} + \mathcal{O}(h^2) \approx \frac{v_{i+1}-2v_i+v_{i-1}}{h^2} = v_i''
    \end{align*}
    where we have approximated $u_i$ with $v_i$ by  neglecting the error term, and $h$ is the step length. We further insert this  into the Poisson equation  and obtain the following:
    \begin{align*}
        -\dv[2]{u}{x}\approx - \dv[2]{v}{x}\bigg\rvert_{x_i} &= -v_i'' = f(x_i) = f_i \\
        \implies -v_i'' &= f_i \\
        \implies \frac{-v_{i+1}+2v_i-v_{i-1}}{h^2} &= f_i \\
        \implies -v_{i+1}+2v_i-v_{i-1} &= h^2f_i \equiv g_i
    \end{align*}





\section*{Problem 4}



\begin{algorithm}[H]
    \caption{Some algorithm}\label{algo:midpoint_rule}
    \begin{algorithmic}
        \State Some maths, e.g $f(x) = x^2$.  \Comment{Here's a comment}
        \For{$i = 0, 1, ..., n-1$}
        \State Do something here 
        \EndFor
        \While{Some condition}
        \State Do something more here 
        \EndWhile
        \State Maybe even some more math here, e.g $\int_0^1 f(x) \dd x$
    \end{algorithmic}
\end{algorithm}
   
\end{document}

