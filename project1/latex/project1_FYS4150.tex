\documentclass[english,notitlepage]{revtex4-1}  % defines the basic parameters of the document
%For preview: skriv i terminal: latexmk -pdf -pvc filnavn




\usepackage[utf8]{inputenc}
%\usepackage[english]{babel}


\usepackage{physics,amssymb} 
\include{amsmath}
\usepackage{graphicx}        
\usepackage{xcolor} 
\usepackage{hyperref} 
\usepackage{listings}     
\usepackage{subfigure}    
\usepackage{float}
\usepackage{algorithm}
\usepackage[noend]{algpseudocode}
\usepackage{subfigure}
\usepackage{tikz}
\usetikzlibrary{quantikz}
% defines the color of hyperref objects
% Blending two colors:  blue!80!black  =  80% blue and 20% black
\hypersetup{ % this is just my personal choice, feel free to change things
colorlinks,
linkcolor={red!50!black},
citecolor={blue!50!black},
urlcolor={blue!80!black}}

\graphicspath{{../output/plots/}} % path for figures

\begin{document}

\title{Project 1 FYS4150} 
\author{Vetle Vikenes, Johan Mylius Kroken and Nanna Bryne}      
\date{\today}                 
\noaffiliation       

\maketitle 
    
\textit{List a link to your github repository here!}
    
\section*{Problem 1}

We have the equation 

\begin{align}\label{eq:poisson_eq}
    -\dv[2]{u}{x} =  100e^{-10x}, \quad \text{where} \quad x\in [0,1], \quad u(0)=u(1)=0.
\end{align}


We want to check that 

\begin{align}\label{eq:p1_analytical_sol}
    u(x) = 1 - (1-e^{-10})x - e^{-10x}
\end{align}

is the solution of eq. \ref{eq:poisson_eq}. We first control that eq. \ref{eq:p1_analytical_sol} satisfies the boundary conditions. 

\begin{align*}
    u(0) &= 1 - 0 - e^{0} = 0 \\
    u(1) &= 1 - (1-e^{-10}) - e^{-10} = 0
\end{align*}
 
We find the double derivative of $u(x)$,

\begin{align*}
    \dv{u}{x} &= -(1-e^{-10}) - (-10)e^{-10x} \quad \Rightarrow \quad \dv[2]{u}{x} = -100e^{-10x},
\end{align*}

and see that this satisfies the differential eq. \ref{eq:poisson_eq}.


\section*{Problem 2}
\begin{figure}[h!]
    \centering
    \includegraphics[width=0.5\textwidth]{u_x.pdf}
    \caption{CAPTION}
\end{figure}

\section*{Problem 3}
    We have the discretisation of the second order derivative as follows:
    \begin{align*}
        \dv[2]{u}{x}\bigg\rvert_{x_i} = u_i'' = \frac{u_{i+1}-2u_i+u_{i-1}}{h^2} + \mathcal{O}(h^2) \approx \frac{v_{i+1}-2v_i+v_{i-1}}{h^2} = v_i''
    \end{align*}
    where we have approximated $u_i$ with $v_i$ by  neglecting the error term, and $h$ is the step length. Inserting $v_i$ into the Poisson equation yields  
    \begin{align}
        -v_i'' = \frac{-v_{i+1}+2v_i-v_{i-1}}{h^2} &= f_i \nonumber \\
        -v_{i+1}+2v_i-v_{i-1} &= h^2f_i \equiv g_i, \label{eq:p3_disc_poisson_eq}
    \end{align}
    which is a discretized version of the Poisson equation. 



\section*{Problem 4}
The computations performed in equation \eqref{eq:p3_disc_poisson_eq} to obtain a value for $g_i$ can be expressed in terms of vectors. If we have two column vectors, $\mathbf{a}$ and $\mathbf{\tilde{v}}$, defined as $a=(-1,\,2,\,-1)$ and $\mathbf{\tilde{v}}=(v_{i-1},\,v_i,\,v_{i+1})$, respectively, equation \eqref{eq:p3_disc_poisson_eq} can be written as  
\begin{align*}
    \mathbf{a}^T \, \mathbf{\tilde{v}} = \begin{pmatrix}
        -1 & 2 & -1 
    \end{pmatrix}
    \begin{pmatrix}
        v_{i-1} \\ 
        v_i \\ 
        v_{i+1}
    \end{pmatrix}
    = -v_{i+1}+2v_i-v_{i-1} = g_i.
\end{align*}
We can extend this to to compute all elements of $\mathbf{g}=(g_0,\,\dots,\,g_n)$ by multiplying $\mathbf{v}=(v_0,\,\dots,\,v_n)$ with a tridiagonal matrix $A$ with $2$ on its main diagonal and $-1$ on its subdiagonal and superdiagonal. 

As a matrix equation, equation \eqref{eq:p3_disc_poisson_eq} can be written as  
\begin{align*}
    A\mathbf{v} = 
    \begin{pmatrix}
        2 & -1 & & \\
        -1 & \ddots & \ddots & \\ 
         & \ddots & & -1 \\
          & & -1 & 2
    \end{pmatrix}
    \begin{pmatrix}
        v_0 \\
        v_1 \\
        \vdots \\ 
        v_n
    \end{pmatrix} = 
    \begin{pmatrix}
        g_0 \\
        g_1 \\ 
        \vdots \\
        g_n
    \end{pmatrix}
    = \mathbf{g},
\end{align*}
where the value of the original differential equation is obtained by $\mathbf{f}=\mathbf{g}/h^2$


\begin{algorithm}[H]
    \caption{Some algorithm}\label{algo:midpoint_rule}
    \begin{algorithmic}
        \State Some maths, e.g $f(x) = x^2$.  \Comment{Here's a comment}
        \For{$i = 0, 1, ..., n-1$}
        \State Do something here 
        \EndFor
        \While{Some condition}
        \State Do something more here 
        \EndWhile
        \State Maybe even some more math here, e.g $\int_0^1 f(x) \dd x$
    \end{algorithmic}
\end{algorithm}
   
\end{document}

