\section{Results}\label{sec:results}

\subsection{Single particle}\label{sec:one_particle}
We start by simulating a single particle, Particle 1, with initial conditions given in \ref{sec:initial_conditions}. The duration of the simulation is set to $T_1$ and we use $n_1$ time steps. In Figure \ref{fig:compare_z_analytical} we plot the $z$-component of the position as function of time. The solutions are found both using the FE and RK4 schemes, shown in red and blue respectively. \Johan{FE and RK4 defined shortening earlier?} \Nanna{- Yes, in introduction:)}. The analytical solution, found from equation \eqref{eq:p1_diffeq_solution_z} where the initial conditions yield $c_1 = z_0 = 20\,\mathrm{\text{\textmu}m}$ and $c_2 = v_{z,0}/\omega_z= 0$ is shown with green in the figure.

Since both FE and RK4 are approximations of the analytical solution we expect them to have a relative error compared with it. We repeat the simulations using $n_2, n_3$ and $n_4$ time steps and find the relative error at each time step according to \Johan{(reference to relative error formula)}. The results are shown in Figures \ref{fig:error_FE} and \ref{fig:error_RK}. We note that for an increasing time step, the relative time step decrease, and the error for the RK4 scheme is significantly less than that of FE. \Johan{(ref some previous discussion with hs?)}

%------------------------------------------------------------------------------
%   displacement in z-direction as function of time. analyitcal, FE, and RK4
%------------------------------------------------------------------------------
\begin{figure}
    \includegraphics[width=\linewidth]{/png/compare_z_analytical.png}
    \caption{Movement in the $z$-plane for a single particle, simulation for $T_1=50$ \textmu s with $n_1=4000$ time steps. The analytical solution from equation \eqref{eq:p1_diffeq_solution_z} is plotted in green, and the integrated solutions using the FE and RK4 scheme is shown in red and blue respectively. Initial conditions are those of Particle 1, given in \ref{sec:initial_conditions}.}
    \label{fig:compare_z_analytical}
\end{figure}


%------------------------------------------------------------------------------
%   Relative error FE methods
%------------------------------------------------------------------------------
\begin{figure}
    \includegraphics[width=\linewidth]{png/rel_error_FE.png}
    \caption{Relative error of the FE scheme, simulated for $T_1=50$ \textmu s, for a different amount of steps $n_k, k\in[1,4]$ as indicated in the figure legend. Pay attention to the logarithmic $y$-axis.}
    \label{fig:error_FE}
\end{figure}


%------------------------------------------------------------------------------
%   Relative error RK method
%------------------------------------------------------------------------------
\begin{figure}
    \includegraphics[width=\linewidth]{png/rel_error_RK.png}
    \caption{Relative error of the RK4 scheme, simulated for $T_1=50$ \textmu s, for a different amount of steps $n_k, k\in[1,4]$ as indicated in the figure legend. Pay attention to the logarithmic $y$-axis.}
    \label{fig:error_RK}
\end{figure}



% r_\mathrm{err}^\mathrm{RK4} = 3.87392
% r_\mathrm{err}^\mathrm{FE} = 1.44620

We compute the convergence rate at according to equation \eqref{eq:error_convergence_rate} and find that $r_\mathrm{err}\simeq 3.9 $ for RK4 and that $r_\mathrm{err}\simeq 1.4 $ for FE. \Nanna{What is dis}


\subsection{Two particles}\label{sec:two_particles}

We simulate Particle 1 and Particle 2 for duration $T_1$,  using $n_2$ time steps. This is a trade off between numerical accuracy and computation time, since we now use the RK4 method \Johan{(Or do we use both?)}, both with and and without Coulomb interactions. For the three figures soon to be mentioned, Particle 1 is shown in blue, Particle 2 in red, initial conditions are as specified in section \ref{sec:initial_conditions} and the start and end point of their trajectories are indicated with a cross and a star respectively. Figure \ref{fig:xy_two_particles} show the movement of the two particles in the $xy$-plane. The left panel when we have no Coulomb interaction, and the right panel when they interact. 

Figure \ref{fig:x_phase_two_particles} show the phase plots of the two particles along the $x$-axis, and Figure \ref{fig:z_phase_two_particles} along the $z$-axis.
% Figures spanning the whole width. 
\newpage
\onecolumngrid


%------------------------------------------------------------------------------
%   movement of two particles in xy-plane with and without interaction
%------------------------------------------------------------------------------
\begin{figure}
    \includegraphics[width=0.8\linewidth]{png/xy_two_particles.png}
    \caption{Movement of two particles in the $xy$-plane simulated for $T_1=50$ $\mu$s. The left panel show their behaviour without interactions and the right panel with. We have used the RK4 scheme to obtain these trajectories. Particle 1 is shown in blue, and Particle 2 in red. Both panels use the same initial conditions as specified in section \ref{sec:initial_conditions}. The starting points of their trajectories are indicated with a cross, their ending points with a star.}
    \label{fig:xy_two_particles}
\end{figure}

%------------------------------------------------------------------------------
%   Phase plot (x, vx)
%------------------------------------------------------------------------------
\begin{figure}
    \includegraphics[width=0.8\linewidth]{png/x_phase_plot.png}
    \caption{Phase plots along the $x$-axis for two different particles simulated for $T_1=50$ $\mu$s. The left panel show their behaviour without interactions and the right panel with. Particle 1 is shown in blue, and Particle 2 in red. Both panels use the same initial conditions as specified in section \ref{sec:initial_conditions}. The starting points of their trajectories are indicated with a cross, their ending points with a star.}
    \label{fig:x_phase_two_particles}
\end{figure} 

%------------------------------------------------------------------------------
%   Phase plot (z, vz)
%------------------------------------------------------------------------------
\begin{figure}
    \includegraphics[width=0.8\linewidth]{png/z_phase_plot.png}
    \caption{Phase plots along the $z$-axis for two different particles simulated for $T_1=50$ $\mu$s. The left panel show their behaviour without interactions and the right panel with. Particle 1 is shown in blue, and Particle 2 in red. Both panels use the same initial conditions as specified in section \ref{sec:initial_conditions}. The starting points of their trajectories are indicated with a cross, their ending points with a star.}
    \label{fig:z_phase_two_particles}
\end{figure}

%------------------------------------------------------------------------------
%   3d plot (x, y z)
%------------------------------------------------------------------------------
\begin{figure}
    \includegraphics[width=0.8\linewidth]{png/trajectory_3d.png}
    \caption{The trajectory of Particle 1 (blue) and Particle 2 (red) in the Penning trap for $T_1=50$ $\mu$s. The left panel show their behaviour without interactions and the right panel with. The starting points of their trajectories are indicated with a cross, their ending points with a star.}
    \label{fig:3d_trajectory}
\end{figure}
\twocolumngrid





\subsection{Many particles}\label{sec:many_particles}

Three broad-band scans are performed, one for each of the amplitudes $f=0.1, 0.4,$ and $0.7$. In particular, we run 300 simulations where we use the same amplitude $f$ and for each run, apply a different frequency $\omega_V \in[0.2,2.5] \,\mathrm{MHz}$. We reused the time step size from section \ref{sec:two_particles}. The fraction of particles remaining is presented in Figure \ref{fig:trapped_without} as a function of applied frequency. 

For the next step, we choose the smallest amplitude and the middlemost response in Figure \ref{fig:trapped_without}, i.e. $f=0.1$ and $\omega_V \in[1.35,1.45] \,\mathrm{MHz}$. We run 50 simulations, each with slightly different applied frequency, for which interactions forces are included. The number of steps is reduced to 20000 in this case, corresponding to a step size $2 T_1/n_1$ (twice that of the first simulation in section \ref{sec:one_particle}). We repeat for negligible interactions. Figure \ref{fig:trapped_with} shows $N_\mathrm{trapped}$ as a function of $\omega_V$ in both cases. 


%------------------------------------------------------------------------------
%   Trapped particles without interaction
%------------------------------------------------------------------------------
\begin{figure}
    \includegraphics[width=\linewidth]{png/trapped_particles_without_interaction.png}
    \caption{Fraction of particles that are still trapped by the Penning trap, $N_\mathrm{trapped}$, after $T_2=500$ $\mu$s, as function of the applied frequency $\omega_V\in[0.2,2.5]$ MHz, for different amplitudes $f\in\{0.1, 0.4, 0.7\}$, without considering particle interactions. All graphs are resulting from simulations using $n=80000$ time steps.}
    \label{fig:trapped_without}
\end{figure}

%------------------------------------------------------------------------------
%   Trapped particles fine grid. 
%------------------------------------------------------------------------------
\begin{figure}
    \includegraphics[width=\linewidth]{png/trapped_particles_fine.png}
    \caption{Fraction of particles that are still trapped by the Penning trap, $N_\mathrm{trapped}$, after $T_2=500$ $\mu$s, as function of selected applied resonance frequencies $\omega_V\in[1.35,1.45]$ MHz, with amplitude $f=0.1$, both with and without particle interactions. Both graphs are resulting from simulations using $n=20000$ time steps.}
    \label{fig:trapped_with}
\end{figure}



%------------------------------------------------------------------------------
%   Nanna, Johan and Vetle working on the project, respectively (from left to right), 2022 (colorized).
%------------------------------------------------------------------------------
\begin{figure}
    \includegraphics[width=\linewidth]{dummy.jpg}
    \caption{Nanna, Johan and Vetle working on the project, respectively (from left to right), 2022 (colorized).}
    \label{fig:some_ref} 
\end{figure} 




 





% ===========================================