\section{Methods}\label{sec:methods}


\subsection{\Nanna{(appropriate title)} - The Penning trap}



\begin{figure}[h!]
    \includegraphics[width=\linewidth]{Trap_schematic.png}
    \caption{Schematic of the Penning trap. The cross-section is in the $xz$-plane. The ring electrode (green) and end caps (dark blue) sets the inhomogenoeous electric field (white arrows). The homogeneous magnetic field (blue arrows) is resulting from the cylinder magnet (light blue) outside of the trap. For simplicity, we will assume that the trap extends the distance $d$ in all directions, see text for further details.}
    \label{fig:schematic_trap}
\end{figure}

Consider an ideal Penning trap with characteristic dimension $d$, a length scale given by $d= \sqrt{z_0^2+ \frac{1}{2}r_0^2}$, where $z_0$ is the distance between the centre and the endcap and $r_0$ the radius of of the ring, as in Figure \ref{fig:schematic_trap}. These electrodes create an external electric field $\vec{E}_\mathrm{ext}$ resulting from the potential

\begin{equation}
    V (\vec{r})= V(x,y,z) = \frac{V_0}{2d^2}(2z^2 - x^2 - y^2), \label{eq:penning_potential}
\end{equation}

where $V_0$ is the potential applied to the electrodes. The external electric field is then 

\begin{equation}
    \vec{E}_\mathrm{ext} = -\nabla V = \frac{V_0}{d^2} (x,\,y,\,-2z). \label{eq:p1_E_field_pot_gradient}
\end{equation}

A constant homogeneous magnetic field $\vec{B}_\mathrm{ext}$ is imposed in the $z$-direction,
\begin{equation}
    \vec{B}_\mathrm{ext} = (0,0,B_0). \label{eq:homogeneous_B_field}
\end{equation}

An important simplification to note is that we regard the extent of the trap as a sphere of radius $d$. This means that in fact

\begin{equation}\label{eq:trap_ext_E_field}
    \vec{E}_\mathrm{ext} (\vec{r}) = 
    \begin{cases}
        \frac{V_0}{d^2} \big( x\hat{e}_x + y\hat{e}_y -2z\hat{e}_z \big), & \abs{\vec{r}} \leq d \\
        \vec{0},  & \abs{\vec{r}} > d
    \end{cases}
\end{equation}

and

\begin{equation}\label{eq:trap_ext_B_field}
    \vec{B}_\mathrm{ext} (\vec{r}) = 
    \begin{cases}
        B_0\hat{e}_z, & \abs{\vec{r}} \leq d \\
        \vec{0}, & \abs{\vec{r}} > d
    \end{cases}.
\end{equation}



\subsection{Coulomb interactions}

A particle of charge $q_1$ at position $\vec{r}_1$ exerts a force on another particle of charge $q_2$ at $\vec{r}_2$ equal $k_e q_1q_2 (\vec{r}_2-\vec{r}_1) / \abs{\vec{r}_2-\vec{r}_1}^3$, where $k_e$ is the Boltzmann constant. When there are several particles present in the trap, we will want to include the contribution from these forces in the equation. The interaction field $\vec{E}_\mathrm{int}$ set up by $N_\mathrm{p}$ point charges $\{q_1, q_2, \dots, q_{N_\mathrm{p}}\}$ at positions $\{\vec{r}_1, \vec{r}_2, \dots, \vec{r}_{N_\mathrm{p}}\}$ is given by

\begin{equation}\label{eq:interaction_field}
    \vec{E}_\mathrm{int} (\vec{r}) = k_e \sum_{p=1}^{N_\mathrm{p}} q_p \frac{\vec{r}-\vec{r}_p}{\abs{\vec{r}-\vec{r}_p}^3}.
\end{equation}

The Coulomb interactions give rise to non-linear differential equations which typically require a numerical approach. Note that this is a comprehensive task when introducing many particles, even for the computer. 

The resulting electric and magnetic field contributing to the Lorentz force in eq. \eqref{eq:lorentz_force} is then $\vec{E} = \vec{E}_\mathrm{ext} + \vec{E}_\mathrm{int}$ and $\vec{B} = \vec{B}_\mathrm{ext}$, respectively, from eqs. \eqref{eq:trap_ext_E_field}, \eqref{eq:trap_ext_B_field} and \eqref{eq:interaction_field}.




\subsection{\Nanna{(appropriate title)} Analytical work}

We restrict ourselves to dealing with only positively charged particles, $q>0$. Consider the single-particle case, i.e. the situation with only one particle present in the trap or when we look at one of several particles that do not interact\footnote{This is not physical.}. Now, the only contributions to the fields are external, hence $\vec{E} = \vec{E}_\mathrm{ext}$ and $\vec{B} = \vec{B}_\mathrm{ext}$. We use eq. \eqref{eq:lorentz_force} to compute the Lorentz force,


\begin{align}
    \vec{F} &= \frac{q V_0}{d^2}(x,\,y,\,-2z) + q B_0 (\dot{y},\,-\dot{x},\,0) \nonumber\\
    &= \frac{m}{2}\omega_z^2 (x,\,y,\,-2z) + m \omega_0 (\dot{y},\,-\dot{x},\,0),\label{eq:p1_lorentz_force_penning}
\end{align}

where we defined $\omega_0 \equiv \frac{qB_0}{m}$ and $\omega_z^2 \equiv \frac{2qV_0}{m d^2}$. The equation of motion in eq. \eqref{eq:eom_lorentz} are three differential equations in each spatial component,

\begin{subequations}\label{eq:p1_eom_xyz}
    \begin{align}
        \ddot{x} - \omega_0 \dot{y} - \frac{1}{2} \omega_z^2 x &= 0, \label{eq:p1_eom_x} \\ 
        \ddot{y} + \omega_0 \dot{x} - \frac{1}{2} \omega_z^2 y &= 0, \label{eq:p1_eom_y} \\ 
        \ddot{z} + \omega_z^2 z &= 0. \label{eq:p1_eom_z}
    \end{align}
\end{subequations}


The general solution of eq. \eqref{eq:p1_eom_z} is

\begin{equation}
    z(t) = c_1 \cos(\omega_z t) + c_2 \sin(\omega_z t), \label{eq:p1_diffeq_solution_z}
\end{equation}

where $c_1, c_2 \in \mathbb{R}$ are determined by initial conditions.

The other two, eqs. \eqref{eq:p1_eom_x} and \eqref{eq:p1_eom_y}, are coupled. We introduce a complex function $f(t)=x(t) + iy(t)$ for which we have the time derivatives $\dot{f} = \dot{x} + i\dot{y}$ and $\ddot{f} = \ddot{x} + i\ddot{y}$. Multiply equation \eqref{eq:p1_eom_y} with $i$ and add to equation \eqref{eq:p1_eom_x} to obtain the following:

\begin{align}
    \ddot{x} + i\ddot{y} + \omega_0(i\dot{x} - \dot{y}) -\frac{1}{2}\omega_z^2(x+iy) &= 0 \nonumber \\
    (\ddot{x} + i\ddot{y}) + i\omega_0(\dot{x} +i \dot{y}) -\frac{1}{2}\omega_z^2(x+iy) &= 0. \label{eq:p2_eom_rewritten}
\end{align}

The single differential equation in terms of the complex function $f(t)$ is therefore

\begin{equation}
    \ddot{f} + i\omega_0 \dot{f} - \frac{1}{2}\omega_z^2 f = 0. \label{eq:p2_f_complex_equation}
\end{equation}

The general solution to equation \eqref{eq:p2_f_complex_equation} is 
\begin{equation}
    f(t) = A_+ e^{-i(\omega_+ t + \phi_+)} + A_- e^{-i(\omega_- t + \phi_-)}, \label{eq:p3_f_general_solution}
\end{equation}
where $\phi_+$ and $\phi_-$ are constant phases, $A_+$ and $A_-$ are positive amplitudes and 
\begin{equation}
    \omega_\pm = \frac{\omega_0 \pm \sqrt{\omega_0^2 - 2\omega_z^2}}{2}. \label{eq:p3_omega_pm}
\end{equation}

The $\omega_+$ is the modified cyclotron frequency and the $\omega_-$ is the magnetron frequency that composes the two modes in the orbital motion we will discuss later.


We see from equation \eqref{eq:p3_f_general_solution} that if $\omega_\pm$ is real, then $\abs{f(t)} < \infty$ as $t\to\infty$ and we have a bounded solution in the $xy$-plane. For $\omega_\pm$ to be real we see from equation \eqref{eq:p3_omega_pm} that the following must hold:

\begin{equation}\label{eq:p3_omega_constraints}
    \omega_0^2 \geq 2\omega_z^2  \quad \implies \quad \frac{q}{m} \geq \frac{4V_0}{(B_0d)^2}
\end{equation} 

We may regard equation \eqref{eq:p3_f_general_solution} as a sum of vectors $\vec{r}_\pm$ with amplitudes $\abs{\vec{r}_\pm}=A_\pm$ and directions given by the angles $\alpha_\pm = \omega_\pm t + \phi_\pm$ in the complex plane that represents our $xy$-plane. The magnitude $\abs{f} \equiv \abs{f(t)}$ for this sum of vectors is

\begin{equation}\label{eq:p3_magnitude_f}
    \abs{f} = \sqrt{\abs{\vec{r}_+}^2 + \abs{\vec{r}_-}^2 + 2\abs{\vec{r}_+}\abs{\vec{r_-}}\cos{(\abs{\alpha_+-\alpha_-})}}.
\end{equation}

The maximum distance from the origin, $R_+$, occurs when the two vectors are pointing in the same direction, i.e. $\abs{\alpha_+-\alpha_-} = 0 \implies \cos0=1$, giving 


\begin{align}
    R_+ &= \sqrt{A_+^2 + A_-^2 + 2A_+A_-} = \sqrt{(A_+ + A_-)^2}\nonumber \\
    &= A_+ + A_-. \label{eq:p4_Rplus}   
\end{align}

Similarily, the minimum distance, $R_-$, is achieved when the vectors are pointing in opposite directions, i.e. $\abs{\alpha_+-\alpha_-} = \pi \implies \cos\pi=-1$ which yields

 
\begin{align}
    R_- &= \sqrt{A_+^2 + A_-^2 - 2A_+A_-} = \sqrt{(A_+ - A_-)^2}\nonumber \\
    &= \abs{A_+ - A_-}. \label{eq:p4_Rminus}  
\end{align}


\Nanna{Where do we use this?}


%\begin{align}
    %\ddot{x}_i - \omega_{0,i} \dot{y}_i - \frac{1}{2}\omega_{z,i}^2 x_i - k_e \frac{q_i}{m_i} \sum_{j\neq i} q_j \frac{x_i - x_j}{\abs{\vec{r}_i - \vec{r}_j}^3} =0, \label{eq:p4_diffeq_many_particles_x} \\
    %\ddot{y}_i + \omega_{0,i} \dot{x}_i - \frac{1}{2}\omega_{z,i}^2 y_i - k_e \frac{q_i}{m_i} \sum_{j\neq i} q_j \frac{y_i - y_j}{\abs{\vec{r}_i - \vec{r}_j}^3} =0, \label{eq:p4_diffeq_many_particles_y} \\ 
    %\ddot{z}_i + \omega_{z,i}^2 z_i - k_e \frac{q_i}{m_i} \sum_{j\neq i} q_j \frac{z_i - z_j}{\abs{\vec{r}_i - \vec{r}_j}^3} =0, \label{eq:p4_diffeq_many_particles_z}
%\end{align}







\subsection{\Nanna{(appropriate title)} Numerical work}\label{sec:code}

We aim to create a program in \CC\, that simulates a set of $N_\mathrm{p}$ particles inside a Penning trap. An object-oriented code is befitting this task and we present in the following descriptions of the classes \verb|Particle| and \verb|PenningTrap|.

The purpose of \texttt{Particle} is to hold the parameters, such as the position, of a particle. We let an object of this class be initialised with a charge $q$, a mass $m$, a position $\vec{r} = (x,y,z)$ and a velocity $\vec{v} = (v_x, v_y, v_z)$. We add functions to update the latter two. 

The \texttt{PenningTrap}-class imitates the physical system that is the Penning trap of external electric and magnetic fields according to eqs. \eqref{eq:trap_ext_E_field} and \eqref{eq:trap_ext_B_field}, and characteristic dimension $d$. It is friend with the \texttt{Particle}-class and, in order to resemble the physical situation as much as possible, accepts only input particles of this type. When filled with $N_\mathrm{p}\geq 1$, an object of \texttt{PenningTrap} is ready to simulate the evolution of the \texttt{Particle} object(s) for a given period of time and time step size with either a forward Euler or a fourth order Runge-Kutta numerical scheme. 

\Nanna{Some more here?}

\texttt{PenningTrap} includes several additional functionalites. Amongst other things, the class

\begin{itemize}
    \item offers the choice between RK4 or FE integration method.
    \item provides the option to include or exclude Coloumb interactions in the simulation, as they are computationally greedy.
    \item includes a method for adding a time-dependent perturbation to the external electric potential, $V_0\rightarrow V_0 (1+ f(1+\cos{(\omega_V t)}))$.
    \item has the ability to count the number of particles still trapped, that is the number particles whose position $\vec{r}$ is such that $\abs{\vec{r}} \leq d$.
    \item can generate a set of identical particles with positions and velocities that are normally distrubuted within the trap's dimensions.
\end{itemize}

\subsubsection*{Numerical integration}
The equation of motion \eqref{eq:eom_lorentz} is split into two first order ODEs as follows:

\begin{equation}\label{eq:split_ODEs}
    \begin{split}
        \dot{\vec{v}} &= \frac{\vec{F}}{m} \\
        \dot{\vec{r}} &= \vec{v}
    \end{split}
\end{equation}

We then get six discretised equations to solve for each particle at each step in time.

The simple FE algorithm \citep{Arkinson1989} \Nanna{...} The FE method is a first order, single-step method, meaning that we only need the current numerical solution to compute the next \Nanna{(Objection, relevance?)}.

The fourth order, single-step RK4 method uses four estimates of the gradient on the interval between to steps in time. It gives rise to much more precise solutions than what the FE method does, but the algorithm requires more computational time. For the RK4 algorithm \citep{Arkinson1989}, we have to keep some things in mind when considering forces between particles. For a singlw step in time, we must make sure that the first gradient estimate is calculated for all particles before moving on to the second estimate, and so on. 


\subsection{Setup and simulation \Nanna{or some other title}} 

Our Penning trap is defined by the following numerical values:

\begin{itemize}
    \item[] $d=500 \,\mathrm{\text{\textmu} m}$
    \item[] $V_0=25.0 \,\mathrm{mV}$
    \item[] $B_0=1.00 \,\mathrm{T}$
\end{itemize}

We convert all to this set of base units:

\begin{itemize}
    \item[] Unit-length $=$ one micrometre $= 1\, \mathrm{\text{\textmu}m} $ 
    \item[] Unit-time $=$ one microsecond $= 1\, \mathrm{\text{\textmu}s} $
    \item[] Unit-mass $=$ one atomic mass unit $= 1\, \mathrm{u} $
    \item[] Unit-charge $=$ one elementary charge $= 1\, \mathrm{e} $
\end{itemize}

We intend to run the simulation for two different durations:
\begin{itemize}
    \item[] $T_1 = $ 50 \textmu s
    \item[] $T_2 = $ 500 \textmu s
\end{itemize}
and four different numbers of time steps:
\begin{itemize}
    \item[] $n_1 = 4000$
    \item[] $n_2 = 8000$
    \item[] $n_3 = 16000$
    \item[] $n_4 = 32000$
\end{itemize}


\subsection{Initial conditions}\label{sec:initial_conditions}

The particles we consider are Calcium ions (Ca$^+$) with charge $q=1 \,\mathrm{e}$ and mass $m = 40.078 \,\mathrm{u}$. 

For simulations of one or two particles, we use the following initial conditions:
\begin{itemize}
    \item Particle 1\label{item:initial_conditions_p1}
    \begin{itemize}
        \item[] $\vec{r}^{(1)}_0 = (20, \,0, \,20) \, \mathrm{\text{\textmu}m} $
        \item[] $\vec{v}^{(1)}_0 = (0,\,25,\,0) \, \mathrm{\text{\textmu}m}/ \mathrm{\text{\textmu}s}$
    \end{itemize}
    \item Particle 2\label{item:initial_conditions_p2}
    \begin{itemize}
        \item[] $\vec{r}^{(2)}_0 = (25, \,25, \,0) \, \mathrm{\text{\textmu}m} $
        \item[] $\vec{v}^{(2)}_0 = (0,\,40,\,5) \, \mathrm{\text{\textmu}m}/ \mathrm{\text{\textmu}s}$
    \end{itemize}
\end{itemize}

For simulations with more than two particles, we initialise the Calcium ions randomly in the sense that their initial positions and velocities are normally distrubuted around the centre of the trap.





































% ===========================================
%\subsection*{The algorithm}


% Started drafting:
% Kinda lost track of what i am trying to do

% \begin{figure}
%     \begin{algorithm}[H]
%     \scriptsize
%     \caption{Forward Euler (FE)}
%     \label{algo:FE}
%         \begin{algorithmic}
%             %\Procedure{FE differentiation}{$f, a, b, n$}
%             \State $U_0 \leftarrow \dots$        \Comment{Initialise velocities in cube} % matrix of row vectors v_0
%             \State $R_0 \leftarrow \dots$        \Comment{Initialise positions in cube}
%             \State $h \leftarrow (t_n-t_0)/n$  \Comment{Compute the time step length}
%             \For{$i = 0, 1, \ldots, n-1$}
%             \For{each particle}
%             \State $\vec{v} \leftarrow U_{i,\text{particle}}$
%             \State $\vec{r} \leftarrow R_{i,\text{particle}}$
%             \State $\text{d}U_{i, \text{particle}} \leftarrow h \vec{F}(\vec{r}, \vec{v})/m$
%             \State $\text{d}R_{i, \text{particle}} \leftarrow h (\vec{v} + \text{d}U_{i, \text{particle}})$ 
%             \EndFor

%             \State $U_{i+1} \leftarrow U_{i} + \text{d}U_{i}$ \Comment{Update velocities}
%             \State $R_{i+1} \leftarrow R_{i} + \text{d}R_{i}$ \Comment{Update positions}
%             \EndFor
%             %\EndProcedure
%         \end{algorithmic}
%     \end{algorithm}
% \end{figure}

% \begin{figure}
%     \begin{algorithm}[H]
%     \scriptsize
%     \caption{4$^\text{th}$ order Runge-Kutta (RK4)}
%     \label{algo:RK4}
%         \begin{algorithmic}
%             %\Procedure{RK4 differentiation}{$\vec{F}, t_n, t_0, n$}
%             \State $U_0 \leftarrow \dots$        \Comment{Initialise velocities in cube} % matrix of row vectors v_0
%             \State $R_0 \leftarrow \dots$        \Comment{Initialise positions in cube}
%             \State $h \leftarrow (t_n-t_0)/n$  \Comment{Compute the time step length}
%             \For{$i = 0, 1, \ldots, n-1$}
%             \For{each particle}
%             \State $\vec{v} \leftarrow U_{i,\text{particle}}$
%             \State $\vec{r} \leftarrow R_{i,\text{particle}}$
%             \State $\vec{k}_{\vec{v},1} \leftarrow h \vec{F}(\vec{r}, \vec{v})/m$ \Comment{...}
%             \State $\vec{k}_{\vec{r},1} \leftarrow h \vec{v}$
%             \State $\vec{k}_{\vec{v},2} \leftarrow h \vec{F}(\vec{r}+1/2\vec{k}_{\vec{r},1}, \vec{v}+1/2\vec{k}_{\vec{v},1})/m$ \Comment{...}
%             \State $\vec{k}_{\vec{r},2} \leftarrow h (\vec{v}+1/2\vec{k}_{\vec{v},1})$ 
%             \State $\vec{k}_{\vec{v},3} \leftarrow h \vec{F}(\vec{r}+1/2\vec{k}_{\vec{r},2}, \vec{v}+1/2\vec{k}_{\vec{v},2})/m$ \Comment{...}
%             \State $\vec{k}_{\vec{r},3} \leftarrow h (\vec{v}+1/2\vec{k}_{\vec{v},2})$ 
%             \State $\vec{k}_{\vec{v},4} \leftarrow h \vec{F}(\vec{r}+\vec{k}_{\vec{r},3}, \vec{v}+\vec{k}_{\vec{v},3})/m$ \Comment{...}
%             \State $\vec{k}_{\vec{r},4} \leftarrow h (\vec{v}+\vec{k}_{\vec{v},3})$ 
%             \
%             \State $\text{d}U_{i, \text{particle}} \leftarrow (\vec{k}_{\vec{v},1} + 2\vec{k}_{\vec{v},2} + 2\vec{k}_{\vec{v},3} + \vec{k}_{\vec{v},4})/6$
%             \State $\text{d}R_{i, \text{particle}} \leftarrow (\vec{k}_{\vec{r},1} + 2\vec{k}_{\vec{r},2} + 2\vec{k}_{\vec{r},3} + \vec{k}_{\vec{r},4})/6$
%             \EndFor

%             \State $U_{i+1} \leftarrow U_{i} + \text{d}U_{i}$ \Comment{Update velocities}
%             \State $R_{i+1} \leftarrow R_{i} + \text{d}R_{i}$ \Comment{Update positions}
%             \EndFor
%             \State $...$  \Comment{...}
%             %\EndProcedure
%         \end{algorithmic}
%     \end{algorithm}
% \end{figure}
    

%
%use equation~\cite{midpoint_rule}
%\begin{figure}
%% NOTE: We only need \begin{figure} ... \end{figure} here because of a compatability issue between the 'revtex4-1' document class and the 'algorithm' environment.
    %\begin{algorithm}[H]
    %\caption{Midpoint rule for integration}
    %\label{algo:midpointrule}
        %\begin{algorithmic}
            %\Procedure{Midpoint rule}{$f, a, b, n$}
            %\State $I \leftarrow 0$        \Comment{Initialize the integral variable}
            %\State $h \leftarrow (b-a)/n$  \Comment{Compute the interval length}
            %\For{$i = 1, 2, \ldots, n$}
            %\State $x \leftarrow a + (i-1/2)h$  \Comment{Assign $x$ to the midpoint}  %This means x is assigned the value x + ih/2.
            %\State $I \leftarrow I + f(x)$  \Comment{Add contribution to integral} %Assign I to I + f(x)
            %\EndFor
            %\State $I \leftarrow Ih$  \Comment{Finalize the computation}
            %\EndProcedure
        %\end{algorithmic}
    %\end{algorithm}
%\end{figure}


% ===========================================
