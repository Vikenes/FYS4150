\documentclass[english,notitlepage,reprint,nofootinbib]{revtex4-1}  % defines the basic parameters of the document
% For preview: skriv i terminal: latexmk -pdf -pvc filnavn
% If you want a single-column, remove "reprint"

% Allows special characters (including æøå)
\usepackage[utf8]{inputenc}
% \usepackage[english]{babel}


\usepackage{physics,amssymb}  % mathematical symbols (physics imports amsmath)
\usepackage{graphicx}         % include graphics such as plots
\usepackage{xcolor}           % set colors
\usepackage{hyperref}         % automagic cross-referencing
\usepackage{listings}         % display code
\usepackage{subcaption}
% \captionsetup{compatibility=false}   
% \usepackage[caption=false]{subfigure}        % imports a lot of cool and useful figure commands    
\usepackage{float}
%\usepackage[section]{placeins}
\usepackage{algorithm}
\usepackage[noend]{algpseudocode}
\usepackage{tikz}
\usepackage{nth}
\usepackage{silence}


\WarningFilter{revtex4-1}{Repair the float}
\usetikzlibrary{quantikz}
% defines the color of hyperref objects
% Blending two colors:  blue!80!black  =  80% blue and 20% black
\hypersetup{ % this is just my personal choice, feel free to change things
    colorlinks,
    linkcolor={red!50!black},
    citecolor={blue!50!black},
    urlcolor={blue!80!black}}


% ===========================================
% New commands here:
\newcommand{\closed}[1]{\left( #1 \right)}
\newcommand{\bracket}[1]{\left[ #1 \right]}
\newcommand{\eps}{\epsilon}
\renewcommand{\vec}{\mathbf}
\newcommand{\svec}{\boldsymbol}
\newcommand{\CC}{C\nolinebreak\hspace{-.05em}\raisebox{.4ex}{\tiny\bf +}\nolinebreak\hspace{-.10em}\raisebox{.4ex}{\tiny\bf +}}

\newcommand{\mum}{\mathrm{\text{\textmu}m}}
\newcommand{\mus}{\mathrm{\text{\textmu}s}}
\newcommand{\mums}{\mathrm{\text{\textmu}m}/ \mathrm{\text{\textmu}s}}

\newcommand{\Johan}[1]{\textcolor{blue}{#1}}
\newcommand{\Vetle}[1]{\textcolor{orange}{#1}}
\newcommand{\Nanna}[1]{\textcolor{green}{#1}}
\newcommand{\alert}[1]{\textcolor{red}{\textbf{#1}}}

\newcommand{\unit}[1]{\,\mathrm{#1}}
\newcommand{\tunit}{\,J/k_B}
\newcommand{\eunit}{\,J}
\newcommand{\bj}{\beta J}
\newcommand{\avge}{\expval{\epsilon}}
\newcommand{\avgm}{\expval{\abs{m}}}
\newcommand{\nmc}{N_\mathrm{MC}}
\newcommand{\nequi}{N_\mathrm{eq}}

\newcommand{\midrule}{\colrule}
\newcommand{\bottomrule}{\botrule}


\newcommand{\im}{\mathrm{i}} % imaginary unit
% ===========================================



%   New environments here:

\newenvironment{itemize*}{
    \begin{itemize}
        \setlength{\itemsep}{0pt}
        \setlength{\parskip}{0pt}
}{
\end{itemize}
}

\graphicspath{{../output/plots/}}

\begin{document}

\title{Project 3 - FYS4150} 
\author{Nanna Bryne, Johan Mylius Kroken, Vetle A. Vikenes} 
\date{\today}                             
\noaffiliation                            

%This is how we create an abstract section.
\begin{abstract}
    The Penning trap is a device that uses a strong axial magnetic field together with a quadrupole electric field to capture charged particles within a spatial region. We model such a trap by assuming spherical symmetry, and consider an external electric and magnetic field, alongside an internal electric field set up by charged particles (Ca$^+$ ions) in the trap. We neglect the self induced magnetic field set up by moving charges. The equations of motion are determined by the Lorentz force, and we solve them numerically using two different integration schemes: The Forward Euler and Runge-Kutta 4 schemes. These methods offer different trade offs between computational expense and numerical accuracy, yielding a global error of $\mathcal{O}(h)$ and $\mathcal{O}(h^4)$ respectively (for step size $h$). We check and confirm these errors by simulating one single particle for 50 $\mus$ for different time steps using both schemes. $h$ is small, thus the Runge-Kutta 4 scheme offers a far lower global error and is used when we make a simulation of two particles, where we also investigate the effect of the Coulomb interaction between particles. The trajectories of the particles, both in the $xy$-plane and in the phase spaces of $x$ and $z$ change when we allow for particle interaction. We also explore how 100 randomly distributed particles behave under the influence of a time dependant sinusoidal perturbation to the external electric potential. We simulate this for different amplitudes and frequencies, and find that certain resonance frequencies of this perturbation has the ability to completely empty the trap. These simulations are computationally very expensive and lengthy. However, we obtain results indicating that the simulation become unstable for applied frequencies of $\omega_V\in\{0.7,1.4, 2.2\}$ MHz. For a larger applied amplitude $f$, we see broader resonance lines centred around these resonance frequencies. 
\end{abstract}
\maketitle


% ===========================================
\section{Introduction}\label{sec:introduction}


Blah blah 



For a single, non-relativistic particle with mass $m_\mathrm{P}$ in a two-dimensional potential $V(t, \vec{x})$, the Schrödinger equation reads\footnote{In position space, that is. \Nanna{Should we comment on this?}}
\begin{equation}\label{eq:2d_schrodinger}
    \im \hbar \pdv{}{t}\Psi(t, \vec{x}) = -\frac{\hbar^2}{2m_\mathrm{P}}\nabla^{\!2} \Psi(t, \vec{x})  + V(t, \vec{x}) \Psi(t, \vec{x}).
\end{equation}
For a set of initial and boundary conditions, the partial differential equation (PDE) describes the temporal and spatial evolution of the complex-valued function $\Psi(t, \vec{x})$ related to the quantum state of the aforementioned particle. In such a case, at a time $t$, the probability density for an experimentalist to locate the particle at $\vec{x}$ \Nanna{("for detecting ..." is better, but I don't want to copy Anders)} is \Nanna{large P or small p???}
\begin{equation}\label{2d_bornrule}
    P(\vec{x}; \, t) = \abs{\Psi(t, \vec{x})}^2 = \Psi^{\!*}(t, \vec{x}) \Psi(t, \vec{x}),
\end{equation} 
originating from the Born rule; \Nanna{fill me}

In this paper we will consider a dimensionless time-independent potential, i.e. we let $V(t, \vec{x}) \to v(\vec{x})$. The specifics of the scaling do not concern us in this paper, and we simply rewrite equation \eqref{eq:2d_schrodinger} to the dimensionless equation
\begin{equation}\label{eq:2d_schrodinger_dimless}
    \im \pdv{}{t}u(t, \vec{x}) = - \nabla^{\!2}u(t, \vec{x})+ v(\vec{x})u(t, \vec{x}),
\end{equation}
where we substituted $\Psi(t, \vec{x}) \to u(t, \vec{x})$. In equation \eqref{eq:2d_schrodinger_dimless} all variables are dimensionless. When demanding the proper normalisation on $u(t,\vec{x})$, it follows that the Born rule now takes the form of
\begin{equation}\label{eq:2d_bornrule_dimless}
    p(\vec{x}; \, t) = \abs{u(t, \vec{x})}^2 = u^{\!*} (t, \vec{x})u(t, \vec{x}).
\end{equation}
\Nanna{Should maybe rephrase this paragraph.}







% ===========================================

% Let $x^\mu = (x^0,x^j)=(t, \vec{x})$ be the (2+1)-dimensional Lorentz vector describing a position $\vec{x}\in \mathbb{R}^2$ at a time $t$. That is, we let Greek indices run form 0 to 2 and Latin indices run from 1 to 2. \Nanna{Rephrase}


% \begin{equation}\label{eq:2d_schrodinger}
%     \im \hbar \partial^0 \Psi(x) = -\frac{\hbar^2}{2m} \partial^j\partial^j \Psi(x)  + V(x) \Psi(x)
% \end{equation}


% \begin{equation}\label{2d_bornrule}
%     p(x) = \abs{\Psi(x)}^2 = \Psi^{*}(x) \Psi(x) 
% \end{equation}

% \begin{equation}\label{eq:2d_schrodinger_dimless}
%     \im \partial^0 u(x) = - \partial^j \partial^j u(x)+ v(\vec{x})u(x)
% \end{equation}


% \begin{equation}\label{eq:2d_bornrule_dimless}
%     p(x) = \abs{u(x)}^2 = u^{*} (x)u(x)
% \end{equation}




\section{Methods}\label{sec:methods}


We aim to create a program in \CC\, that simulates a set of $N_\mathrm{p}$ particles inside a Penning trap. An object-oriented code is befitting this task and we present in the following descriptions of the classes \verb|Particle| and \verb|PenningTrap|.

The purpose of \texttt{Particle} is to hold the parameters, such as the position, of a particle. We let an object of this class be initialised with a charge $q$, a mass $m$, a position $\vec{r} = (x,y,z)$ and a velocity $\vec{v} = (v_x, v_y, v_z)$. We add functions to update the latter two. 

The \texttt{PenningTrap}-class imitates the physical system that is the Penning trap of magnetic field $\vec{B}=B_0\hat{e}_z$, electric potential $V = \frac{V_0}{2d^2} (2z^2-x^2-y^2)$ and characteristic dimension $d$. It is friend with the \texttt{Particle}-class and, in order to resemble the physical situation as much as possible, accepts only input particles of this type. When filled with $N_\mathrm{p}\geq 1$, an object of \texttt{PenningTrap} is ready to simulate the evolution of the \texttt{Particle} object(s) for a given period of time and time step with either a Forward-Euler or a \nth{4} order Runge-Kutta numerical scheme. 

\dots


\verb|PenningTrap| offers several additional functionalites. Amongst other things, the class:

\begin{itemize}
    \item offers the choice between \nth{4} order Runge-Kutta or Forward-Euler as integration method %idk
    \item provides the option to include or exclude Coloumb interactions in the simulation, as they are computationally greedy
    \item includes a method for adding a time-dependent perturbation to the external electric potential, $V_0\rightarrow V_0 (1+ f(1+\cos{(\omega_V t)}))$
    \item has the ability to count the number of particles still trapped, that is the number particles whose position $\vec{r}$ is such that $\abs{\vec{r}} \leq d$
    \item can generate a set of identical particles with positions and velocities that are normally distrubuted within the trap's dimensions
\end{itemize}




% ===========================================
%\subsection*{The algorithm}


% Started drafting:
% Kinda lost track of what i am trying to do

% \begin{figure}
%     \begin{algorithm}[H]
%     \scriptsize
%     \caption{Forward Euler (FE)}
%     \label{algo:FE}
%         \begin{algorithmic}
%             %\Procedure{FE differentiation}{$f, a, b, n$}
%             \State $U_0 \leftarrow \dots$        \Comment{Initialise velocities in cube} % matrix of row vectors v_0
%             \State $R_0 \leftarrow \dots$        \Comment{Initialise positions in cube}
%             \State $h \leftarrow (t_n-t_0)/n$  \Comment{Compute the time step length}
%             \For{$i = 0, 1, \ldots, n-1$}
%             \For{each particle}
%             \State $\vec{v} \leftarrow U_{i,\text{particle}}$
%             \State $\vec{r} \leftarrow R_{i,\text{particle}}$
%             \State $\text{d}U_{i, \text{particle}} \leftarrow h \vec{F}(\vec{r}, \vec{v})/m$
%             \State $\text{d}R_{i, \text{particle}} \leftarrow h (\vec{v} + \text{d}U_{i, \text{particle}})$ 
%             \EndFor

%             \State $U_{i+1} \leftarrow U_{i} + \text{d}U_{i}$ \Comment{Update velocities}
%             \State $R_{i+1} \leftarrow R_{i} + \text{d}R_{i}$ \Comment{Update positions}
%             \EndFor
%             %\EndProcedure
%         \end{algorithmic}
%     \end{algorithm}
% \end{figure}

% \begin{figure}
%     \begin{algorithm}[H]
%     \scriptsize
%     \caption{4$^\text{th}$ order Runge-Kutta (RK4)}
%     \label{algo:RK4}
%         \begin{algorithmic}
%             %\Procedure{RK4 differentiation}{$\vec{F}, t_n, t_0, n$}
%             \State $U_0 \leftarrow \dots$        \Comment{Initialise velocities in cube} % matrix of row vectors v_0
%             \State $R_0 \leftarrow \dots$        \Comment{Initialise positions in cube}
%             \State $h \leftarrow (t_n-t_0)/n$  \Comment{Compute the time step length}
%             \For{$i = 0, 1, \ldots, n-1$}
%             \For{each particle}
%             \State $\vec{v} \leftarrow U_{i,\text{particle}}$
%             \State $\vec{r} \leftarrow R_{i,\text{particle}}$
%             \State $\vec{k}_{\vec{v},1} \leftarrow h \vec{F}(\vec{r}, \vec{v})/m$ \Comment{...}
%             \State $\vec{k}_{\vec{r},1} \leftarrow h \vec{v}$
%             \State $\vec{k}_{\vec{v},2} \leftarrow h \vec{F}(\vec{r}+1/2\vec{k}_{\vec{r},1}, \vec{v}+1/2\vec{k}_{\vec{v},1})/m$ \Comment{...}
%             \State $\vec{k}_{\vec{r},2} \leftarrow h (\vec{v}+1/2\vec{k}_{\vec{v},1})$ 
%             \State $\vec{k}_{\vec{v},3} \leftarrow h \vec{F}(\vec{r}+1/2\vec{k}_{\vec{r},2}, \vec{v}+1/2\vec{k}_{\vec{v},2})/m$ \Comment{...}
%             \State $\vec{k}_{\vec{r},3} \leftarrow h (\vec{v}+1/2\vec{k}_{\vec{v},2})$ 
%             \State $\vec{k}_{\vec{v},4} \leftarrow h \vec{F}(\vec{r}+\vec{k}_{\vec{r},3}, \vec{v}+\vec{k}_{\vec{v},3})/m$ \Comment{...}
%             \State $\vec{k}_{\vec{r},4} \leftarrow h (\vec{v}+\vec{k}_{\vec{v},3})$ 
%             \
%             \State $\text{d}U_{i, \text{particle}} \leftarrow (\vec{k}_{\vec{v},1} + 2\vec{k}_{\vec{v},2} + 2\vec{k}_{\vec{v},3} + \vec{k}_{\vec{v},4})/6$
%             \State $\text{d}R_{i, \text{particle}} \leftarrow (\vec{k}_{\vec{r},1} + 2\vec{k}_{\vec{r},2} + 2\vec{k}_{\vec{r},3} + \vec{k}_{\vec{r},4})/6$
%             \EndFor

%             \State $U_{i+1} \leftarrow U_{i} + \text{d}U_{i}$ \Comment{Update velocities}
%             \State $R_{i+1} \leftarrow R_{i} + \text{d}R_{i}$ \Comment{Update positions}
%             \EndFor
%             \State $...$  \Comment{...}
%             %\EndProcedure
%         \end{algorithmic}
%     \end{algorithm}
% \end{figure}
    

%
%use equation~\cite{midpoint_rule}
%\begin{figure}
%% NOTE: We only need \begin{figure} ... \end{figure} here because of a compatability issue between the 'revtex4-1' document class and the 'algorithm' environment.
    %\begin{algorithm}[H]
    %\caption{Midpoint rule for integration}
    %\label{algo:midpointrule}
        %\begin{algorithmic}
            %\Procedure{Midpoint rule}{$f, a, b, n$}
            %\State $I \leftarrow 0$        \Comment{Initialize the integral variable}
            %\State $h \leftarrow (b-a)/n$  \Comment{Compute the interval length}
            %\For{$i = 1, 2, \ldots, n$}
            %\State $x \leftarrow a + (i-1/2)h$  \Comment{Assign $x$ to the midpoint}  %This means x is assigned the value x + ih/2.
            %\State $I \leftarrow I + f(x)$  \Comment{Add contribution to integral} %Assign I to I + f(x)
            %\EndFor
            %\State $I \leftarrow Ih$  \Comment{Finalize the computation}
            %\EndProcedure
        %\end{algorithmic}
    %\end{algorithm}
%\end{figure}


% ===========================================


\section{Results}\label{sec:results}


\begin{figure}[!ht]
    \includegraphics[width=0.9\linewidth]{temp/equilibriation_time_energy.png}
    \caption{Hei}
    \label{fig:equil_energy}
\end{figure} 



\begin{figure}[!ht]
    \includegraphics[width=0.9\linewidth]{temp/equilibriation_time_magnetization.png}
    \caption{Sug meg}
    \label{fig:equil_magn}
\end{figure}


\begin{figure}[!ht]
    \includegraphics[width=0.9\linewidth]{temp/histogram.png}
    \caption{Ser man det }
    \label{fig:histogram}
\end{figure}


\begin{figure}[!ht]
    \includegraphics[width=0.9\linewidth]{temp/phase_transitionNT50.png} 
    \caption{Hvem vinner VM?}
    \label{fig:phase_transition}
\end{figure} 

\begin{figure}[!ht]
    \includegraphics[width=0.9\linewidth]{temp/phase_transitionNT101.png} 
    \caption{Spania?}
    \label{fig:phase_transition_zoomed}
\end{figure} 

\begin{figure}[!ht]
    \includegraphics[width=0.9\linewidth]{temp/critical_temperatures.png} 
    \caption{Spania?}
    \label{fig:critical_temperatures}
\end{figure} 

\begin{figure}[!ht]
    \includegraphics[width=0.9\linewidth]{temp/eficiency.png} 
    \caption{Spania?}
    \label{fig:efficiency}
\end{figure} 




\begin{table}[!ht]
    \begin{tabular}{lrrrr}
\toprule
$N$ & $\langle \epsilon \rangle$ & $\langle \abs{m} \rangle$ & $\langle C_V \rangle$ & $\langle \chi \rangle$ \\
\midrule
$10^{2}$ & -1.940 & 0.980 & 0.466 & 0.058 \\
$10^{3}$ & -1.984 & 0.994 & 0.127 & 0.023 \\
$10^{4}$ & -1.996 & 0.999 & 0.029 & 0.004 \\
$10^{5}$ & -1.996 & 0.999 & 0.029 & 0.003 \\
Analytical & -1.996 & 0.999 & 0.032 & 0.004 \\
\bottomrule 
\end{tabular}

    \caption{Comparison of analytical results. }
    \label{tab:compare_analytical}
\end{table}

\begin{table}[!ht]
    \begin{tabular}{lrrr}
\toprule
$L$ & $\quad T_c^\mathrm{C_V}\,[\tunit]$ & $\quad T_c^\mathrm{linreg}$ [$\tunit$] & $\quad \mathrm{MSE}(C_V/J k_B)$ \\
\midrule
40 & 2.2875 & 2.2873 & 0.0004 \\
60 & 2.2793 & 2.2799 & 0.0008 \\
80 & 2.2764 & 2.2762 & 0.0011 \\
100 & 2.2741 & 2.2740 & 0.0031 \\
$\infty$ &   & 2.2651 &  \\
\bottomrule
\end{tabular}

    \caption{Critical temperatures.}
    \label{tab:critical_temperatures}
\end{table}

% =========================================== 

\section{Discussion}\label{sec:discussion}

\subsection{Single particle}

We provide the movement in the $z$-direction in Figure \ref{fig:compare_z_analytical} as some sort of test to see whether the numerical integrations methods works or not. Hence, we use the fewest number of time steps, $n_1$ in order reveal the largest potential errors. From the figure they seem to overlap perfectly, however there is reason to think the error in the calculated solutions grow with time. This is indeed the case, as the error for the small duration $T_1$ is not visible due to the thickness of plotted lines and the scale of the plot. Although not shown here, if we were to zoom in on the graphs at a late time they do not overlap, i.e. indicating that there is an error although hard to spot. Nevertheless, the calculated solution seems to fit the analytical quite well, but we are still interested in quantifying the error, which we expect to be smaller if we increase the number of time steps. 

Using the analytical solution as reference, the relative error as found from equation \ref{eq:relative_error_size} is shown in Figure \ref{fig:error_FE} for the FE scheme and in Figure \ref{fig:error_RK} for the RK4 scheme. Here, all four time steps $n_k, k\in[1,4]$ is included. The increase in error for both plots as we increase the number of time steps seem to follow the order of the global error for each scheme. This also becomes apparent when considering the actual relative error in the plots ($y$-axis values), and the magnitudinal decrease in error when we increase the number of time steps. The latter is significantly more apparent for the RK4 scheme, which is to be expected as the global error should go as $\mathcal{O}(h^4)$, rather than $\mathcal{O}(h)$ for the FE scheme. There is a noticeable increase in the widths of the different graphs for the RK4 scheme in Figure \ref{fig:error_RK}. This can be explained by the small numerical values of the relative error, and the fact that we use a logarithmic $y$-axis which emphasize the variation for small errors (broad fuzzy looking graph for $n_4$).


\subsection{Two particles}


In the simulations where we use two noninteracting particles, the trajectories are easily predictable as they correspond to two single-particle cases. The movement in the $xy$-plane in this case is shown in the left panel of Figure \ref{fig:xy_two_particles}, whereas the interacting particles' orbital motion is presented in the right panel. The noninteracting particles move in the radial plane according to two modes \Nanna{rephrase}, one with frequency $\omega_+$ orbiting around the magnetic field lines and another with frequency $\omega_-$ that gives rise to the slow orbit around the centre of the trap. The former is called the modified cyclotron motion and the latter is the magnetron motion. The interacting particles show similar tendencies, but the oribiting magnetron motion is shifted a bit in the radial plane due to the particles' repulsion between each other. The radii said orbits tend to increase in both cases, although not very obvoiusly from the figure for Particle 2. The other mode, the modified cyclotron motion, is affected oppositely by the interaction forces for the two particels. The orbit Particle 1 has around the magnetic field lines generally seems to increase in radius where the reciprocal radius for Particle 2 decreases, and vice versa.

\Nanna{Comment about magnetron motion being more unstable the the mod. cyclotron motion?} 

We present the phase plots in the $x$-plane in Figure \ref{fig:x_phase_two_particles}, in which the left panel indicates the periodic motion in one of the radial directions when there are no interaction forces present. The elliptical phases have for both particles a more or less constant size, but move around in the phase space as each particle moves along the $x$-axis. The former phenomena arise from the modified cyclotron motion, whereas the latter is due to the slow magnetron motion. The same nature is seen for the other radial direction, although we do not present the phase plot in the $y$-plane here. Coulomb interactions seem to especially affect the slow orbital motion around the centre as the periodicity of the ellipses in the phase space is quickly reduced \Nanna{rephrase?}. However, after a while, the phase space ellipses change significantly in size as well. This is consistent with what we see in Figure \ref{fig:xy_two_particles}.  

The left panel in Figure \ref{fig:z_phase_two_particles} tells us simply that the axial motion is a harmonic oscillator \Nanna{comment about frequency $\omega_z$?} when the particles are not interacting. In the right panel of the same figure, we see how the repulsion between them affects this oscillating movement. The interacting particles are moving up and down along th $z$-axis with varying ranges and frequencies. We notice that the particles are still confined to a small length in the axial direction. 

The three-dimensional trajectories in Figure \ref{fig:3d_trajectory} substantiate the discussion of periodicity and quasiperiodicity for noninteracting and interacting particles, respectively. In summary, introducing Coulomb forces gives rise to more chaotic and possible unstable motions, for which the original magnetron motion seems especially prone to instabilities.


\Nanna{NOTE: magnetron motion is unstable because energy ass. w this movement is negative, and collisions and perturbations tend to make radius increase (only write if properly discussed and/or cited)}



\subsection{Many particles}

\par - Why these frequencies? 
\par - Effect of increasing amplitude \Nanna{OK i guess}
\par - Effect of Coulomb interactions \Nanna{I think so}
\par - Comment on number of time steps \Nanna{Written, but have some inquiries}

From Figure \ref{fig:trapped_without} we see how an oscillating electric potential gives rise to the escapement of particles in the Penning trap. For some bands of applied frequencies, basically all particles have escaped before the simulation has ended. Increasing the magnitude of the perturbation has the effect of broadening these band widths, as well as to make them occur more frequently. The main bands for which the simulations become unstable are located around $\omega_V \sim 1.4$ MHz, the most obvoius in the figure, and $\omega_V \sim 2.2$ MHz, where it is not obvoius for the smallest amplitude. There is also a good response at $\omega_V\sim 0.7$ MHz, but this is only present for the two larger amplitudes.

Without interactions, it seems to be the case that either all particles escape or stay trapped, referring to the steep graphs in Figures \ref{fig:trapped_without} and \ref{fig:trapped_with}, but especially the latter as we see how it differs from the graph representing simulations with interacting particles. In this figure, interacting particles in a sense push the resonance frequencies upwards: they demand higher frequencies in order for all to escape, but as the noninteracting particles are trapped again, the interacting particles are only partly trapped. We believe that a somewhat broader range of frequencies would show us that very rarely will interacting particles stay trapped here after a period $T_2$, due to instabilities in the magnetron motion arisen by the interaction forces. The steep part of the graph connected with such particles in Figure \ref{fig:trapped_with} is, as mentioned, located at higher frequencies than for the noninteracting particles, which can be due to difference in the eigenfrequencies between these motions.

The time step size is relatively large for the simulations with Coulomb interactions turned on. This could be a source of instabilities, but as these simulations are computationally expensive, we do not explore this theory any further. Referring to our previous error discussion, the relative error arising from the simulations without interacting particles should be around $10^{-6}$, right above the error for $n_1$ in Figure \ref{fig:error_RK}, for a period $T_1$. Assuming an increase rate as in this figure, we should not have an error much larger than $\sim 10^{-4}$ after a time $T_2$, but this is based on our visual estimate. \Nanna{Refer to convergence rate? Comment on the effect of extra forces?}


% ==========================================

\section{Conclusion}\label{sec:conclusion}

We conclude that our implementation and testing of both the FE ad RK4 scheme is correct since the solutions appear accurate in Figure \ref{fig:compare_z_analytical}, and the errors from Figures \ref{fig:error_FE} and \ref{fig:error_RK} behave as we expect them to. We also conclude that the convergence rates found for both methods are satisfying as they resemble their numerical order. 


For the two particle simulation, allowing them to interact pushes the particles further away from each other, while both still show periodic behaviour. This is the physical solution as we are not able to switch off these interaction in a physical Penning trap. The effect we see when including the Coulomb forces is a more chaotic behaviour. 

Introducing a time-dependent perturbation to the electric potential does indeed give rise to resonance phenomena, as certain resonance frequencies empty the trap completely by pushing all particles out of it. We find these frequencies to be bands whose widths are determined by the applied amplitudes $f$ and centred around $\omega_V\in\{0.7, 1.4, 2.2\}$. Without particle interactions, the majority of applied frequencies seem to either leave all the particles contained within the trap, or eject them all. When we include Coulomb interactions, there seems to be slight middle ground where some have escaped and some are still trapped. Thus particles interactions seem to sometimes work against the resonance frequencies, preventing some particles from escaping the trap as quickly as they otherwise would. 

We also question the validity of these results due to the larger time step used in order to speed to computation time. A natural extension of this would be to run the simulation for smaller time steps, although this will increase computation time. Another approach could be to look at a broader range of $\omega_V$ values, with smaller step sized. To better understand which frequencies the resonance behaviour is related to, a natural choice of further exploration is to study the spatial distribution of the escaped particles. By also defining the extent of the trap in terms of spatial coordinates, rather than a fixed radius, resonance effects in different directions can be better understood. It would also be interesting to implement the self induced magnetic field set up by moving charges to see whether it has a major impact on our results. 

% ===========================================

% \onecolumngrid

\bibliographystyle{apalike}
\bibliography{ref} 


\end{document}
