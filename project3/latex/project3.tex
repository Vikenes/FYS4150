\documentclass[english,notitlepage,reprint,nofootinbib]{revtex4-1}  % defines the basic parameters of the document
% For preview: skriv i terminal: latexmk -pdf -pvc filnavn
% If you want a single-column, remove "reprint"

% Allows special characters (including æøå)
\usepackage[utf8]{inputenc}
% \usepackage[english]{babel}


\usepackage{physics,amssymb}  % mathematical symbols (physics imports amsmath)
\include{amsmath}
\usepackage{graphicx}         % include graphics such as plots
\usepackage{xcolor}           % set colors
\usepackage{hyperref}         % automagic cross-referencing
\usepackage{listings}         % display code
\usepackage{subfigure}        % imports a lot of cool and useful figure commands
% \usepackage{float}
%\usepackage[section]{placeins}
\usepackage{algorithm}
\usepackage[noend]{algpseudocode}
\usepackage{subfigure}
\usepackage{tikz}
\usetikzlibrary{quantikz}
% defines the color of hyperref objects
% Blending two colors:  blue!80!black  =  80% blue and 20% black
\hypersetup{ % this is just my personal choice, feel free to change things
    colorlinks,
    linkcolor={red!50!black},
    citecolor={blue!50!black},
    urlcolor={blue!80!black}}


% ===========================================
% New commands here:
\newcommand{\closed}[1]{\left( #1 \right)}
\newcommand{\bracket}[1]{\left[ #1 \right]}
\newcommand{\eps}{\epsilon}
\renewcommand{\vec}{\mathbf}


% ===========================================

\graphicspath{{../output/plots/}}


\begin{document}

\title{Project 3 - FYS4150} 
\author{Nanna Bryne, Johan Mylius Kroken, Vetle A. Vikenes} 
\date{\today}                             
\noaffiliation                            

%This is how we create an abstract section.
\begin{abstract}
    We provide an overview of how to structure a scientific report. For concreteness, we consider the example of writing a report about an implementation of the midpoint rule of integration. For each section of the report we briefly discuss what the purpose of the given section is. We also provide examples of how to properly include equations, tables, algorithms, figures and references.
\end{abstract}
\maketitle


% ===========================================
\section{Introduction}
%
Write scientific introduction here 

\subsection*{Equations (temporary title)}
We will consider an ideal Penning trap, where the electric field is defined by the electric potential 
\begin{equation}
    V(x,y,z) = \frac{V_0}{2d^2}(2z^2 - x^2 - y^2), \label{eq:penning_potential}
\end{equation}
where $V_0$ is the potential applied to the electrodes, $d$ is the characteristic dimension. And ... Define equation more appropriately. 

For a single particle, the electric field is given by the negative gradient of the potential in equation \eqref{eq:penning_potential}
\begin{equation}
    \vec{E} = -\nabla V = \frac{V_0}{d^2} (x,\,y,\,-2z) \label{eq:p1_E_field_pot_gradient}
\end{equation}

A homogeneous magnetic field is imposed in the $z$-direction 
\begin{equation}
    \mathbf{B}= B_0 \hat{e}_z = (0,0,B_0). \label{eq:homogeneous_B_field}
\end{equation}

The force $\mathbf{F}$ on a single particle with charge $q$, is given by the Lorentz Force 
\begin{equation}
    \vec{F} = q \vec{E} + q \vec{v} \cross \vec{B}. \label{eq:em_force}
\end{equation}
Inserting the expressions for $\vec{E}$ and computing the cross product $\vec{v}\cross \vec{B}$ the resulting Lorentz force on the particle is 
\begin{equation} \label{eq:p1_lorentz_force_penning}
    \begin{split}
        \vec{F} &= \frac{q V_0}{d^2}(x,\,y,\,-2z) + q B_0 (\dot{y},\,-\dot{x},\,0) \\
        &= \frac{m}{2}\omega_z^2 (x,\,y,\,-2z) + m \omega_0 (\dot{y},\,-\dot{x},\,0),
    \end{split}
\end{equation} 
where we defined $\omega_0\equiv \frac{qB_0}{m}$ and $\omega_z^2 \equiv \frac{2qV_0}{m d^2}$. From Newton's second law, we have $\ddot{\vec{r}}=\vec{F}/m$. Dividing equation \eqref{eq:p1_lorentz_force_penning} by $m$, we get three equations of motion, one for each spatial component. These differential equations are 
\begin{align}
    \ddot{x} - \omega_0 \dot{y} - \frac{1}{2} \omega_z^2 x &= 0, \label{eq:p1_eom_x} \\ 
    \ddot{y} + \omega_0 \dot{x} - \frac{1}{2} \omega_z^2 y &= 0, \label{eq:p1_eom_y} \\ 
    \ddot{z} + \omega_z^2 z &= 0. \label{eq:p1_eom_z}
\end{align}
Equation \eqref{eq:p1_eom_z} has the general solution   
\begin{equation}
    z(t) = c_1 \cos(\omega_z t) + c_2 \sin(\omega_z t). \label{eq:p1_diffeq_solution_z}
\end{equation}
For this project, we will assume $q>0$. (Move this?)

Equations \eqref{eq:p1_eom_x} and \eqref{eq:p1_eom_y} are coupled. We therefore introduce a complex function $f(t)=x(t)+i y(t)$. Explaining maths. 

The single differential equation is therefore 
\begin{equation}
    \ddot{f} + i\omega_0 \dot{f} - \frac{1}{2}\omega_z^2 f = 0. \label{eq:p2_f_complex_equation}
\end{equation}
The general solution to equation \eqref{eq:p2_f_complex_equation} is 
\begin{equation}
    f(t) = A_+ e^{-i(\omega_+ t + \phi_+)} + A_- e^{-i(\omega_- t + \phi_-)}, \label{eq:p3_f_general_solution}
\end{equation}
where $\phi_+$ and $\phi_-$ are constant phases, $A_+$ and $A_-$ are positive amplitudes and 
\begin{equation}
    \omega_\pm = \frac{\omega_0 \pm \sqrt{\omega_0^2 - 2\omega_z^2}}{2}. \label{eq:p3_omega_pm}
\end{equation}




\begin{align}
    \ddot{x}_i - \omega_{0,i} \dot{y}_i - \frac{1}{2}\omega_{z,i}^2 x_i - k_e \frac{q_i}{m_i} \sum_{j\neq i} q_j \frac{x_i - x_j}{\abs{\vec{r}_i - \vec{r}_j}^3} =0, \label{eq:p4_diffeq_many_particles_x} \\
    \ddot{y}_i + \omega_{0,i} \dot{x}_i - \frac{1}{2}\omega_{z,i}^2 y_i - k_e \frac{q_i}{m_i} \sum_{j\neq i} q_j \frac{y_i - y_j}{\abs{\vec{r}_i - \vec{r}_j}^3} =0, \label{eq:p4_diffeq_many_particles_y} \\ 
    \ddot{z}_i + \omega_{z,i}^2 z_i - k_e \frac{q_i}{m_i} \sum_{j\neq i} q_j \frac{z_i - z_j}{\abs{\vec{r}_i - \vec{r}_j}^3} =0, \label{eq:p4_diffeq_many_particles_z}
\end{align}



% ===========================================
\section{Methods}\label{sec:methods}
%

% ===========================================
\subsection*{The algorithm}
%
use equation~\cite{midpoint_rule}
\begin{figure}
% NOTE: We only need \begin{figure} ... \end{figure} here because of a compatability issue between the 'revtex4-1' document class and the 'algorithm' environment.
    \begin{algorithm}[H]
    \caption{Midpoint rule for integration}
    \label{algo:midpointrule}
        \begin{algorithmic}
            \Procedure{Midpoint rule}{$f, a, b, n$}
            \State $I \leftarrow 0$        \Comment{Initialize the integral variable}
            \State $h \leftarrow (b-a)/n$  \Comment{Compute the interval length}
            \For{$i = 1, 2, \ldots, n$}
            \State $x \leftarrow a + (i-1/2)h$  \Comment{Assign $x$ to the midpoint}  %This means x is assigned the value x + ih/2.
            \State $I \leftarrow I + f(x)$  \Comment{Add contribution to integral} %Assign I to I + f(x)
            \EndFor
            \State $I \leftarrow Ih$  \Comment{Finalize the computation}
            \EndProcedure
        \end{algorithmic}
    \end{algorithm}
\end{figure}


% ===========================================
\section{Results}\label{sec:results}
%

% ===========================================
\section{Discussion}\label{sec:discussion}
%




% ===========================================
\section{Conclusion}\label{sec:conclusion}
\textit{In this section we state three things in a concise manner: what we have done, what we have found, and what should or could be done in the future.}


\onecolumngrid

%\bibliographystyle{apalike}
\bibliography{ref}


\end{document}
