\section{Introduction}\label{sec:introduction}


\textbf{Suggested introduction outline:}
\begin{enumerate}
    \item Introduce Penning Trap 
    \item Mention Lorentz force from potential and magnetic field of trap. 
    \item Single particle and multiple particles 
    \item Interactions 
    \item FE vs. RK4 
    \item Others (Trapped, ressonance, error, etc.)
    \item Project outline 
\end{enumerate}

Say we were to contain one or more charged particles in motion inside some closed volume. A natural starting point would be to create a potential whose minimum is three-dimensional. Such a quadrupole potential would take the form $V(\vec{r}) = Ax^2 + By^2 + Cy^2$, however, Laplace's equation demands $\nabla^{\! 2}V = A+ B+ C = 0$. What this means, is that a three-dimensional static field will not do. One way around this is to use a strong homogeneous axial magnetic field together with a quadrupole electric field. The former incarcerate the particle(s) radially, whereas the latter limits their axial motion. A device of such a structure is called a "Penning trap" and was first built by Hans Georg Dehmelt under the influence of Frans Michel Penning's work. \textcolor{pink}{(((smooth transition)))}


A particle with charge $q$ in an electric field $\vec{E}$ and a magnetic field $\vec{B}$ undergoes the Lorentz force $\vec{F}$ given by

\begin{equation}\label{eq:lorentz_force}
    \vec{F} = q\vec{E} + q \vec{v} \cross \vec{B},
\end{equation}
where $\vec{v}$ is the particle's velocity. Suppose the particle has mass $m$, then Newton's second law gives us the equation of motion for the particle:

\begin{equation}\label{eq:eom_lorentz}
 \dv{\vec{v}}{t} = \frac{\vec{F}}{m}  \Rightarrow \dv[2]{\vec{r}}{t}= \frac{q}{m} (\vec{E} + \vec{v} \cross \vec{B} ) 
\end{equation}

% Comment about initial conditions? No...
The external electric field is related to the potential through $\vec{E} = -\nabla V$. We now have simple equations for predicting a single particle's motion inside of a Penning trap, and we will later discuss what is required of the initial conditions in order for it to stay put. If we were to accompany the single particle with another charged particle, however, we would need to consider the contribution to the electric field from the Coulomb interactions between the two of them. Adding more charged particles complicates the system even more in such a way that we generally will struggle to solve it analytically \textcolor{blue}{(IS THIS CORRECT?)}; we need to use numerical integration methods. 

We consider two different numerical schemes for solving ordinary differential equations (ODEs)\footnote{We shall have to split the eq. \eqref{eq:eom_lorentz} in two first order ODEs.}, forward Euler (FE) and fourth order Runge-Kutta (RK4). The former is a numerical procedure of first order, whereas the latter is of fourth order, resulting in global errors proportional to the step size and the step size to the power of four, respectively. That is, say we have a step size $h\sim 0.1$, the error arising from FE is $h^{1-4} \sim 1000$ times larger than that of RK4 \textcolor{blue}{(WAIT, WHAT?)}. However, one should keep in mind that the RK4 algorithm demands many more floating points operations (FLOPs) from our computer than what the FE algorithm does.

Now being able to solve the system of several interacting charged particles in the Penning trap in question, we should ponder the effect the Coulomb interactions have on the trap's ability to contain the particles. 
\textcolor{green}{FILL WITH PONDERING}

In addition, we will investigate the effect of adding an oscillating time-dependent perturbation to the external electric potential $V$. This is indeed similar to another approach for particle storage, named the Paul trap, in which the oscillating field is used instead of the magnetic field. The combined ion trap, the combination of these two structures, allows for storage of oppositely charged particles, but is vulnerable to some ranges of oscillation frequencies and amplitudes \textcolor{blue}{(CHECK THIS)}.
% move to right after first paragraph?

\textcolor{green}{OUTLINE}



% ===========================================
