\section{Introduction}\label{sec:introduction}


\textbf{Things left to do:}
\begin{itemize}
    \item Write abstract
    \item Write conclusion
    \item Finish discussion
    \item Rewrite results, or change it up so we dont repeat ourselves. 
    \item Check figure captions (conventional writing, references etc)
    \item Read through most of the text for typos and coherence
\end{itemize}


\textbf{Suggested introduction outline:}
\begin{enumerate}
    \item Introduce Penning trap 
    \item Mention Lorentz force from potential and magnetic field of trap. 
    \item Single particle and multiple particles 
    \item Interactions 
    \item FE vs. RK4 
    \item Others (Trapped, ressonance, error, etc.)
    \item Project outline 
\end{enumerate}

Trapping charged particles is a common and useful method in order to perform various measurements and experiments. A natural starting point would be to create a three-dimensional electric potential which pulls the particles towards a center. According to Earnshaw's theorem \citep{GriffithsEarnshaw}, keeping a charged particle in stable equilibrium with an electrostatic force alone is not possible. One way around this is to use a strong homogeneous axial magnetic field together with a quadrupole electric field. The former incarcerate the particle(s) radially, whereas the latter confines the particles radially. A device of such a structure is called a "Penning trap" and was first built by Hans Georg Dehmelt under the influence of Frans Michel Penning's work \citep{Dehmelt1989}. The Lorentz force $\vec{F}$ acting on a particle with charge $q$ that is subjected to this electric field and a magnetic field, $\vec{B}$, is given by 
\begin{equation}\label{eq:lorentz_force}
    \vec{F} = q\vec{E} + q \vec{v} \cross \vec{B},
\end{equation}
where $\vec{v}$ is the particle's velocity. Suppose the particle has mass $m$, then Newton's second law gives us the equation of motion for the particle:

\begin{equation}\label{eq:eom_lorentz}
 \dv{\vec{v}}{t} = \frac{\vec{F}}{m}  \Rightarrow \dv[2]{\vec{r}}{t}= \frac{q}{m} (\vec{E} + \vec{v} \cross \vec{B} ).
\end{equation}
The external electric field is related to the potential through $\vec{E} = -\nabla V$. By choosing a particular electric potential and magnetic field, we can solve equation \eqref{eq:eom_lorentz} to predict the motion of a single particle inside a Penning trap. In reality one is often concerned with storing multiple particles, not just a single one. This introduces Coulomb interactions between the particles, and the resulting equations of motion will then be much more cumbersome to tackle analytically. To study the motion of the trapped particles we will therefore resort to numerical integration methods. 



\Vetle{Remove a lot of the shit below. Remember to discuss single particle analytical shit.}
Equation \eqref{eq:eom_lorentz} can be split into two ordinary differential equations (ODEs) that we can solve numerically. The two methods we will consider are forward Euler (FE) and fourth order Runge-Kutta (RK4). The former is a numerical procedure of first order, whereas the latter is of fourth order, resulting in global errors proportional to the step size and the step size to the power of four, respectively. That is, say we have a step size $h\sim 0.1$, \textcolor{red}{Fix following error discussion:} the error arising from FE is $h^{1-4} \sim 1000$ times larger than that of RK4 \textcolor{blue}{(WAIT, WHAT?)}. The RK4 algorithm does however demand many more floating points operations (FLOPs) than the FE algorithm, and will therefore be more computationally expensive at a given step size. However, the RK4 algorithm could be much faster than the FE algorithm, since the FE algorithm requires a much smaller step size in order to achieve a certain level of precision. We will therefore study the relative error of the two algorithms at different step sizes, by comparing with the analytical solution from a single particle inside the trap.          





\textcolor{red}{Not wuite sure about the next two paragraphs.}
Now being able to solve the system of several interacting charged particles in the Penning trap in question, we should ponder the effect the Coulomb interactions have on the trap's ability to contain the particles. 
\textcolor{green}{FILL WITH PONDERING}. 

In addition, we will investigate the effect of adding an oscillating time-dependent perturbation to the external electric potential $V$. This is indeed similar to another approach for particle storage, named the Paul trap, in which the oscillating field is used instead of the magnetic field. The combined ion trap, the combination of these two structures, allows for storage of oppositely charged particles, but is vulnerable to some ranges of oscillation frequencies and amplitudes \textcolor{blue}{(CHECK THIS)}.
% move to right after first paragraph?


In this report we begin by introducing the Penning trap in section \ref{sec:penning_trap}. We describe the electric potential and magnetic field of the trap in section \ref{subsec_methods:penning_trap_setup}. In section \ref{subsec_methods:single_particle_analytical} we find the equations of motion for a single particle in the Penning trap, where we also present the general analytical solutions. In section \ref{subsec_methods:multiple_particles} we present the equations of motion when we have multiple particles in the trap interacting with each other through the Coulomb force. 


In the following section we present our solver, explaining its main features as well as how the ODEs are solved by the two algorithms in question. After that we present our choice of parameters for the trap. In addition to this, we present initial conditions for two particles that we will use for initial testing of our solver. Using two particles, we will study their motion in a plane, their trajectories in phase space and their motion in 3D space, both with and without interactions. For one of the particles we will also study its axial motion, comparing the analytical solution to that of the two algorithms. Using our analytical results for the single particle, we will study the size of the relative error of our two algorithms as a function of time for four different choices of step size. We will use these results to estimate the error convergence rate $r_\mathrm{err}$ for the two methods. 


% ===========================================
