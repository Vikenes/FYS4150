\section{Introduction}\label{sec:introduction}
%
Write scientific introduction here 

\subsection*{Equations (temporary title)}
We will consider an ideal Penning trap, where the electric field is defined by the electric potential 
\begin{equation}
    V(x,y,z) = \frac{V_0}{2d^2}(2z^2 - x^2 - y^2), \label{eq:penning_potential}
\end{equation}
where $V_0$ is the potential applied to the electrodes, $d$ is the characteristic dimension. And ... Define equation more appropriately. 

For a single particle, the electric field is given by the negative gradient of the potential in equation \eqref{eq:penning_potential}
\begin{equation}
    \vec{E} = -\nabla V = \frac{V_0}{d^2} (x,\,y,\,-2z) \label{eq:p1_E_field_pot_gradient}
\end{equation}

A homogeneous magnetic field is imposed in the $z$-direction 
\begin{equation}
    \mathbf{B}= B_0 \hat{e}_z = (0,0,B_0). \label{eq:homogeneous_B_field}
\end{equation}

The force $\mathbf{F}$ on a single particle with charge $q$, is given by the Lorentz Force 
\begin{equation}
    \vec{F} = q \vec{E} + q \vec{v} \cross \vec{B}. \label{eq:em_force}
\end{equation}
Inserting the expressions for $\vec{E}$ and computing the cross product $\vec{v}\cross \vec{B}$ the resulting Lorentz force on the particle is 
\begin{equation} \label{eq:p1_lorentz_force_penning}
    \begin{split}
        \vec{F} &= \frac{q V_0}{d^2}(x,\,y,\,-2z) + q B_0 (\dot{y},\,-\dot{x},\,0) \\
        &= \frac{m}{2}\omega_z^2 (x,\,y,\,-2z) + m \omega_0 (\dot{y},\,-\dot{x},\,0),
    \end{split}
\end{equation} 
where we defined $\omega_0\equiv \frac{qB_0}{m}$ and $\omega_z^2 \equiv \frac{2qV_0}{m d^2}$. From Newton's second law, we have $\ddot{\vec{r}}=\vec{F}/m$. Dividing equation \eqref{eq:p1_lorentz_force_penning} by $m$, we get three equations of motion, one for each spatial component. These differential equations are 
\begin{align}
    \ddot{x} - \omega_0 \dot{y} - \frac{1}{2} \omega_z^2 x &= 0, \label{eq:p1_eom_x} \\ 
    \ddot{y} + \omega_0 \dot{x} - \frac{1}{2} \omega_z^2 y &= 0, \label{eq:p1_eom_y} \\ 
    \ddot{z} + \omega_z^2 z &= 0. \label{eq:p1_eom_z}
\end{align}
Equation \eqref{eq:p1_eom_z} has the general solution   
\begin{equation}
    z(t) = c_1 \cos(\omega_z t) + c_2 \sin(\omega_z t). \label{eq:p1_diffeq_solution_z}
\end{equation}
For this project, we will assume $q>0$. (Move this?)

Equations \eqref{eq:p1_eom_x} and \eqref{eq:p1_eom_y} are coupled. We therefore introduce a complex function $f(t)=x(t)+i y(t)$ whose time derivatives are:
\begin{align}
    \dot{f} &= \dot{x}+i\dot{y} \\
    \ddot{f} &= \ddot{x} + i\ddot{y}.
\end{align}
We then multiply equation \ref{eq:p1_eom_y} with $i$ and add to equation \ref{eq:p1_eom_x} to obtain:
\begin{align}
    \ddot{x} + i\ddot{y} + \omega_0(i\dot{x} - \dot{y}) -\frac{1}{2}\omega_z^2(x+iy) &= 0 \nonumber \\
    (\ddot{x} + i\ddot{y}) + i\omega_0(\dot{x} +i \dot{y}) -\frac{1}{2}\omega_z^2(x+iy) &= 0. \label{eq:p2_eom_rewritten}
\end{align}
The single differential equation in terms of the complex function $f(t)$ is therefore: 
\begin{equation}
    \ddot{f} + i\omega_0 \dot{f} - \frac{1}{2}\omega_z^2 f = 0. \label{eq:p2_f_complex_equation}
\end{equation}
The general solution to equation \eqref{eq:p2_f_complex_equation} is 
\begin{equation}
    f(t) = A_+ e^{-i(\omega_+ t + \phi_+)} + A_- e^{-i(\omega_- t + \phi_-)}, \label{eq:p3_f_general_solution}
\end{equation}
where $\phi_+$ and $\phi_-$ are constant phases, $A_+$ and $A_-$ are positive amplitudes and 
\begin{equation}
    \omega_\pm = \frac{\omega_0 \pm \sqrt{\omega_0^2 - 2\omega_z^2}}{2}. \label{eq:p3_omega_pm}
\end{equation}
We see from equation  \ref{eq:p3_f_general_solution} that if $\omega_\pm$ is real, then $\abs{f(t)} < \infty$ for $t\to\infty$ and we have a bounded solution in the $xy$-plane. For $\omega_\pm$ to be real we see from equation \ref{eq:p3_omega_pm} that the following must hold:
\begin{align}
    \omega_0^2 &\geq 2\omega_z^2  \\
    \implies \frac{q}{m} &\geq \frac{4V_0}{(B_0d)^2}.
\end{align} 
We may regard equation \ref{eq:p3_f_general_solution} as a sum of vectors $\vec{r}_\pm$ with amplitudes $r_\pm=A_\pm$ and directions given by the angles $\alpha_\pm = \omega_\pm t + \phi_\pm$ in the complex plane (representing the $xy$-plane). The magnitude $\abs{f(t)} = \abs{f}$ for this sum of vectors is then given by:
\begin{align}
    \abs{f} = \sqrt{\abs{\vec{r}_+}^2 + \abs{\vec{r}_-}^2 + 2\abs{\vec{r}_+}\abs{\vec{r_-}}\cos{(\abs{\alpha_+-\alpha_-})}}.
\end{align}








The maximum distance from the origin, $R_+$, occurs when the two vectors are pointing in the same direction, i.e. $\abs{\alpha_+-\alpha_-} = 0 \implies \cos0=1$, which yields:
\begin{align}
    R_+ &= \sqrt{A_+^2 + A_-^2 + 2A_+A_-} = \sqrt{(A_+ + A_-)^2} \nonumber \\
    &= A_+ + A_-
\end{align}
Similarily, the minimum distance, $R_-$, from the origin occurs when the vectors are pointing in opposite directions, i.e. $\abs{\alpha_+-\alpha_-} = \pi \implies \cos\pi=-1$, which yields:
\begin{align}
    R_- &= \sqrt{A_+^2 + A_-^2 - 2A_+A_-} = \sqrt{(A_+ - A_-)^2} \nonumber \\
    &= \abs{A_+ - A_-}
\end{align}




\begin{align}
    \ddot{x}_i - \omega_{0,i} \dot{y}_i - \frac{1}{2}\omega_{z,i}^2 x_i - k_e \frac{q_i}{m_i} \sum_{j\neq i} q_j \frac{x_i - x_j}{\abs{\vec{r}_i - \vec{r}_j}^3} =0, \label{eq:p4_diffeq_many_particles_x} \\
    \ddot{y}_i + \omega_{0,i} \dot{x}_i - \frac{1}{2}\omega_{z,i}^2 y_i - k_e \frac{q_i}{m_i} \sum_{j\neq i} q_j \frac{y_i - y_j}{\abs{\vec{r}_i - \vec{r}_j}^3} =0, \label{eq:p4_diffeq_many_particles_y} \\ 
    \ddot{z}_i + \omega_{z,i}^2 z_i - k_e \frac{q_i}{m_i} \sum_{j\neq i} q_j \frac{z_i - z_j}{\abs{\vec{r}_i - \vec{r}_j}^3} =0, \label{eq:p4_diffeq_many_particles_z}
\end{align}



% ===========================================
