\section{Discussion}\label{sec:discussion}


<<<<<<< HEAD

\subsection{Adding a time-dependent perturbation to the potential}

\par - Why these frequencies?
\par - Effect of increasing amplitude
\par - Effect of Coulomb interactions
\par - Comment on number of time steps

From Figure \ref{fig:trapped_without} we see how an oscillating electric potential gives rise to the escapement of particles in the Penning trap. For some bands of applied frequencies, basically all particles have escaped before the simulation has ended. Increasing the magnitude of the perturbation has the effect of broadening these band widths, as well as to make them occur more frequently. The largest bands for which the simulations become unstable are ???

Without interactions, it seems to be the case that either all particles escape or stay trapped, referring to the steep graphs in Figures \ref{fig:trapped_without} and \ref{fig:trapped_with}, but especially the latter as we see how it differs from the graph representing simulations with interacting particles. Interacting particles in a sense push the resonance frequencies upwards: they demand higher frequencies in order for all to escape, but as the non-interacting particles are trapped again, the interacting particles are only partly trapped. 



=======
>>>>>>> b1563a9c6a66302cac3992a3e75008c9bc0105bc
% ===========================================