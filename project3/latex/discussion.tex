\section{Discussion}\label{sec:discussion}

\subsection{Single particle}

We provide the movement in the $z$-direction in Figure \ref{fig:compare_z_analytical} as some sort of test to see whether the numerical integrations methods works or not. Hence, we use the fewest number of time steps, $n_1$ in order reveal the largest potential errors. From the figure they seem to overlap perfectly, however there is reason to think the error in the calculated solutions grow with time. This is indeed the case, as the error for the small duration $T_1$ is not visible due to the thickness of plotted lines and the scale of the plot. Although not shown here, if we were to zoom in on the graphs at a late time they do not overlap, i.e. indicating that there is an error although hard to spot. Nevertheless, the calculated solution seems to fit the analytical quite well, but we are still interested in quantifying the error, which we expect to be smaller if we increase the number of time steps. 

Using the analytical solution as reference, the relative error as found from equation \ref{eq:relative_error_size} is shown in Figure \ref{fig:error_FE} for the FE scheme and in Figure \ref{fig:error_RK} for the RK4 scheme. Here, all four time steps $n_k, k\in[1,4]$ is included. The increase in error for both plots as we increase the number of time steps seem to follow the order of the global error for each scheme. This also becomes apparent when considering the actual relative error in the plots ($y$-axis values), and the magnitudinal decrease in error when we increase the number of time steps. The latter is significantly more apparent for the RK4 scheme, which is to be expected as the global error should go as $\mathcal{O}(h^4)$, rather than $\mathcal{O}(h)$ for the FE scheme. There is a noticeable increase in the widths of the different graphs for the RK4 scheme in Figure \ref{fig:error_RK}. This can be explained by the small numerical values of the relative error, and the fact that we use a logarithmic $y$-axis which emphasize the variation for small errors (broad fuzzy looking graph for $n_4$).


\subsection{Two particles}


In the simulations where we use two noninteracting particles, the trajectories are easily predictable as they correspond to two single-particle cases. The movement in the $xy$-plane in this case is shown in the left panel of Figure \ref{fig:xy_two_particles}, whereas the interacting particles' orbital motion is presented in the right panel. The noninteracting particles move in the radial plane according to two modes \Nanna{rephrase}, one with frequency $\omega_+$ orbiting around the magnetic field lines and another with frequency $\omega_-$ that gives rise to the slow orbit around the centre of the trap. The former is called the modified cyclotron motion and the latter is the magnetron motion. The interacting particles show similar tendencies, but the oribiting magnetron motion is shifted a bit in the radial plane due to the particles' repulsion between each other. The radii said orbits tend to increase in both cases, although not very obvoiusly from the figure for Particle 2. The other mode, the modified cyclotron motion, is affected oppositely by the interaction forces for the two particels. The orbit Particle 1 has around the magnetic field lines generally seems to increase in radius where the reciprocal radius for Particle 2 decreases, and vice versa.

\Nanna{Comment about magnetron motion being more unstable the the mod. cyclotron motion?} 

We present the phase plots in the $x$-plane in Figure \ref{fig:x_phase_two_particles}, in which the left panel indicates the periodic motion in one of the radial directions when there are no interaction forces present. The elliptical phases have for both particles a more or less constant size, but move around in the phase space as each particle moves along the $x$-axis. The former phenomena arise from the modified cyclotron motion, whereas the latter is due to the slow magnetron motion. The same nature is seen for the other radial direction, although we do not present the phase plot in the $y$-plane here. Coulomb interactions seem to especially affect the slow orbital motion around the centre as the periodicity of the ellipses in the phase space is quickly reduced \Nanna{rephrase?}. However, after a while, the phase space ellipses change significantly in size as well. This is consistent with what we see in Figure \ref{fig:xy_two_particles}.  

The left panel in Figure \ref{fig:z_phase_two_particles} tells us simply that the axial motion is a harmonic oscillator \Nanna{comment about frequency $\omega_z$?} when the particles are not interacting. In the right panel of the same figure, we see how the repulsion between them affects this oscillating movement. The interacting particles are moving up and down along th $z$-axis with varying ranges and frequencies. We notice that the particles are still confined to a small length in the axial direction. 

The three-dimensional trajectories in Figure \ref{fig:3d_trajectory} substantiate the discussion of periodicity and quasiperiodicity for noninteracting and interacting particles, respectively. In summary, introducing Coulomb forces gives rise to more chaotic and possible unstable motions, for which the original magnetron motion seems especially prone to instabilities.


\Nanna{NOTE: magnetron motion is unstable because energy ass. w this movement is negative, and collisions and perturbations tend to make radius increase (only write if properly discussed and/or cited)}



\subsection{Many particles}

\par - Why these frequencies? 
\par - Effect of increasing amplitude \Nanna{OK i guess}
\par - Effect of Coulomb interactions \Nanna{I think so}
\par - Comment on number of time steps \Nanna{Written, but have some inquiries}

From Figure \ref{fig:trapped_without} we see how an oscillating electric potential gives rise to the escapement of particles in the Penning trap. For some bands of applied frequencies, basically all particles have escaped before the simulation has ended. Increasing the magnitude of the perturbation has the effect of broadening these band widths, as well as to make them occur more frequently. The main bands for which the simulations become unstable are located around $\omega_V \sim 1.4$ MHz, the most obvoius in the figure, and $\omega_V \sim 2.2$ MHz, where it is not obvoius for the smallest amplitude. There is also a good response at $\omega_V\sim 0.7$ MHz, but this is only present for the two larger amplitudes.

Without interactions, it seems to be the case that either all particles escape or stay trapped, referring to the steep graphs in Figures \ref{fig:trapped_without} and \ref{fig:trapped_with}, but especially the latter as we see how it differs from the graph representing simulations with interacting particles. In this figure, interacting particles in a sense push the resonance frequencies upwards: they demand higher frequencies in order for all to escape, but as the noninteracting particles are trapped again, the interacting particles are only partly trapped. We believe that a somewhat broader range of frequencies would show us that very rarely will interacting particles stay trapped here after a period $T_2$, due to instabilities in the magnetron motion arisen by the interaction forces. The steep part of the graph connected with such particles in Figure \ref{fig:trapped_with} is, as mentioned, located at higher frequencies than for the noninteracting particles, which can be due to difference in the eigenfrequencies between these motions.

The time step size is relatively large for the simulations with Coulomb interactions turned on. This could be a source of instabilities, but as these simulations are computationally expensive, we do not explore this theory any further. Referring to our previous error discussion, the relative error arising from the simulations without interacting particles should be around $10^{-6}$, right above the error for $n_1$ in Figure \ref{fig:error_RK}, for a period $T_1$. Assuming an increase rate as in this figure, we should not have an error much larger than $\sim 10^{-4}$ after a time $T_2$, but this is based on our visual estimate. \Nanna{Refer to convergence rate? Comment on the effect of extra forces?}


% ==========================================