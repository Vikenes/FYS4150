\section{Discussion}\label{sec:discussion}

\subsection{Single particle}

We provide the movement in the $z$-direction in Figure \ref{fig:compare_z_analytical} as some sort of test to see whether the numerical integrations methods works or not. Hence, we use the fewest number of time steps, $n_1$ in order reveal the largest potential errors. From the figure they seem to overlap perfectly, however there is reason to think the errors in the calculated solutions grow with time. This is indeed the case, as the error for the small duration $T_1$ is not visible due to the thickness of plotted lines and the scale of the plot. Although not shown here, if we were to zoom in on the graphs at a late time they do not overlap, i.e. indicating that there is an error although hard to spot. Nevertheless, the calculated solution seems to fit the analytical quite well, but we are still interested in quantifying the error, which we expect to be smaller if we increase the number of time steps. 

Using the analytical solution as reference, the relative error as found from equation \eqref{eq:relative_error_size} is shown in Figure \ref{fig:error_FE} for the FE scheme and in Figure \ref{fig:error_RK} for the RK4 scheme. Here, all four time steps $n_k$ with $k=1,2,3,4$ are included. The increase in error for both plots as we increase the number of time steps seems to follow the order of the global error for each scheme. This also becomes apparent when considering the actual relative error in the plots ($y$-axis values), and the magnitudinal decrease in error when we increase the number of time steps. The latter is significantly more apparent for the RK4 scheme, which is to be expected as the global error should go as $\mathcal{O}(h^4)$, rather than $\mathcal{O}(h)$ for the FE scheme. This coincides with the values of $r_\mathrm{err}$ we previously estimated. The RK4 method converges much faster towards the actual solution. Thus, increasing the number of time steps results in a much larger change in relative error than for the FE method. The apparent large error variations of the graphs in Figure \ref{fig:error_RK} can be explained by the small numerical values of the relative error, and the fact that we use a logarithmic $y$-axis, which emphasize the variation for small errors (broad fuzzy looking graph for $n_4$). 


\subsection{Two particles}


In the simulations where we use two noninteracting particles, the trajectories are easily predictable as they correspond to two single-particle cases. The movement in the $(x,y)$-plane in this case is shown in the left panel of Figure \ref{fig:xy_two_particles}, whereas the interacting particles' orbital motion is presented in the right panel. The noninteracting particles move in the radial plane according to two modes, one with frequency $\omega_+$ orbiting around the magnetic field lines and another with frequency $\omega_-$ that gives rise to the slow orbit around the centre of the trap. The former is called the modified cyclotron motion and the latter is the magnetron motion. The interacting particles show similar tendencies, but the oribiting magnetron motion is shifted a bit in the radial plane due to the particles' repulsion between each other. The radii of said orbits tend to increase in both cases, although not very obvoiusly from the figure for Particle 2. The other mode, the modified cyclotron motion, is affected oppositely by the interaction forces for the two particels. The orbit Particle 1 has around the magnetic field lines generally seems to increase in radius where the reciprocal radius for Particle 2 decreases, and vice versa.

We present the phase plots in the $x$-plane in Figure \ref{fig:x_phase_two_particles}, in which the left panel indicates the periodic motion in one of the radial directions when there are no interaction forces present. The elliptical phases have for both particles a more or less constant size, but move around in the phase space as each particle moves along the $x$-axis. The former phenomena arise from the modified cyclotron motion, whereas the latter is due to the slow magnetron motion. The same nature is seen for the other radial direction, although we do not present the phase plot in the $y$-plane here. Coulomb interactions seem to especially affect the slow orbital motion around the centre as the periodicity of the ellipses in the phase space is quickly reduced. However, after a while, the phase space ellipses change significantly in size as well. This is consistent with what we see in Figure \ref{fig:xy_two_particles}.  

The left panel in Figure \ref{fig:z_phase_two_particles} tells us simply that the axial motion is a harmonic oscillator. In the right panel of the same figure, we see how the repulsion between them affects this oscillating movement. The interacting particles are moving up and down along the $z$-axis with varying ranges and frequencies. We notice that the particles are still confined to a small length in the axial direction. 

The three-dimensional trajectories in Figure \ref{fig:3d_trajectory} substantiate the discussion of periodicity and quasiperiodicity for noninteracting and interacting particles, respectively. In summary, introducing Coulomb forces gives rise to more chaotic and possibly unstable motions, for which the original magnetron motion seems especially prone to instabilities.




\subsection{Resonance}


From Figure \ref{fig:trapped_without} we see how an oscillating electric potential gives rise to the \textit{escapement} of particles in the Penning trap. For some bands of applied frequencies, all particles have escaped before the simulation has ended. Increasing the amplitude of the perturbation has two main effects. The first effect is that for resonance occuring near a particular frequency, a broader range of neighbouring frequencies also produce resonance effects. This is apparent for $\omega_V\sim1.4\pm\Delta\omega_V\,\mathrm{MHz}$, where resonance is evident for $\Delta\omega_V\sim0.2\,\mathrm{MHz}$ when $f=0.7$. For $f=0.1$, Figure \ref{fig:trapped_with} indicates that $\Delta\omega_V\lesssim0.05\,\mathrm{MHz}$ in comparison. The other main effect is that the fraction of particles escaping at certain frequencies increases. At $\omega_V\sim2.2\mathrm{MHz}$ we se from Figure \ref{fig:trapped_without} that $f=0.1$ and $f=0.7$ yields $N_\mathrm{trapped}/N_\mathrm{p}\sim0.2$ and $N_\mathrm{trapped}/N_\mathrm{p}=0$, respectively. At $\omega_V\sim0.7\,\mathrm{MHz}$ we see that $f=0.1$ is insufficient for particles to escape, and for $\omega_V<0.5\,\mathrm{MHz}$ particles escape when $f=0.7$ only. 

When it comes to the frequency dependence of resonance behaviour there may be numerous reasons why the particles in the Penning trap is particularly vulnurable to some specific frequencies. We will only present some of the plausible explanations. A thorough discussion of resonance phenomena is beyond the scope of this report. 

The particles' motion in the Penning trap have some quasi-periodic features. For a single particle, we know that the axial motion is governed by oscillations with a frequency of $\omega_z$. For a harmonic oscillator, the resonance frequency due to an applied oscillating force will be close its eigenfrequency. Our perturbation alters the differential equation for $z$, however, since the applied oscillation is multiplied with $z$ in equation \eqref{eq:p1_eom_z}. Nonetheless, it's reasonable to assume that $\omega_V\sim\omega_z$ induces resonance, which would correspond to the drops at $\omega_V\sim0.7\,\mathrm{MHz}$ in figure \ref{fig:trapped_without}. Similarly for the movement in the $(x,y)-$plane, we may expect that certain values of $\omega_V$ expressed as different compinations of $\omega_+,\,\omega_-$ and $\omega_z$ result in the particles being pushed further out in the plane, causing them to eventually escape the trap. $\omega_+$ and $\omega_-$ governed the periodic motions in the $(x,y)-$plane for the static potential. Although these frequencies are altered by the time-dependent potential, we find it reasonable to assume that a somewhat similar quasi-periodic behaviour may occur in the time-dependent potential trap, with some corresponding new frequency values. It is then likely that for $\omega_V$ close to these new frequency values, resonance will occur. 


Without any further investigations, we can only speculate why the resonance frequencies have their particular values. One major drawback of our analysis is that we only sampled $N_\mathrm{p}=100$ random particles once. The resonance frequencies we see in Figure \ref{fig:trapped_without} are therefore not necessarily the true resonance frequencies, as other initial particle configurations will be affected differently by the applied frequencies in question. Computing the resulting average of multiple initial configurations would reduce the statistical uncertainties, and is a natural starting point for further analysis. We can then estimate the true resonance frequencies with higher accuracy. One possible way of determining what actually sets the value of the resonance frequencies, is to repeat the forementioned analysis several times, using different parameters for the Penning trap and different particles each time. If resonance occurs at frequencies that are different combinations of $\omega_+,\,\omega_-$ and $\omega_z$, one may study whether the numerically estimated resonance frequencies for different setups concur with analytical expressions. The drawback of this method is that 
running multiple scans to explore resonance frequencies is computationally expensive, even for a single Penning trap. Repeating this for many different trap setups will not necesarilly be the most effective approach. One method of gaining more insight into which parameters govern resonance at different frequencies, is to consider the trap limits in $(x,y,z)-$space, and count the number of particles escaping in different directions as a function of applied frequencies. The positions at which particles escaped are unknown to us from our method of choice.     

Without interactions, Figure \ref{fig:trapped_without} indicates that there are sharp transitions in applied frequencies where either all particles escpae or they all remain trapped. This may be a consequence of considering one sample only for our analysis, and could be resolved with multiple samples. Another possible reason for this, is that our step size of $\omega_V$ is too large. In Figure \ref{fig:trapped_with} we see that a higher frequency resolution do yield less sharp transitions, but not by a significant amount. 

When particle interactions are included, we see from Figure \ref{fig:trapped_with} that many particles continues to escape at the high frequency end, where we previously had no particles escaping. This is likely due to repulsive Coulomb forces pushing some particles further out from the origin. A surprising feature is that for the low frequency end, a larger fraction of the interacting particles remain trapped at frequencies where the non-interacting particles escapes. The reason for this is not apparent, and additional sampling should be considered to determine if this effect is statistically significant.     

The time step size is relatively large for the simulations with Coulomb interactions turned on. This could be a source of instabilities, but as these simulations are computationally expensive, we do not explore potential consequences of this any further. 


% ==========================================