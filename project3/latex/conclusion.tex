\section{Conclusion}\label{sec:conclusion}

We conclude that our implementation and testing of both the FE ad RK4 scheme is correct since the solutions appear accurate in Figure \ref{fig:compare_z_analytical}, and the errors from Figures \ref{fig:error_FE} and \ref{fig:error_RK} behave as we expect them to. We also conclude that the convergence rates found for both methods are satisfying as they resemble their numerical order. 


For the two particle simulation, allowing them to interact pushes the particles further away from each other, while both still show periodic behaviour. This is the physical solution as we are not able to switch off these interaction in a physical Penning trap. The effect we see when including the Coulomb forces is a more chaotic behaviour. 

Introducing a time-dependent perturbation to the electric potential does indeed give rise to resonance phenomena, as certain resonance frequencies empty the trap completely by pushing all particles out of it. We find these frequencies to be bands whose widths are determined by the applied amplitudes $f$ and centred around $\omega_V\in\{0.7, 1.4, 2.2\}$. Without particle interactions, the majority of applied frequencies seem to either leave all the particles contained within the trap, or eject them all. When we include Coulomb interactions, there seems to be slight middle ground where some have escaped and some are still trapped. Thus particles interactions seem to sometimes work against the resonance frequencies, preventing some particles from escaping the trap as quickly as they otherwise would. 

We also question the validity of these results due to the larger time step used in order to speed to computation time. A natural extension of this would be to run the simulation for smaller time steps, although this will increase computation time. Another approach could be to look at a broader range of $\omega_V$ values, with smaller step sized. To better understand which frequencies the resonance behaviour is related to, a natural choice of further exploration is to study the spatial distribution of the escaped particles. By also defining the extent of the trap in terms of spatial coordinates, rather than a fixed radius, resonance effects in different directions can be better understood. It would also be interesting to implement the self induced magnetic field set up by moving charges to see whether it has a major impact on our results. 

% ===========================================